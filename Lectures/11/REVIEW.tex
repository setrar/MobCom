\documentclass[11pt]{article}

    \usepackage[breakable]{tcolorbox}
    \usepackage{parskip} % Stop auto-indenting (to mimic markdown behaviour)
    

    % Basic figure setup, for now with no caption control since it's done
    % automatically by Pandoc (which extracts ![](path) syntax from Markdown).
    \usepackage{graphicx}
    % Keep aspect ratio if custom image width or height is specified
    \setkeys{Gin}{keepaspectratio}
    % Maintain compatibility with old templates. Remove in nbconvert 6.0
    \let\Oldincludegraphics\includegraphics
    % Ensure that by default, figures have no caption (until we provide a
    % proper Figure object with a Caption API and a way to capture that
    % in the conversion process - todo).
    \usepackage{caption}
    \DeclareCaptionFormat{nocaption}{}
    \captionsetup{format=nocaption,aboveskip=0pt,belowskip=0pt}

    \usepackage{float}
    \floatplacement{figure}{H} % forces figures to be placed at the correct location
    \usepackage{xcolor} % Allow colors to be defined
    \usepackage{enumerate} % Needed for markdown enumerations to work
    \usepackage{geometry} % Used to adjust the document margins
    \usepackage{amsmath} % Equations
    \usepackage{amssymb} % Equations
    \usepackage{textcomp} % defines textquotesingle
    % Hack from http://tex.stackexchange.com/a/47451/13684:
    \AtBeginDocument{%
        \def\PYZsq{\textquotesingle}% Upright quotes in Pygmentized code
    }
    \usepackage{upquote} % Upright quotes for verbatim code
    \usepackage{eurosym} % defines \euro

    \usepackage{iftex}
    \ifPDFTeX
        \usepackage[T1]{fontenc}
        \IfFileExists{alphabeta.sty}{
              \usepackage{alphabeta}
          }{
              \usepackage[mathletters]{ucs}
              \usepackage[utf8x]{inputenc}
          }
    \else
        \usepackage{fontspec}
        \usepackage{unicode-math}
    \fi

    \usepackage{fancyvrb} % verbatim replacement that allows latex
    \usepackage{grffile} % extends the file name processing of package graphics
                         % to support a larger range
    \makeatletter % fix for old versions of grffile with XeLaTeX
    \@ifpackagelater{grffile}{2019/11/01}
    {
      % Do nothing on new versions
    }
    {
      \def\Gread@@xetex#1{%
        \IfFileExists{"\Gin@base".bb}%
        {\Gread@eps{\Gin@base.bb}}%
        {\Gread@@xetex@aux#1}%
      }
    }
    \makeatother
    \usepackage[Export]{adjustbox} % Used to constrain images to a maximum size
    \adjustboxset{max size={0.9\linewidth}{0.9\paperheight}}

    % The hyperref package gives us a pdf with properly built
    % internal navigation ('pdf bookmarks' for the table of contents,
    % internal cross-reference links, web links for URLs, etc.)
    \usepackage{hyperref}
    % The default LaTeX title has an obnoxious amount of whitespace. By default,
    % titling removes some of it. It also provides customization options.
    \usepackage{titling}
    \usepackage{longtable} % longtable support required by pandoc >1.10
    \usepackage{booktabs}  % table support for pandoc > 1.12.2
    \usepackage{array}     % table support for pandoc >= 2.11.3
    \usepackage{calc}      % table minipage width calculation for pandoc >= 2.11.1
    \usepackage[inline]{enumitem} % IRkernel/repr support (it uses the enumerate* environment)
    \usepackage[normalem]{ulem} % ulem is needed to support strikethroughs (\sout)
                                % normalem makes italics be italics, not underlines
    \usepackage{soul}      % strikethrough (\st) support for pandoc >= 3.0.0
    \usepackage{mathrsfs}
    

    
    % Colors for the hyperref package
    \definecolor{urlcolor}{rgb}{0,.145,.698}
    \definecolor{linkcolor}{rgb}{.71,0.21,0.01}
    \definecolor{citecolor}{rgb}{.12,.54,.11}

    % ANSI colors
    \definecolor{ansi-black}{HTML}{3E424D}
    \definecolor{ansi-black-intense}{HTML}{282C36}
    \definecolor{ansi-red}{HTML}{E75C58}
    \definecolor{ansi-red-intense}{HTML}{B22B31}
    \definecolor{ansi-green}{HTML}{00A250}
    \definecolor{ansi-green-intense}{HTML}{007427}
    \definecolor{ansi-yellow}{HTML}{DDB62B}
    \definecolor{ansi-yellow-intense}{HTML}{B27D12}
    \definecolor{ansi-blue}{HTML}{208FFB}
    \definecolor{ansi-blue-intense}{HTML}{0065CA}
    \definecolor{ansi-magenta}{HTML}{D160C4}
    \definecolor{ansi-magenta-intense}{HTML}{A03196}
    \definecolor{ansi-cyan}{HTML}{60C6C8}
    \definecolor{ansi-cyan-intense}{HTML}{258F8F}
    \definecolor{ansi-white}{HTML}{C5C1B4}
    \definecolor{ansi-white-intense}{HTML}{A1A6B2}
    \definecolor{ansi-default-inverse-fg}{HTML}{FFFFFF}
    \definecolor{ansi-default-inverse-bg}{HTML}{000000}

    % common color for the border for error outputs.
    \definecolor{outerrorbackground}{HTML}{FFDFDF}

    % commands and environments needed by pandoc snippets
    % extracted from the output of `pandoc -s`
    \providecommand{\tightlist}{%
      \setlength{\itemsep}{0pt}\setlength{\parskip}{0pt}}
    \DefineVerbatimEnvironment{Highlighting}{Verbatim}{commandchars=\\\{\}}
    % Add ',fontsize=\small' for more characters per line
    \newenvironment{Shaded}{}{}
    \newcommand{\KeywordTok}[1]{\textcolor[rgb]{0.00,0.44,0.13}{\textbf{{#1}}}}
    \newcommand{\DataTypeTok}[1]{\textcolor[rgb]{0.56,0.13,0.00}{{#1}}}
    \newcommand{\DecValTok}[1]{\textcolor[rgb]{0.25,0.63,0.44}{{#1}}}
    \newcommand{\BaseNTok}[1]{\textcolor[rgb]{0.25,0.63,0.44}{{#1}}}
    \newcommand{\FloatTok}[1]{\textcolor[rgb]{0.25,0.63,0.44}{{#1}}}
    \newcommand{\CharTok}[1]{\textcolor[rgb]{0.25,0.44,0.63}{{#1}}}
    \newcommand{\StringTok}[1]{\textcolor[rgb]{0.25,0.44,0.63}{{#1}}}
    \newcommand{\CommentTok}[1]{\textcolor[rgb]{0.38,0.63,0.69}{\textit{{#1}}}}
    \newcommand{\OtherTok}[1]{\textcolor[rgb]{0.00,0.44,0.13}{{#1}}}
    \newcommand{\AlertTok}[1]{\textcolor[rgb]{1.00,0.00,0.00}{\textbf{{#1}}}}
    \newcommand{\FunctionTok}[1]{\textcolor[rgb]{0.02,0.16,0.49}{{#1}}}
    \newcommand{\RegionMarkerTok}[1]{{#1}}
    \newcommand{\ErrorTok}[1]{\textcolor[rgb]{1.00,0.00,0.00}{\textbf{{#1}}}}
    \newcommand{\NormalTok}[1]{{#1}}

    % Additional commands for more recent versions of Pandoc
    \newcommand{\ConstantTok}[1]{\textcolor[rgb]{0.53,0.00,0.00}{{#1}}}
    \newcommand{\SpecialCharTok}[1]{\textcolor[rgb]{0.25,0.44,0.63}{{#1}}}
    \newcommand{\VerbatimStringTok}[1]{\textcolor[rgb]{0.25,0.44,0.63}{{#1}}}
    \newcommand{\SpecialStringTok}[1]{\textcolor[rgb]{0.73,0.40,0.53}{{#1}}}
    \newcommand{\ImportTok}[1]{{#1}}
    \newcommand{\DocumentationTok}[1]{\textcolor[rgb]{0.73,0.13,0.13}{\textit{{#1}}}}
    \newcommand{\AnnotationTok}[1]{\textcolor[rgb]{0.38,0.63,0.69}{\textbf{\textit{{#1}}}}}
    \newcommand{\CommentVarTok}[1]{\textcolor[rgb]{0.38,0.63,0.69}{\textbf{\textit{{#1}}}}}
    \newcommand{\VariableTok}[1]{\textcolor[rgb]{0.10,0.09,0.49}{{#1}}}
    \newcommand{\ControlFlowTok}[1]{\textcolor[rgb]{0.00,0.44,0.13}{\textbf{{#1}}}}
    \newcommand{\OperatorTok}[1]{\textcolor[rgb]{0.40,0.40,0.40}{{#1}}}
    \newcommand{\BuiltInTok}[1]{{#1}}
    \newcommand{\ExtensionTok}[1]{{#1}}
    \newcommand{\PreprocessorTok}[1]{\textcolor[rgb]{0.74,0.48,0.00}{{#1}}}
    \newcommand{\AttributeTok}[1]{\textcolor[rgb]{0.49,0.56,0.16}{{#1}}}
    \newcommand{\InformationTok}[1]{\textcolor[rgb]{0.38,0.63,0.69}{\textbf{\textit{{#1}}}}}
    \newcommand{\WarningTok}[1]{\textcolor[rgb]{0.38,0.63,0.69}{\textbf{\textit{{#1}}}}}


    % Define a nice break command that doesn't care if a line doesn't already
    % exist.
    \def\br{\hspace*{\fill} \\* }
    % Math Jax compatibility definitions
    \def\gt{>}
    \def\lt{<}
    \let\Oldtex\TeX
    \let\Oldlatex\LaTeX
    \renewcommand{\TeX}{\textrm{\Oldtex}}
    \renewcommand{\LaTeX}{\textrm{\Oldlatex}}
    % Document parameters
    % Document title
    \title{REVIEW}
    
    
    
    
    
    
    
% Pygments definitions
\makeatletter
\def\PY@reset{\let\PY@it=\relax \let\PY@bf=\relax%
    \let\PY@ul=\relax \let\PY@tc=\relax%
    \let\PY@bc=\relax \let\PY@ff=\relax}
\def\PY@tok#1{\csname PY@tok@#1\endcsname}
\def\PY@toks#1+{\ifx\relax#1\empty\else%
    \PY@tok{#1}\expandafter\PY@toks\fi}
\def\PY@do#1{\PY@bc{\PY@tc{\PY@ul{%
    \PY@it{\PY@bf{\PY@ff{#1}}}}}}}
\def\PY#1#2{\PY@reset\PY@toks#1+\relax+\PY@do{#2}}

\@namedef{PY@tok@w}{\def\PY@tc##1{\textcolor[rgb]{0.73,0.73,0.73}{##1}}}
\@namedef{PY@tok@c}{\let\PY@it=\textit\def\PY@tc##1{\textcolor[rgb]{0.24,0.48,0.48}{##1}}}
\@namedef{PY@tok@cp}{\def\PY@tc##1{\textcolor[rgb]{0.61,0.40,0.00}{##1}}}
\@namedef{PY@tok@k}{\let\PY@bf=\textbf\def\PY@tc##1{\textcolor[rgb]{0.00,0.50,0.00}{##1}}}
\@namedef{PY@tok@kp}{\def\PY@tc##1{\textcolor[rgb]{0.00,0.50,0.00}{##1}}}
\@namedef{PY@tok@kt}{\def\PY@tc##1{\textcolor[rgb]{0.69,0.00,0.25}{##1}}}
\@namedef{PY@tok@o}{\def\PY@tc##1{\textcolor[rgb]{0.40,0.40,0.40}{##1}}}
\@namedef{PY@tok@ow}{\let\PY@bf=\textbf\def\PY@tc##1{\textcolor[rgb]{0.67,0.13,1.00}{##1}}}
\@namedef{PY@tok@nb}{\def\PY@tc##1{\textcolor[rgb]{0.00,0.50,0.00}{##1}}}
\@namedef{PY@tok@nf}{\def\PY@tc##1{\textcolor[rgb]{0.00,0.00,1.00}{##1}}}
\@namedef{PY@tok@nc}{\let\PY@bf=\textbf\def\PY@tc##1{\textcolor[rgb]{0.00,0.00,1.00}{##1}}}
\@namedef{PY@tok@nn}{\let\PY@bf=\textbf\def\PY@tc##1{\textcolor[rgb]{0.00,0.00,1.00}{##1}}}
\@namedef{PY@tok@ne}{\let\PY@bf=\textbf\def\PY@tc##1{\textcolor[rgb]{0.80,0.25,0.22}{##1}}}
\@namedef{PY@tok@nv}{\def\PY@tc##1{\textcolor[rgb]{0.10,0.09,0.49}{##1}}}
\@namedef{PY@tok@no}{\def\PY@tc##1{\textcolor[rgb]{0.53,0.00,0.00}{##1}}}
\@namedef{PY@tok@nl}{\def\PY@tc##1{\textcolor[rgb]{0.46,0.46,0.00}{##1}}}
\@namedef{PY@tok@ni}{\let\PY@bf=\textbf\def\PY@tc##1{\textcolor[rgb]{0.44,0.44,0.44}{##1}}}
\@namedef{PY@tok@na}{\def\PY@tc##1{\textcolor[rgb]{0.41,0.47,0.13}{##1}}}
\@namedef{PY@tok@nt}{\let\PY@bf=\textbf\def\PY@tc##1{\textcolor[rgb]{0.00,0.50,0.00}{##1}}}
\@namedef{PY@tok@nd}{\def\PY@tc##1{\textcolor[rgb]{0.67,0.13,1.00}{##1}}}
\@namedef{PY@tok@s}{\def\PY@tc##1{\textcolor[rgb]{0.73,0.13,0.13}{##1}}}
\@namedef{PY@tok@sd}{\let\PY@it=\textit\def\PY@tc##1{\textcolor[rgb]{0.73,0.13,0.13}{##1}}}
\@namedef{PY@tok@si}{\let\PY@bf=\textbf\def\PY@tc##1{\textcolor[rgb]{0.64,0.35,0.47}{##1}}}
\@namedef{PY@tok@se}{\let\PY@bf=\textbf\def\PY@tc##1{\textcolor[rgb]{0.67,0.36,0.12}{##1}}}
\@namedef{PY@tok@sr}{\def\PY@tc##1{\textcolor[rgb]{0.64,0.35,0.47}{##1}}}
\@namedef{PY@tok@ss}{\def\PY@tc##1{\textcolor[rgb]{0.10,0.09,0.49}{##1}}}
\@namedef{PY@tok@sx}{\def\PY@tc##1{\textcolor[rgb]{0.00,0.50,0.00}{##1}}}
\@namedef{PY@tok@m}{\def\PY@tc##1{\textcolor[rgb]{0.40,0.40,0.40}{##1}}}
\@namedef{PY@tok@gh}{\let\PY@bf=\textbf\def\PY@tc##1{\textcolor[rgb]{0.00,0.00,0.50}{##1}}}
\@namedef{PY@tok@gu}{\let\PY@bf=\textbf\def\PY@tc##1{\textcolor[rgb]{0.50,0.00,0.50}{##1}}}
\@namedef{PY@tok@gd}{\def\PY@tc##1{\textcolor[rgb]{0.63,0.00,0.00}{##1}}}
\@namedef{PY@tok@gi}{\def\PY@tc##1{\textcolor[rgb]{0.00,0.52,0.00}{##1}}}
\@namedef{PY@tok@gr}{\def\PY@tc##1{\textcolor[rgb]{0.89,0.00,0.00}{##1}}}
\@namedef{PY@tok@ge}{\let\PY@it=\textit}
\@namedef{PY@tok@gs}{\let\PY@bf=\textbf}
\@namedef{PY@tok@ges}{\let\PY@bf=\textbf\let\PY@it=\textit}
\@namedef{PY@tok@gp}{\let\PY@bf=\textbf\def\PY@tc##1{\textcolor[rgb]{0.00,0.00,0.50}{##1}}}
\@namedef{PY@tok@go}{\def\PY@tc##1{\textcolor[rgb]{0.44,0.44,0.44}{##1}}}
\@namedef{PY@tok@gt}{\def\PY@tc##1{\textcolor[rgb]{0.00,0.27,0.87}{##1}}}
\@namedef{PY@tok@err}{\def\PY@bc##1{{\setlength{\fboxsep}{\string -\fboxrule}\fcolorbox[rgb]{1.00,0.00,0.00}{1,1,1}{\strut ##1}}}}
\@namedef{PY@tok@kc}{\let\PY@bf=\textbf\def\PY@tc##1{\textcolor[rgb]{0.00,0.50,0.00}{##1}}}
\@namedef{PY@tok@kd}{\let\PY@bf=\textbf\def\PY@tc##1{\textcolor[rgb]{0.00,0.50,0.00}{##1}}}
\@namedef{PY@tok@kn}{\let\PY@bf=\textbf\def\PY@tc##1{\textcolor[rgb]{0.00,0.50,0.00}{##1}}}
\@namedef{PY@tok@kr}{\let\PY@bf=\textbf\def\PY@tc##1{\textcolor[rgb]{0.00,0.50,0.00}{##1}}}
\@namedef{PY@tok@bp}{\def\PY@tc##1{\textcolor[rgb]{0.00,0.50,0.00}{##1}}}
\@namedef{PY@tok@fm}{\def\PY@tc##1{\textcolor[rgb]{0.00,0.00,1.00}{##1}}}
\@namedef{PY@tok@vc}{\def\PY@tc##1{\textcolor[rgb]{0.10,0.09,0.49}{##1}}}
\@namedef{PY@tok@vg}{\def\PY@tc##1{\textcolor[rgb]{0.10,0.09,0.49}{##1}}}
\@namedef{PY@tok@vi}{\def\PY@tc##1{\textcolor[rgb]{0.10,0.09,0.49}{##1}}}
\@namedef{PY@tok@vm}{\def\PY@tc##1{\textcolor[rgb]{0.10,0.09,0.49}{##1}}}
\@namedef{PY@tok@sa}{\def\PY@tc##1{\textcolor[rgb]{0.73,0.13,0.13}{##1}}}
\@namedef{PY@tok@sb}{\def\PY@tc##1{\textcolor[rgb]{0.73,0.13,0.13}{##1}}}
\@namedef{PY@tok@sc}{\def\PY@tc##1{\textcolor[rgb]{0.73,0.13,0.13}{##1}}}
\@namedef{PY@tok@dl}{\def\PY@tc##1{\textcolor[rgb]{0.73,0.13,0.13}{##1}}}
\@namedef{PY@tok@s2}{\def\PY@tc##1{\textcolor[rgb]{0.73,0.13,0.13}{##1}}}
\@namedef{PY@tok@sh}{\def\PY@tc##1{\textcolor[rgb]{0.73,0.13,0.13}{##1}}}
\@namedef{PY@tok@s1}{\def\PY@tc##1{\textcolor[rgb]{0.73,0.13,0.13}{##1}}}
\@namedef{PY@tok@mb}{\def\PY@tc##1{\textcolor[rgb]{0.40,0.40,0.40}{##1}}}
\@namedef{PY@tok@mf}{\def\PY@tc##1{\textcolor[rgb]{0.40,0.40,0.40}{##1}}}
\@namedef{PY@tok@mh}{\def\PY@tc##1{\textcolor[rgb]{0.40,0.40,0.40}{##1}}}
\@namedef{PY@tok@mi}{\def\PY@tc##1{\textcolor[rgb]{0.40,0.40,0.40}{##1}}}
\@namedef{PY@tok@il}{\def\PY@tc##1{\textcolor[rgb]{0.40,0.40,0.40}{##1}}}
\@namedef{PY@tok@mo}{\def\PY@tc##1{\textcolor[rgb]{0.40,0.40,0.40}{##1}}}
\@namedef{PY@tok@ch}{\let\PY@it=\textit\def\PY@tc##1{\textcolor[rgb]{0.24,0.48,0.48}{##1}}}
\@namedef{PY@tok@cm}{\let\PY@it=\textit\def\PY@tc##1{\textcolor[rgb]{0.24,0.48,0.48}{##1}}}
\@namedef{PY@tok@cpf}{\let\PY@it=\textit\def\PY@tc##1{\textcolor[rgb]{0.24,0.48,0.48}{##1}}}
\@namedef{PY@tok@c1}{\let\PY@it=\textit\def\PY@tc##1{\textcolor[rgb]{0.24,0.48,0.48}{##1}}}
\@namedef{PY@tok@cs}{\let\PY@it=\textit\def\PY@tc##1{\textcolor[rgb]{0.24,0.48,0.48}{##1}}}

\def\PYZbs{\char`\\}
\def\PYZus{\char`\_}
\def\PYZob{\char`\{}
\def\PYZcb{\char`\}}
\def\PYZca{\char`\^}
\def\PYZam{\char`\&}
\def\PYZlt{\char`\<}
\def\PYZgt{\char`\>}
\def\PYZsh{\char`\#}
\def\PYZpc{\char`\%}
\def\PYZdl{\char`\$}
\def\PYZhy{\char`\-}
\def\PYZsq{\char`\'}
\def\PYZdq{\char`\"}
\def\PYZti{\char`\~}
% for compatibility with earlier versions
\def\PYZat{@}
\def\PYZlb{[}
\def\PYZrb{]}
\makeatother


    % For linebreaks inside Verbatim environment from package fancyvrb.
    \makeatletter
        \newbox\Wrappedcontinuationbox
        \newbox\Wrappedvisiblespacebox
        \newcommand*\Wrappedvisiblespace {\textcolor{red}{\textvisiblespace}}
        \newcommand*\Wrappedcontinuationsymbol {\textcolor{red}{\llap{\tiny$\m@th\hookrightarrow$}}}
        \newcommand*\Wrappedcontinuationindent {3ex }
        \newcommand*\Wrappedafterbreak {\kern\Wrappedcontinuationindent\copy\Wrappedcontinuationbox}
        % Take advantage of the already applied Pygments mark-up to insert
        % potential linebreaks for TeX processing.
        %        {, <, #, %, $, ' and ": go to next line.
        %        _, }, ^, &, >, - and ~: stay at end of broken line.
        % Use of \textquotesingle for straight quote.
        \newcommand*\Wrappedbreaksatspecials {%
            \def\PYGZus{\discretionary{\char`\_}{\Wrappedafterbreak}{\char`\_}}%
            \def\PYGZob{\discretionary{}{\Wrappedafterbreak\char`\{}{\char`\{}}%
            \def\PYGZcb{\discretionary{\char`\}}{\Wrappedafterbreak}{\char`\}}}%
            \def\PYGZca{\discretionary{\char`\^}{\Wrappedafterbreak}{\char`\^}}%
            \def\PYGZam{\discretionary{\char`\&}{\Wrappedafterbreak}{\char`\&}}%
            \def\PYGZlt{\discretionary{}{\Wrappedafterbreak\char`\<}{\char`\<}}%
            \def\PYGZgt{\discretionary{\char`\>}{\Wrappedafterbreak}{\char`\>}}%
            \def\PYGZsh{\discretionary{}{\Wrappedafterbreak\char`\#}{\char`\#}}%
            \def\PYGZpc{\discretionary{}{\Wrappedafterbreak\char`\%}{\char`\%}}%
            \def\PYGZdl{\discretionary{}{\Wrappedafterbreak\char`\$}{\char`\$}}%
            \def\PYGZhy{\discretionary{\char`\-}{\Wrappedafterbreak}{\char`\-}}%
            \def\PYGZsq{\discretionary{}{\Wrappedafterbreak\textquotesingle}{\textquotesingle}}%
            \def\PYGZdq{\discretionary{}{\Wrappedafterbreak\char`\"}{\char`\"}}%
            \def\PYGZti{\discretionary{\char`\~}{\Wrappedafterbreak}{\char`\~}}%
        }
        % Some characters . , ; ? ! / are not pygmentized.
        % This macro makes them "active" and they will insert potential linebreaks
        \newcommand*\Wrappedbreaksatpunct {%
            \lccode`\~`\.\lowercase{\def~}{\discretionary{\hbox{\char`\.}}{\Wrappedafterbreak}{\hbox{\char`\.}}}%
            \lccode`\~`\,\lowercase{\def~}{\discretionary{\hbox{\char`\,}}{\Wrappedafterbreak}{\hbox{\char`\,}}}%
            \lccode`\~`\;\lowercase{\def~}{\discretionary{\hbox{\char`\;}}{\Wrappedafterbreak}{\hbox{\char`\;}}}%
            \lccode`\~`\:\lowercase{\def~}{\discretionary{\hbox{\char`\:}}{\Wrappedafterbreak}{\hbox{\char`\:}}}%
            \lccode`\~`\?\lowercase{\def~}{\discretionary{\hbox{\char`\?}}{\Wrappedafterbreak}{\hbox{\char`\?}}}%
            \lccode`\~`\!\lowercase{\def~}{\discretionary{\hbox{\char`\!}}{\Wrappedafterbreak}{\hbox{\char`\!}}}%
            \lccode`\~`\/\lowercase{\def~}{\discretionary{\hbox{\char`\/}}{\Wrappedafterbreak}{\hbox{\char`\/}}}%
            \catcode`\.\active
            \catcode`\,\active
            \catcode`\;\active
            \catcode`\:\active
            \catcode`\?\active
            \catcode`\!\active
            \catcode`\/\active
            \lccode`\~`\~
        }
    \makeatother

    \let\OriginalVerbatim=\Verbatim
    \makeatletter
    \renewcommand{\Verbatim}[1][1]{%
        %\parskip\z@skip
        \sbox\Wrappedcontinuationbox {\Wrappedcontinuationsymbol}%
        \sbox\Wrappedvisiblespacebox {\FV@SetupFont\Wrappedvisiblespace}%
        \def\FancyVerbFormatLine ##1{\hsize\linewidth
            \vtop{\raggedright\hyphenpenalty\z@\exhyphenpenalty\z@
                \doublehyphendemerits\z@\finalhyphendemerits\z@
                \strut ##1\strut}%
        }%
        % If the linebreak is at a space, the latter will be displayed as visible
        % space at end of first line, and a continuation symbol starts next line.
        % Stretch/shrink are however usually zero for typewriter font.
        \def\FV@Space {%
            \nobreak\hskip\z@ plus\fontdimen3\font minus\fontdimen4\font
            \discretionary{\copy\Wrappedvisiblespacebox}{\Wrappedafterbreak}
            {\kern\fontdimen2\font}%
        }%

        % Allow breaks at special characters using \PYG... macros.
        \Wrappedbreaksatspecials
        % Breaks at punctuation characters . , ; ? ! and / need catcode=\active
        \OriginalVerbatim[#1,codes*=\Wrappedbreaksatpunct]%
    }
    \makeatother

    % Exact colors from NB
    \definecolor{incolor}{HTML}{303F9F}
    \definecolor{outcolor}{HTML}{D84315}
    \definecolor{cellborder}{HTML}{CFCFCF}
    \definecolor{cellbackground}{HTML}{F7F7F7}

    % prompt
    \makeatletter
    \newcommand{\boxspacing}{\kern\kvtcb@left@rule\kern\kvtcb@boxsep}
    \makeatother
    \newcommand{\prompt}[4]{
        {\ttfamily\llap{{\color{#2}[#3]:\hspace{3pt}#4}}\vspace{-\baselineskip}}
    }
    

    
    % Prevent overflowing lines due to hard-to-break entities
    \sloppy
    % Setup hyperref package
    \hypersetup{
      breaklinks=true,  % so long urls are correctly broken across lines
      colorlinks=true,
      urlcolor=urlcolor,
      linkcolor=linkcolor,
      citecolor=citecolor,
      }
    % Slightly bigger margins than the latex defaults
    
    \geometry{verbose,tmargin=1in,bmargin=1in,lmargin=1in,rmargin=1in}
    
    

\begin{document}
    
    \maketitle
    
    

    
    \(\Huge\textbf{Quick Summary of P2P Channels}\)

    \begin{tcolorbox}[breakable, size=fbox, boxrule=1pt, pad at break*=1mm,colback=cellbackground, colframe=cellborder]
\prompt{In}{incolor}{1}{\boxspacing}
\begin{Verbatim}[commandchars=\\\{\}]
\PY{k}{using}\PY{+w}{ }\PY{n}{Plots}\PY{p}{,}\PY{+w}{ }\PY{n}{LaTeXStrings}
\end{Verbatim}
\end{tcolorbox}

    \begin{tcolorbox}[breakable, size=fbox, boxrule=1pt, pad at break*=1mm,colback=cellbackground, colframe=cellborder]
\prompt{In}{incolor}{2}{\boxspacing}
\begin{Verbatim}[commandchars=\\\{\}]
\PY{c}{\PYZsh{} Define coordinates for transmitted and received states}
\PY{n}{tx}\PY{+w}{ }\PY{o}{=}\PY{+w}{ }\PY{p}{[}\PY{l+m+mi}{0}\PY{p}{,}\PY{+w}{ }\PY{l+m+mi}{1}\PY{p}{]}
\PY{n}{rx}\PY{+w}{ }\PY{o}{=}\PY{+w}{ }\PY{p}{[}\PY{l+m+mi}{0}\PY{p}{,}\PY{+w}{ }\PY{l+m+mi}{1}\PY{p}{]}

\PY{c}{\PYZsh{} Define probabilities as labels}
\PY{n}{p}\PY{+w}{ }\PY{o}{=}\PY{+w}{ }\PY{p}{[}\PY{l+s+sa}{L}\PY{l+s}{\PYZdq{}}\PY{l+s}{\PYZbs{}}\PY{l+s}{epsilon}\PY{l+s}{\PYZdq{}}\PY{p}{,}\PY{+w}{ }\PY{l+s+sa}{L}\PY{l+s}{\PYZdq{}}\PY{l+s}{1 \PYZhy{} }\PY{l+s}{\PYZbs{}}\PY{l+s}{epsilon}\PY{l+s}{\PYZdq{}}\PY{p}{]}\PY{p}{;}
\end{Verbatim}
\end{tcolorbox}

    \begin{tcolorbox}[breakable, size=fbox, boxrule=1pt, pad at break*=1mm,colback=cellbackground, colframe=cellborder]
\prompt{In}{incolor}{3}{\boxspacing}
\begin{Verbatim}[commandchars=\\\{\}]
\PY{c}{\PYZsh{} Plot the BSC diagram}
\PY{n}{plot}\PY{p}{(}\PY{n}{grid}\PY{o}{=}\PY{n+nb}{false}
\PY{+w}{    }\PY{p}{,}\PY{+w}{ }\PY{n}{xaxis}\PY{o}{=}\PY{n+nb}{false}\PY{p}{,}\PY{+w}{ }\PY{n}{yaxis}\PY{o}{=}\PY{n+nb}{false}
\PY{+w}{    }\PY{p}{,}\PY{+w}{ }\PY{n}{framestyle}\PY{o}{=}\PY{l+s+ss}{:none}\PY{p}{,}\PY{+w}{ }\PY{n}{size}\PY{+w}{ }\PY{o}{=}\PY{+w}{ }\PY{p}{(}\PY{l+m+mi}{200}\PY{p}{,}\PY{l+m+mi}{200}\PY{p}{)}
\PY{+w}{    }\PY{p}{,}\PY{+w}{ }\PY{n}{title}\PY{+w}{ }\PY{o}{=}\PY{+w}{ }\PY{l+s}{\PYZdq{}}\PY{l+s}{BEC}\PY{l+s}{\PYZdq{}}
\PY{p}{)}

\PY{n}{plot!}\PY{p}{(}\PY{p}{[}\PY{l+m+mi}{0}\PY{p}{,}\PY{+w}{ }\PY{l+m+mf}{0.5}\PY{p}{]}\PY{p}{,}\PY{+w}{ }\PY{p}{[}\PY{l+m+mi}{1}\PY{p}{,}\PY{+w}{ }\PY{l+m+mf}{0.5}\PY{p}{]}\PY{p}{,}\PY{+w}{ }\PY{n}{arrow}\PY{o}{=}\PY{l+s+ss}{:arrow}\PY{p}{,}\PY{+w}{ }\PY{n}{label}\PY{o}{=}\PY{l+s}{\PYZdq{}}\PY{l+s}{\PYZdq{}}\PY{p}{,}\PY{+w}{ }\PY{n}{color}\PY{o}{=}\PY{l+s+ss}{:blue}\PY{p}{)}\PY{+w}{ }\PY{c}{\PYZsh{} p line}
\PY{n}{plot!}\PY{p}{(}\PY{p}{[}\PY{l+m+mi}{0}\PY{p}{,}\PY{+w}{ }\PY{l+m+mf}{0.5}\PY{p}{]}\PY{p}{,}\PY{+w}{ }\PY{p}{[}\PY{l+m+mi}{0}\PY{p}{,}\PY{+w}{ }\PY{l+m+mf}{0.5}\PY{p}{]}\PY{p}{,}\PY{+w}{ }\PY{n}{arrow}\PY{o}{=}\PY{l+s+ss}{:arrow}\PY{p}{,}\PY{+w}{ }\PY{n}{label}\PY{o}{=}\PY{l+s}{\PYZdq{}}\PY{l+s}{\PYZdq{}}\PY{p}{,}\PY{+w}{ }\PY{n}{color}\PY{o}{=}\PY{l+s+ss}{:blue}\PY{p}{)}\PY{+w}{ }\PY{c}{\PYZsh{} p line}
\PY{n}{plot!}\PY{p}{(}\PY{p}{[}\PY{l+m+mi}{0}\PY{p}{,}\PY{+w}{ }\PY{l+m+mi}{1}\PY{p}{]}\PY{p}{,}\PY{+w}{ }\PY{p}{[}\PY{l+m+mi}{1}\PY{p}{,}\PY{+w}{ }\PY{l+m+mi}{1}\PY{p}{]}\PY{p}{,}\PY{+w}{ }\PY{n}{label}\PY{o}{=}\PY{l+s}{\PYZdq{}}\PY{l+s}{\PYZdq{}}\PY{p}{,}\PY{+w}{ }\PY{n}{color}\PY{o}{=}\PY{l+s+ss}{:black}\PY{p}{)}\PY{+w}{ }\PY{c}{\PYZsh{} 1\PYZhy{}p}
\PY{n}{plot!}\PY{p}{(}\PY{p}{[}\PY{l+m+mi}{0}\PY{p}{,}\PY{+w}{ }\PY{l+m+mi}{1}\PY{p}{]}\PY{p}{,}\PY{+w}{ }\PY{p}{[}\PY{l+m+mi}{0}\PY{p}{,}\PY{+w}{ }\PY{l+m+mi}{0}\PY{p}{]}\PY{p}{,}\PY{+w}{ }\PY{n}{label}\PY{o}{=}\PY{l+s}{\PYZdq{}}\PY{l+s}{\PYZdq{}}\PY{p}{,}\PY{+w}{ }\PY{n}{color}\PY{o}{=}\PY{l+s+ss}{:black}\PY{p}{)}

\PY{c}{\PYZsh{} Annotate the graph}
\PY{n}{annotate!}\PY{p}{(}\PY{o}{\PYZhy{}}\PY{l+m+mf}{0.05}\PY{p}{,}\PY{+w}{ }\PY{l+m+mf}{1.05}\PY{p}{,}\PY{+w}{ }\PY{n}{tx}\PY{p}{[}\PY{l+m+mi}{1}\PY{p}{]}\PY{p}{)}\PY{p}{;}\PY{+w}{ }\PY{n}{annotate!}\PY{p}{(}\PY{l+m+mf}{1.05}\PY{p}{,}\PY{+w}{ }\PY{l+m+mf}{1.05}\PY{p}{,}\PY{+w}{ }\PY{n}{rx}\PY{p}{[}\PY{l+m+mi}{1}\PY{p}{]}\PY{p}{)}
\PY{n}{annotate!}\PY{p}{(}\PY{o}{\PYZhy{}}\PY{l+m+mf}{0.05}\PY{p}{,}\PY{+w}{ }\PY{o}{\PYZhy{}}\PY{l+m+mf}{0.05}\PY{p}{,}\PY{+w}{ }\PY{n}{tx}\PY{p}{[}\PY{l+m+mi}{2}\PY{p}{]}\PY{p}{)}\PY{p}{;}\PY{+w}{ }\PY{n}{annotate!}\PY{p}{(}\PY{l+m+mf}{1.05}\PY{p}{,}\PY{+w}{ }\PY{o}{\PYZhy{}}\PY{l+m+mf}{0.05}\PY{p}{,}\PY{+w}{ }\PY{n}{rx}\PY{p}{[}\PY{l+m+mi}{2}\PY{p}{]}\PY{p}{)}
\PY{n}{annotate!}\PY{p}{(}\PY{l+m+mf}{0.4}\PY{p}{,}\PY{+w}{ }\PY{l+m+mf}{0.5}\PY{p}{,}\PY{+w}{ }\PY{n}{p}\PY{p}{[}\PY{l+m+mi}{1}\PY{p}{]}\PY{p}{)}\PY{p}{;}\PY{+w}{ }\PY{n}{annotate!}\PY{p}{(}\PY{l+m+mf}{0.5}\PY{p}{,}\PY{+w}{ }\PY{o}{\PYZhy{}}\PY{l+m+mf}{0.1}\PY{p}{,}\PY{+w}{ }\PY{n}{p}\PY{p}{[}\PY{l+m+mi}{2}\PY{p}{]}\PY{p}{)}\PY{p}{;}\PY{+w}{ }\PY{n}{annotate!}\PY{p}{(}\PY{l+m+mf}{0.5}\PY{p}{,}\PY{+w}{ }\PY{l+m+mf}{1.1}\PY{p}{,}\PY{+w}{ }\PY{n}{p}\PY{p}{[}\PY{l+m+mi}{2}\PY{p}{]}\PY{p}{)}
\end{Verbatim}
\end{tcolorbox}
 
            
\prompt{Out}{outcolor}{3}{}
    
    \begin{center}
    \adjustimage{max size={0.9\linewidth}{0.9\paperheight}}{REVIEW_files/REVIEW_3_0.pdf}
    \end{center}
    { \hspace*{\fill} \\}
    

    \section{Binary Erasure Channel (BEC)}\label{binary-erasure-channel-bec}

\begin{enumerate}
\def\labelenumi{\arabic{enumi}.}
\tightlist
\item
  \textbf{Channel Model}:

  \begin{itemize}
  \tightlist
  \item
    Transmits binary symbols (\(0\) or \(1\)).
  \item
    Each transmitted bit is either:

    \begin{itemize}
    \tightlist
    \item
      \textbf{Received correctly} with probability \(1 - \epsilon\), or
    \item
      \textbf{Erased} with probability \(\epsilon\), represented as an
      erasure symbol (\(e\)).
    \end{itemize}
  \end{itemize}

  Example:

  \begin{itemize}
  \tightlist
  \item
    \(0 \to 0\) or \(e\),
  \item
    \(1 \to 1\) or \(e\).
  \end{itemize}
\item
  \textbf{Capacity (\(C\))}: \(\boxed{C = 1 - \epsilon}\)

  \begin{itemize}
  \tightlist
  \item
    \textbf{\(1\)}: Maximum capacity with no erasures
    (\(\epsilon = 0\)).
  \item
    \textbf{\(\epsilon\)}: Fraction of bits erased by the channel,
    reducing capacity.
  \end{itemize}
\item
  \textbf{Behavior}:

  \begin{itemize}
  \tightlist
  \item
    \(\epsilon = 0\): Perfect channel, \(C = 1\).
  \item
    \(\epsilon = 1\): Completely erasing channel, \(C = 0\).
  \item
    For \(0 < \epsilon < 1\): Capacity decreases linearly as
    \(\epsilon\) increases.
  \end{itemize}
\end{enumerate}

\begin{center}\rule{0.5\linewidth}{0.5pt}\end{center}

\paragraph{Compact Intuition:}\label{compact-intuition}

The \textbf{Binary Erasure Channel} (BEC) capacity is the fraction of
bits successfully transmitted. Erasures (\(\epsilon\)) reduce capacity
by removing information from the channel.

    \begin{tcolorbox}[breakable, size=fbox, boxrule=1pt, pad at break*=1mm,colback=cellbackground, colframe=cellborder]
\prompt{In}{incolor}{4}{\boxspacing}
\begin{Verbatim}[commandchars=\\\{\}]
\PY{c}{\PYZsh{} Plot the BSC diagram}
\PY{n}{plot}\PY{p}{(}\PY{n}{grid}\PY{o}{=}\PY{n+nb}{false}
\PY{+w}{    }\PY{p}{,}\PY{+w}{ }\PY{n}{xaxis}\PY{o}{=}\PY{n+nb}{false}\PY{p}{,}\PY{+w}{ }\PY{n}{yaxis}\PY{o}{=}\PY{n+nb}{false}
\PY{+w}{    }\PY{p}{,}\PY{+w}{ }\PY{n}{framestyle}\PY{o}{=}\PY{l+s+ss}{:none}\PY{p}{,}\PY{+w}{ }\PY{n}{size}\PY{+w}{ }\PY{o}{=}\PY{+w}{ }\PY{p}{(}\PY{l+m+mi}{200}\PY{p}{,}\PY{l+m+mi}{200}\PY{p}{)}
\PY{+w}{    }\PY{p}{,}\PY{+w}{ }\PY{n}{title}\PY{+w}{ }\PY{o}{=}\PY{+w}{ }\PY{l+s}{\PYZdq{}}\PY{l+s}{BSC}\PY{l+s}{\PYZdq{}}
\PY{p}{)}

\PY{n}{plot!}\PY{p}{(}\PY{p}{[}\PY{l+m+mi}{0}\PY{p}{,}\PY{+w}{ }\PY{l+m+mi}{1}\PY{p}{]}\PY{p}{,}\PY{+w}{ }\PY{p}{[}\PY{l+m+mi}{1}\PY{p}{,}\PY{+w}{ }\PY{l+m+mi}{0}\PY{p}{]}\PY{p}{,}\PY{+w}{ }\PY{n}{arrow}\PY{o}{=}\PY{l+s+ss}{:arrow}\PY{p}{,}\PY{+w}{ }\PY{n}{label}\PY{o}{=}\PY{l+s}{\PYZdq{}}\PY{l+s}{\PYZdq{}}\PY{p}{,}\PY{+w}{ }\PY{n}{color}\PY{o}{=}\PY{l+s+ss}{:blue}\PY{p}{)}\PY{+w}{ }\PY{c}{\PYZsh{} p line}
\PY{n}{plot!}\PY{p}{(}\PY{p}{[}\PY{l+m+mi}{0}\PY{p}{,}\PY{+w}{ }\PY{l+m+mi}{1}\PY{p}{]}\PY{p}{,}\PY{+w}{ }\PY{p}{[}\PY{l+m+mi}{0}\PY{p}{,}\PY{+w}{ }\PY{l+m+mi}{1}\PY{p}{]}\PY{p}{,}\PY{+w}{ }\PY{n}{arrow}\PY{o}{=}\PY{l+s+ss}{:arrow}\PY{p}{,}\PY{+w}{ }\PY{n}{label}\PY{o}{=}\PY{l+s}{\PYZdq{}}\PY{l+s}{\PYZdq{}}\PY{p}{,}\PY{+w}{ }\PY{n}{color}\PY{o}{=}\PY{l+s+ss}{:blue}\PY{p}{)}\PY{+w}{ }\PY{c}{\PYZsh{} p line}
\PY{n}{plot!}\PY{p}{(}\PY{p}{[}\PY{l+m+mi}{0}\PY{p}{,}\PY{+w}{ }\PY{l+m+mi}{1}\PY{p}{]}\PY{p}{,}\PY{+w}{ }\PY{p}{[}\PY{l+m+mi}{1}\PY{p}{,}\PY{+w}{ }\PY{l+m+mi}{1}\PY{p}{]}\PY{p}{,}\PY{+w}{ }\PY{n}{label}\PY{o}{=}\PY{l+s}{\PYZdq{}}\PY{l+s}{\PYZdq{}}\PY{p}{,}\PY{+w}{ }\PY{n}{color}\PY{o}{=}\PY{l+s+ss}{:black}\PY{p}{)}\PY{+w}{ }\PY{c}{\PYZsh{} 1\PYZhy{}p}
\PY{n}{plot!}\PY{p}{(}\PY{p}{[}\PY{l+m+mi}{0}\PY{p}{,}\PY{+w}{ }\PY{l+m+mi}{1}\PY{p}{]}\PY{p}{,}\PY{+w}{ }\PY{p}{[}\PY{l+m+mi}{0}\PY{p}{,}\PY{+w}{ }\PY{l+m+mi}{0}\PY{p}{]}\PY{p}{,}\PY{+w}{ }\PY{n}{label}\PY{o}{=}\PY{l+s}{\PYZdq{}}\PY{l+s}{\PYZdq{}}\PY{p}{,}\PY{+w}{ }\PY{n}{color}\PY{o}{=}\PY{l+s+ss}{:black}\PY{p}{)}

\PY{c}{\PYZsh{} Annotate the graph}
\PY{n}{annotate!}\PY{p}{(}\PY{o}{\PYZhy{}}\PY{l+m+mf}{0.05}\PY{p}{,}\PY{+w}{ }\PY{l+m+mf}{1.05}\PY{p}{,}\PY{+w}{ }\PY{n}{tx}\PY{p}{[}\PY{l+m+mi}{1}\PY{p}{]}\PY{p}{)}\PY{p}{;}\PY{+w}{ }\PY{n}{annotate!}\PY{p}{(}\PY{l+m+mf}{1.05}\PY{p}{,}\PY{+w}{ }\PY{l+m+mf}{1.05}\PY{p}{,}\PY{+w}{ }\PY{n}{rx}\PY{p}{[}\PY{l+m+mi}{1}\PY{p}{]}\PY{p}{)}
\PY{n}{annotate!}\PY{p}{(}\PY{o}{\PYZhy{}}\PY{l+m+mf}{0.05}\PY{p}{,}\PY{+w}{ }\PY{o}{\PYZhy{}}\PY{l+m+mf}{0.05}\PY{p}{,}\PY{+w}{ }\PY{n}{tx}\PY{p}{[}\PY{l+m+mi}{2}\PY{p}{]}\PY{p}{)}\PY{p}{;}\PY{+w}{ }\PY{n}{annotate!}\PY{p}{(}\PY{l+m+mf}{1.05}\PY{p}{,}\PY{+w}{ }\PY{o}{\PYZhy{}}\PY{l+m+mf}{0.05}\PY{p}{,}\PY{+w}{ }\PY{n}{rx}\PY{p}{[}\PY{l+m+mi}{2}\PY{p}{]}\PY{p}{)}
\PY{n}{annotate!}\PY{p}{(}\PY{l+m+mf}{0.4}\PY{p}{,}\PY{+w}{ }\PY{l+m+mf}{0.5}\PY{p}{,}\PY{+w}{ }\PY{n}{p}\PY{p}{[}\PY{l+m+mi}{1}\PY{p}{]}\PY{p}{)}\PY{p}{;}\PY{+w}{ }\PY{n}{annotate!}\PY{p}{(}\PY{l+m+mf}{0.5}\PY{p}{,}\PY{+w}{ }\PY{o}{\PYZhy{}}\PY{l+m+mf}{0.1}\PY{p}{,}\PY{+w}{ }\PY{n}{p}\PY{p}{[}\PY{l+m+mi}{2}\PY{p}{]}\PY{p}{)}\PY{p}{;}\PY{+w}{ }\PY{n}{annotate!}\PY{p}{(}\PY{l+m+mf}{0.5}\PY{p}{,}\PY{+w}{ }\PY{l+m+mf}{1.1}\PY{p}{,}\PY{+w}{ }\PY{n}{p}\PY{p}{[}\PY{l+m+mi}{2}\PY{p}{]}\PY{p}{)}
\end{Verbatim}
\end{tcolorbox}
 
            
\prompt{Out}{outcolor}{4}{}
    
    \begin{center}
    \adjustimage{max size={0.9\linewidth}{0.9\paperheight}}{REVIEW_files/REVIEW_5_0.pdf}
    \end{center}
    { \hspace*{\fill} \\}
    

    \section{Binary Symmetric Channel
(BSC)}\label{binary-symmetric-channel-bsc}

\begin{enumerate}
\def\labelenumi{\arabic{enumi}.}
\tightlist
\item
  \textbf{Channel Model}:

  \begin{itemize}
  \tightlist
  \item
    Transmits binary symbols (\(0\) or \(1\)).
  \item
    Each bit has a probability \(\epsilon\) of being flipped.
  \item
    Error probability: \(P(0 \to 1) = P(1 \to 0) = \epsilon\).
  \item
    Correct transmission probability:
    \(P(0 \to 0) = P(1 \to 1) = 1 - \epsilon\).
  \end{itemize}
\item
  \textbf{Capacity (\(C_{BSC}\))}:
  \(\boxed{C_{BSC} = 1 - H_2(\epsilon)}\)

  \begin{itemize}
  \tightlist
  \item
    \textbf{\(1\)}: Maximum capacity without errors.
  \item
    \textbf{\(H_2(\epsilon)\)}: Binary entropy function, representing
    uncertainty due to errors.
  \end{itemize}
\item
  \textbf{Binary Entropy Function (\(H_2(\epsilon)\))}:
  \(\boxed{H_2(\epsilon) = - \epsilon \cdot \log_2(\epsilon) - (1 - \epsilon) \cdot \log_2(1 - \epsilon)}\)

  \begin{itemize}
  \tightlist
  \item
    \(H_2(0) = 0\): No errors, full capacity.
  \item
    \(H_2(0.5) = 1\): Maximum uncertainty, no capacity.
  \end{itemize}
\item
  \textbf{Behavior}:

  \begin{itemize}
  \tightlist
  \item
    \(\epsilon = 0\): Perfect channel, \(C_{BSC} = 1\).
  \item
    \(\epsilon = 0.5\): Completely noisy, \(C_{BSC} = 0\).
  \item
    For \(0 < \epsilon < 0.5\): Capacity decreases as \(\epsilon\)
    increases.
  \end{itemize}
\end{enumerate}

\begin{center}\rule{0.5\linewidth}{0.5pt}\end{center}

\paragraph{Compact Intuition:}\label{compact-intuition}

The BSC capacity is the theoretical maximum rate of reliable data
transmission, reduced by the uncertainty caused by errors. Lower
\(\epsilon\) means higher capacity, while higher \(\epsilon\) reduces
it.

    \subsubsection{AWGN Channel Summary}\label{awgn-channel-summary}

\begin{enumerate}
\def\labelenumi{\arabic{enumi}.}
\item
  \textbf{Channel Model}: \(\boxed{y = x + w, \quad w \sim N(0, N_0)}\)

  \begin{itemize}
  \tightlist
  \item
    \(x\): Transmitted signal.
  \item
    \(y\): Received signal.
  \item
    \(w\): Gaussian noise with zero mean and variance \(N_0\) the noise
    power spectral density.
  \end{itemize}
\item
  \textbf{Signal Power}: \(\boxed{P = E[|x|^2]}\)
\item
  \textbf{Signal-to-Noise Ratio (SNR)}: \(\boxed{SNR = \frac{P}{N_0}}\)
\item
  \textbf{Channel Capacity}:
  \(\boxed{C = \log_2\left(1 + SNR\right) = \log_2\left(1 + \frac{P}{N_0}\right)}\)

  \begin{itemize}
  \tightlist
  \item
    \(C\): Maximum achievable data rate (in bps/Hz).
  \end{itemize}
\item
  \textbf{Key Behavior}:

  \begin{itemize}
  \tightlist
  \item
    \(P \uparrow\) (high signal power): \(C \uparrow\) (more capacity).
  \item
    \(N_0 \uparrow\) (high noise): \(C \downarrow\) (less capacity).
  \end{itemize}
\end{enumerate}

\textbf{Insight}: The AWGN channel capacity quantifies the theoretical
limit of reliable communication over a noisy channel.

    \begin{tcolorbox}[breakable, size=fbox, boxrule=1pt, pad at break*=1mm,colback=cellbackground, colframe=cellborder]
\prompt{In}{incolor}{5}{\boxspacing}
\begin{Verbatim}[commandchars=\\\{\}]
\PY{k}{using}\PY{+w}{ }\PY{n}{Plots}

\PY{c}{\PYZsh{} Define the binary entropy function}
\PY{k}{function}\PY{+w}{ }\PY{n}{H₂}\PY{p}{(}\PY{n}{ε}\PY{p}{)}
\PY{+w}{    }\PY{k}{if}\PY{+w}{ }\PY{n}{ε}\PY{+w}{ }\PY{o}{==}\PY{+w}{ }\PY{l+m+mi}{0}\PY{+w}{ }\PY{o}{||}\PY{+w}{ }\PY{n}{ε}\PY{+w}{ }\PY{o}{==}\PY{+w}{ }\PY{l+m+mi}{1}
\PY{+w}{        }\PY{k}{return}\PY{+w}{ }\PY{l+m+mf}{0.0}
\PY{+w}{    }\PY{k}{end}
\PY{+w}{    }\PY{k}{return}\PY{+w}{ }\PY{o}{\PYZhy{}}\PY{n}{ε}\PY{+w}{ }\PY{o}{*}\PY{+w}{ }\PY{n}{log2}\PY{p}{(}\PY{n}{ε}\PY{p}{)}\PY{+w}{ }\PY{o}{\PYZhy{}}\PY{+w}{ }\PY{p}{(}\PY{l+m+mi}{1}\PY{+w}{ }\PY{o}{\PYZhy{}}\PY{+w}{ }\PY{n}{ε}\PY{p}{)}\PY{+w}{ }\PY{o}{*}\PY{+w}{ }\PY{n}{log2}\PY{p}{(}\PY{l+m+mi}{1}\PY{+w}{ }\PY{o}{\PYZhy{}}\PY{+w}{ }\PY{n}{ε}\PY{p}{)}
\PY{k}{end}

\PY{c}{\PYZsh{} Define the BSC capacity function}
\PY{k}{function}\PY{+w}{ }\PY{n}{Cᵦₛ₍}\PY{p}{(}\PY{n}{ε}\PY{p}{)}
\PY{+w}{    }\PY{k}{return}\PY{+w}{ }\PY{l+m+mi}{1}\PY{+w}{ }\PY{o}{\PYZhy{}}\PY{+w}{ }\PY{n}{H₂}\PY{p}{(}\PY{n}{ε}\PY{p}{)}
\PY{k}{end}

\PY{c}{\PYZsh{} Generate values of ε from 0 to 1}
\PY{n}{ε\PYZus{}values}\PY{+w}{ }\PY{o}{=}\PY{+w}{ }\PY{l+m+mi}{0}\PY{o}{:}\PY{l+m+mf}{0.01}\PY{o}{:}\PY{l+m+mi}{1}
\PY{n}{capacities}\PY{+w}{ }\PY{o}{=}\PY{+w}{ }\PY{n}{Cᵦₛ₍}\PY{o}{.}\PY{p}{(}\PY{n}{ε\PYZus{}values}\PY{p}{)}

\PY{c}{\PYZsh{} Plot the capacity curve}
\PY{n}{plot}\PY{p}{(}\PY{n}{ε\PYZus{}values}\PY{p}{,}\PY{+w}{ }\PY{n}{capacities}\PY{p}{,}
\PY{+w}{     }\PY{n}{label}\PY{+w}{ }\PY{o}{=}\PY{+w}{ }\PY{l+s}{\PYZdq{}}\PY{l+s}{BSC Capacity}\PY{l+s}{\PYZdq{}}
\PY{+w}{    }\PY{p}{,}\PY{+w}{ }\PY{n}{xlabel}\PY{+w}{ }\PY{o}{=}\PY{+w}{ }\PY{l+s}{\PYZdq{}}\PY{l+s}{Error Probability (ε)}\PY{l+s}{\PYZdq{}}\PY{p}{,}\PY{+w}{ }\PY{n}{ylabel}\PY{+w}{ }\PY{o}{=}\PY{+w}{ }\PY{l+s}{\PYZdq{}}\PY{l+s}{Capacity (C)}\PY{l+s}{\PYZdq{}}
\PY{+w}{    }\PY{p}{,}\PY{+w}{ }\PY{n}{title}\PY{+w}{ }\PY{o}{=}\PY{+w}{ }\PY{l+s}{\PYZdq{}}\PY{l+s}{Binary Symmetric Channel (BSC) Capacity}\PY{l+s}{\PYZdq{}}
\PY{+w}{    }\PY{p}{,}\PY{+w}{ }\PY{n}{legend}\PY{+w}{ }\PY{o}{=}\PY{+w}{ }\PY{l+s+ss}{:top}\PY{p}{,}\PY{+w}{ }\PY{n}{size}\PY{+w}{ }\PY{o}{=}\PY{+w}{ }\PY{p}{(}\PY{l+m+mi}{500}\PY{p}{,}\PY{l+m+mi}{400}\PY{p}{)}
\PY{+w}{    }\PY{p}{,}\PY{+w}{ }\PY{n}{xticks}\PY{+w}{ }\PY{o}{=}\PY{+w}{ }\PY{p}{(}\PY{l+m+mi}{0}\PY{o}{:}\PY{l+m+mf}{0.5}\PY{o}{:}\PY{l+m+mi}{1}\PY{p}{,}\PY{+w}{ }\PY{p}{[}\PY{l+s}{\PYZdq{}}\PY{l+s}{0}\PY{l+s}{\PYZdq{}}\PY{p}{,}\PY{+w}{ }\PY{l+s+sa}{L}\PY{l+s}{\PYZdq{}}\PY{l+s+se}{\PYZbs{}f}\PY{l+s}{rac\PYZob{}1\PYZcb{}\PYZob{}2\PYZcb{}}\PY{l+s}{\PYZdq{}}\PY{p}{,}\PY{+w}{ }\PY{l+s}{\PYZdq{}}\PY{l+s}{1}\PY{l+s}{\PYZdq{}}\PY{p}{]}\PY{p}{)}
\PY{p}{)}
\end{Verbatim}
\end{tcolorbox}
 
            
\prompt{Out}{outcolor}{5}{}
    
    \begin{center}
    \adjustimage{max size={0.9\linewidth}{0.9\paperheight}}{REVIEW_files/REVIEW_8_0.pdf}
    \end{center}
    { \hspace*{\fill} \\}
    

    \subsection{Communication System
Components:}\label{communication-system-components}

\$ \begin{gather}
\boxed{\text{Source}} \quad \to \underline{x} = (x_1, \dots, x_n) \to \quad \boxed{\text{Encoder}} \quad  \to \underline{c} = (c_1, \dots, c_n)  \qquad \rceil \\
\qquad \qquad \qquad \qquad \qquad \qquad \qquad \qquad \qquad \qquad \qquad \qquad \qquad \qquad \qquad \boxed{\text{Channel}} \\
\boxed{\text{Sink}} \quad \gets \hat{x} = (\hat{x}_1, \dots, \hat{x}_n) \gets \quad \boxed{\text{Decoder}} \quad  \gets \underline{y} = (y_1, \dots, y_n) \qquad \rfloor
\end{gather} \$

\begin{enumerate}
\def\labelenumi{\arabic{enumi}.}
\tightlist
\item
  \textbf{Source}: Produces the information
  \(\underline{x} = (x_1, \dots, x_n)\).\\
\item
  \textbf{Encoder}: Transforms \(\underline{x}\) into a codeword
  \(\underline{c} = (c_1, \dots, c_n)\), adding redundancy.\\
\item
  \textbf{Channel}: Transmits \(\underline{c}\), introducing errors,
  resulting in \(\underline{y} = (y_1, \dots, y_n)\).\\
\item
  \textbf{Decoder}: Processes \(\underline{y}\) to estimate
  \(\hat{\underline{x}} = (\hat{x}_1, \dots, \hat{x}_n)\), correcting
  errors.\\
\item
  \textbf{Sink}: Receives \(\hat{\underline{x}}\), ideally matching
  \(\underline{x}\).
\end{enumerate}

\subsubsection{Objective:}\label{objective}

Ensure \(\hat{\underline{x}} = \underline{x}\) despite channel errors.

\begin{center}\rule{0.5\linewidth}{0.5pt}\end{center}

\subsubsection{Properties of Linear Block
Code}\label{properties-of-linear-block-code}

\begin{enumerate}
\def\labelenumi{\arabic{enumi}.}
\tightlist
\item
  \textbf{Linearity}:

  \begin{itemize}
  \tightlist
  \item
    The set of codewords \(\mathcal{X}\) forms a \textbf{linear
    subspace} of \(\mathbb{F}_2^n\).
  \item
    Any linear combination of codewords is also a valid codeword:
    \(\underline{v}_1 + \underline{v}_2 \in \mathcal{X}, \; \forall \underline{v}_1, \underline{v}_2 \in \mathcal{X}.\)
  \end{itemize}
\item
  \textbf{Generator Matrix (\(G\))}:

  \begin{itemize}
  \tightlist
  \item
    The \(k \times n\) generator matrix \(G\) maps \(k\)-bit input
    vectors (\(\underline{u} \in \mathbb{F}_2^k\)) to \(n\)-bit
    codewords (\(\underline{v} = \underline{u}^\top G\)).
  \item
    Defines the structure of the codebook \(\mathcal{X}\).
  \end{itemize}
\item
  \textbf{Code Rate (\(R\))}:

  \begin{itemize}
  \tightlist
  \item
    The ratio of information bits to total bits: \(R = \frac{k}{n}\)
  \item
    Indicates the efficiency of the code.
  \end{itemize}
\item
  \textbf{Code Size (\(|\mathcal{X}|\))}:

  \begin{itemize}
  \tightlist
  \item
    The number of unique codewords: \(|\mathcal{X}| = 2^k\)
  \end{itemize}
\item
  \textbf{Minimum Hamming Distance (\(d_\text{min}\))}:

  \begin{itemize}
  \tightlist
  \item
    The smallest Hamming distance between any two distinct codewords.
  \item
    Determines the error-detecting and error-correcting capability:
    \(t = \left\lfloor \frac{d_\text{min} - 1}{2} \right\rfloor\)

    \begin{itemize}
    \tightlist
    \item
      \(t\): Maximum correctable errors.
    \item
      \(d_\text{min} - 1\): Maximum detectable errors.
    \end{itemize}
  \end{itemize}
\item
  \textbf{Parity Check Matrix (\(H\))}:

  \begin{itemize}
  \tightlist
  \item
    The \(H\) matrix defines the null space of \(G\): \(H G^\top = 0\)
  \item
    Used to verify codewords:
    \(H \underline{v}^\top = 0 \implies \underline{v} \in \mathcal{X}.\)
  \end{itemize}
\item
  \textbf{Error Detection and Correction}:

  \begin{itemize}
  \tightlist
  \item
    Error detection: Capable of detecting up to \(d_\text{min} - 1\)
    errors.
  \item
    Error correction: Can correct up to
    \(\lfloor \frac{d_\text{min} - 1}{2} \rfloor\) errors.
  \end{itemize}
\item
  \textbf{Redundancy}:

  \begin{itemize}
  \tightlist
  \item
    The number of redundant bits added for error correction is
    \(n - k\), where \(n\) is the codeword length.
  \end{itemize}
\end{enumerate}

\begin{center}\rule{0.5\linewidth}{0.5pt}\end{center}

\subsubsection{Compact Summary:}\label{compact-summary}

\begin{itemize}
\tightlist
\item
  \textbf{Linear subspace}: Codewords form a subspace of
  \(\mathbb{F}_2^n\).
\item
  \textbf{Size}: \(|\mathcal{X}| = 2^k\).
\item
  \textbf{Rate}: \(R = \frac{k}{n}\).
\item
  \textbf{Error capabilities}: Based on \(d_\text{min}\).
\item
  Defined by:

  \begin{itemize}
  \tightlist
  \item
    \textbf{Generator matrix (\(G\))}.
  \item
    \textbf{Parity check matrix (\(H\))}.
  \end{itemize}
\end{itemize}

\begin{center}\rule{0.5\linewidth}{0.5pt}\end{center}

\paragraph{Error Correction}\label{error-correction}

Error Correction \textbf{Code \(\mathcal{X}\)}\\
Linear error corr. \textbf{Code \(\mathcal{X}\)}

\paragraph{\texorpdfstring{\textbf{Code
Rate}:}{Code Rate:}}\label{code-rate}

\(\boxed{R = \frac{k}{n}}\)

\subparagraph{Explanation:}\label{explanation}

\begin{itemize}
\tightlist
\item
  \textbf{Code Rate} (\(R\)) is the fraction of a codeword used for
  information:

  \begin{itemize}
  \tightlist
  \item
    \(k\): Number of information bits.
  \item
    \(n\): Total number of bits in the codeword (information +
    redundancy).
  \end{itemize}
\item
  \textbf{Trade-off}:

  \begin{itemize}
  \tightlist
  \item
    Higher \(R\) (\(R \to 1\)): More efficient but less error
    protection.
  \item
    Lower \(R\) (\(R \to 0\)): Less efficient but better error
    correction.
  \end{itemize}
\end{itemize}

\paragraph{\texorpdfstring{\textbf{Linear Block Code
Definition}:}{Linear Block Code Definition:}}\label{linear-block-code-definition}

\(\boxed{\mathcal{X} = \{ \underline{v} = \underline{u}^\top G, \; \underline{u} \in \mathbb{F}_2^k \}}\)

\subparagraph{Explanation:}\label{explanation-1}

\begin{itemize}
\tightlist
\item
  \(\mathcal{X}\): The \textbf{set of all codewords} in the code
  (codebook).
\item
  \(\underline{u}\): A binary \textbf{message vector} of length \(k\)
  (\(\underline{u} \in \mathbb{F}_2^k\)).
\item
  \(G\): The \textbf{generator matrix} (\(k \times n\)) used to map
  \(\underline{u}\) to a codeword.
\item
  \(\underline{v}\): A binary \textbf{codeword} of length \(n\)
  (\(\underline{v} \in \mathbb{F}_2^n\)).
\end{itemize}

\subparagraph{Key Points:}\label{key-points}

\begin{enumerate}
\def\labelenumi{\arabic{enumi}.}
\tightlist
\item
  Each \(\underline{u}\) maps to a unique \(\underline{v}\) via
  \(\underline{v} = \underline{u}^\top G\).
\item
  The total number of codewords is \(2^k\) (one for each
  \(\underline{u}\)).
\item
  \(\mathcal{X}\) is a \textbf{linear subspace} of dimension \(k\) in
  \(\mathbb{F}_2^n\).
\end{enumerate}

\subparagraph{Compact Summary:}\label{compact-summary-1}

\(\mathcal{X}\) is the \textbf{codebook} of a linear block code, where
each codeword \(\underline{v}\) is generated by multiplying a \(k\)-bit
message vector \(\underline{u}\) with the generator matrix \(G\).

\begin{center}\rule{0.5\linewidth}{0.5pt}\end{center}

\begin{itemize}
\tightlist
\item
  \textbf{Generator Matrix}: \(G\)
\end{itemize}

\textbf{Code Size}: \(|\mathcal{X}| = 2^k\)

\begin{itemize}
\item
  \(k\)-bit message vectors (\(\underline{u} \in \mathbb{F}_2^k\)) are
  mapped to \(n\)-bit codewords
  (\(\underline{v} = \underline{u}^\top G\)) via the generator matrix
  \(G\).
\item
  The codebook \(\mathcal{X}\) forms a linear subspace with \(2^k\)
  unique codewords, corresponding to the \(k\)-bit input combinations.
\item
  \textbf{Minimum Hamming Distance}:\\
  \(d_\text{min} = \min \{ d_H(x_i, x_j) \}, \; \forall x_i, x_j \in \mathcal{X}, \; x_i \neq x_j\)
  (Hamming distance)
\item
  \textbf{Linear Code Minimum Distance}:\\
  \(d_\text{min} = \min_\limits{\binom{\underline{x}_i, \underline{x}_j \in \mathcal{X}}{x_i \neq x_j}} \big\{ d_H(x_i, x_j) \big\} = \min_\limits{x \in \mathcal{X}} \big\{ W_H(\underline{x}) \big\}\)
\item
  \textbf{Parity Check Relation}:\\
  \(\underline{v} H^\top = 0 \implies \underline{v} \in \mathcal{X} \subseteq \mathbb{F}_2^n\)
\end{itemize}

\subsubsection{Parity Check Matrix}\label{parity-check-matrix}

\begin{itemize}
\tightlist
\item
  \(H \in \mathbb{F}_2^{(n-k) \times n}\)
\item
  \(\dim(\text{Im}(G)) = k\)
\item
  \(H = \text{null}(G^\top), \; \dim = n - k\)
\end{itemize}

    \begin{tcolorbox}[breakable, size=fbox, boxrule=1pt, pad at break*=1mm,colback=cellbackground, colframe=cellborder]
\prompt{In}{incolor}{6}{\boxspacing}
\begin{Verbatim}[commandchars=\\\{\}]
\PY{k}{using}\PY{+w}{ }\PY{n}{LinearAlgebra}

\PY{c}{\PYZsh{} Define the parity check matrix H (size 3x7 for (7, 4) code)}
\PY{n}{H}\PY{+w}{ }\PY{o}{=}\PY{+w}{ }\PY{p}{[}
\PY{+w}{    }\PY{l+m+mi}{1}\PY{+w}{ }\PY{l+m+mi}{0}\PY{+w}{ }\PY{l+m+mi}{0}\PY{+w}{ }\PY{l+m+mi}{1}\PY{+w}{ }\PY{l+m+mi}{1}\PY{+w}{ }\PY{l+m+mi}{1}\PY{+w}{ }\PY{l+m+mi}{0}\PY{p}{;}
\PY{+w}{    }\PY{l+m+mi}{0}\PY{+w}{ }\PY{l+m+mi}{1}\PY{+w}{ }\PY{l+m+mi}{0}\PY{+w}{ }\PY{l+m+mi}{1}\PY{+w}{ }\PY{l+m+mi}{1}\PY{+w}{ }\PY{l+m+mi}{0}\PY{+w}{ }\PY{l+m+mi}{1}\PY{p}{;}
\PY{+w}{    }\PY{l+m+mi}{0}\PY{+w}{ }\PY{l+m+mi}{0}\PY{+w}{ }\PY{l+m+mi}{1}\PY{+w}{ }\PY{l+m+mi}{1}\PY{+w}{ }\PY{l+m+mi}{0}\PY{+w}{ }\PY{l+m+mi}{1}\PY{+w}{ }\PY{l+m+mi}{1}
\PY{p}{]}

\PY{c}{\PYZsh{} Define the generator matrix G}
\PY{n}{G}\PY{+w}{ }\PY{o}{=}\PY{+w}{ }\PY{p}{[}
\PY{+w}{    }\PY{l+m+mi}{1}\PY{+w}{ }\PY{l+m+mi}{0}\PY{+w}{ }\PY{l+m+mi}{0}\PY{+w}{ }\PY{l+m+mi}{0}\PY{+w}{ }\PY{l+m+mi}{1}\PY{+w}{ }\PY{l+m+mi}{1}\PY{+w}{ }\PY{l+m+mi}{0}\PY{p}{;}
\PY{+w}{    }\PY{l+m+mi}{0}\PY{+w}{ }\PY{l+m+mi}{1}\PY{+w}{ }\PY{l+m+mi}{0}\PY{+w}{ }\PY{l+m+mi}{0}\PY{+w}{ }\PY{l+m+mi}{1}\PY{+w}{ }\PY{l+m+mi}{0}\PY{+w}{ }\PY{l+m+mi}{1}\PY{p}{;}
\PY{+w}{    }\PY{l+m+mi}{0}\PY{+w}{ }\PY{l+m+mi}{0}\PY{+w}{ }\PY{l+m+mi}{1}\PY{+w}{ }\PY{l+m+mi}{0}\PY{+w}{ }\PY{l+m+mi}{0}\PY{+w}{ }\PY{l+m+mi}{1}\PY{+w}{ }\PY{l+m+mi}{1}\PY{p}{;}
\PY{+w}{    }\PY{l+m+mi}{0}\PY{+w}{ }\PY{l+m+mi}{0}\PY{+w}{ }\PY{l+m+mi}{0}\PY{+w}{ }\PY{l+m+mi}{1}\PY{+w}{ }\PY{l+m+mi}{1}\PY{+w}{ }\PY{l+m+mi}{1}\PY{+w}{ }\PY{l+m+mi}{1}
\PY{p}{]}\PY{p}{;}
\end{Verbatim}
\end{tcolorbox}

    \begin{tcolorbox}[breakable, size=fbox, boxrule=1pt, pad at break*=1mm,colback=cellbackground, colframe=cellborder]
\prompt{In}{incolor}{7}{\boxspacing}
\begin{Verbatim}[commandchars=\\\{\}]
\PY{c}{\PYZsh{} Message vector}
\PY{n}{u̲}\PY{+w}{ }\PY{o}{=}\PY{+w}{ }\PY{p}{[}\PY{l+m+mi}{1}\PY{+w}{ }\PY{l+m+mi}{0}\PY{+w}{ }\PY{l+m+mi}{1}\PY{+w}{ }\PY{l+m+mi}{0}\PY{p}{]}

\PY{c}{\PYZsh{} Generate codeword}
\PY{n}{v̲}\PY{+w}{ }\PY{o}{=}\PY{+w}{ }\PY{n}{mod}\PY{o}{.}\PY{p}{(}\PY{n}{u̲}\PY{+w}{ }\PY{o}{*}\PY{+w}{ }\PY{n}{G}\PY{p}{,}\PY{+w}{ }\PY{l+m+mi}{2}\PY{p}{)}
\PY{n}{println}\PY{p}{(}\PY{l+s}{\PYZdq{}}\PY{l+s}{Generated codeword: }\PY{l+s}{\PYZdq{}}\PY{p}{,}\PY{+w}{ }\PY{n}{v̲}\PY{p}{)}\PY{+w}{  }\PY{c}{\PYZsh{} Should be a valid codeword}
\end{Verbatim}
\end{tcolorbox}

    \begin{Verbatim}[commandchars=\\\{\}]
Generated codeword: [1 0 1 0 1 0 1]
    \end{Verbatim}

    \begin{tcolorbox}[breakable, size=fbox, boxrule=1pt, pad at break*=1mm,colback=cellbackground, colframe=cellborder]
\prompt{In}{incolor}{8}{\boxspacing}
\begin{Verbatim}[commandchars=\\\{\}]
\PY{c}{\PYZsh{} Parity check validation}
\PY{n}{parity\PYZus{}check}\PY{+w}{ }\PY{o}{=}\PY{+w}{ }\PY{n}{mod}\PY{o}{.}\PY{p}{(}\PY{n}{v̲}\PY{+w}{ }\PY{o}{*}\PY{+w}{ }\PY{n}{H}\PY{o}{\PYZsq{}}\PY{p}{,}\PY{+w}{ }\PY{l+m+mi}{2}\PY{p}{)}
\PY{n}{println}\PY{p}{(}\PY{l+s}{\PYZdq{}}\PY{l+s}{Parity check result: }\PY{l+s}{\PYZdq{}}\PY{p}{,}\PY{+w}{ }\PY{n}{parity\PYZus{}check}\PY{p}{)}\PY{+w}{  }\PY{c}{\PYZsh{} Should be [0, 0, 0]}
\end{Verbatim}
\end{tcolorbox}

    \begin{Verbatim}[commandchars=\\\{\}]
Parity check result: [0 0 0]
    \end{Verbatim}

    \begin{tcolorbox}[breakable, size=fbox, boxrule=1pt, pad at break*=1mm,colback=cellbackground, colframe=cellborder]
\prompt{In}{incolor}{9}{\boxspacing}
\begin{Verbatim}[commandchars=\\\{\}]
\PY{k}{using}\PY{+w}{ }\PY{n}{LinearAlgebra}

\PY{c}{\PYZsh{} Define the generator matrix G (4x7 for (7,4) code)}
\PY{n}{G}\PY{+w}{ }\PY{o}{=}\PY{+w}{ }\PY{p}{[}
\PY{+w}{    }\PY{l+m+mi}{1}\PY{+w}{ }\PY{l+m+mi}{0}\PY{+w}{ }\PY{l+m+mi}{0}\PY{+w}{ }\PY{l+m+mi}{0}\PY{+w}{ }\PY{l+m+mi}{1}\PY{+w}{ }\PY{l+m+mi}{1}\PY{+w}{ }\PY{l+m+mi}{0}\PY{p}{;}
\PY{+w}{    }\PY{l+m+mi}{0}\PY{+w}{ }\PY{l+m+mi}{1}\PY{+w}{ }\PY{l+m+mi}{0}\PY{+w}{ }\PY{l+m+mi}{0}\PY{+w}{ }\PY{l+m+mi}{1}\PY{+w}{ }\PY{l+m+mi}{0}\PY{+w}{ }\PY{l+m+mi}{1}\PY{p}{;}
\PY{+w}{    }\PY{l+m+mi}{0}\PY{+w}{ }\PY{l+m+mi}{0}\PY{+w}{ }\PY{l+m+mi}{1}\PY{+w}{ }\PY{l+m+mi}{0}\PY{+w}{ }\PY{l+m+mi}{0}\PY{+w}{ }\PY{l+m+mi}{1}\PY{+w}{ }\PY{l+m+mi}{1}\PY{p}{;}
\PY{+w}{    }\PY{l+m+mi}{0}\PY{+w}{ }\PY{l+m+mi}{0}\PY{+w}{ }\PY{l+m+mi}{0}\PY{+w}{ }\PY{l+m+mi}{1}\PY{+w}{ }\PY{l+m+mi}{1}\PY{+w}{ }\PY{l+m+mi}{1}\PY{+w}{ }\PY{l+m+mi}{1}
\PY{p}{]}

\PY{c}{\PYZsh{} Generate all possible messages (4\PYZhy{}bit binary combinations)}
\PY{n}{messages}\PY{+w}{ }\PY{o}{=}\PY{+w}{ }\PY{p}{[}\PY{n}{bitstring}\PY{p}{(}\PY{n}{i}\PY{p}{)}\PY{p}{[}\PY{k}{end}\PY{o}{\PYZhy{}}\PY{l+m+mi}{3}\PY{o}{:}\PY{k}{end}\PY{p}{]}\PY{+w}{ }\PY{k}{for}\PY{+w}{ }\PY{n}{i}\PY{+w}{ }\PY{k}{in}\PY{+w}{ }\PY{l+m+mi}{0}\PY{o}{:}\PY{l+m+mi}{2}\PY{o}{\PYZca{}}\PY{l+m+mi}{4}\PY{o}{\PYZhy{}}\PY{l+m+mi}{1}\PY{p}{]}\PY{+w}{  }\PY{c}{\PYZsh{} 4\PYZhy{}bit binary strings}
\PY{n}{u̲}\PY{+w}{ }\PY{o}{=}\PY{+w}{ }\PY{p}{[}\PY{n}{parse}\PY{o}{.}\PY{p}{(}\PY{k+kt}{Int}\PY{p}{,}\PY{+w}{ }\PY{n}{split}\PY{p}{(}\PY{n}{m}\PY{p}{,}\PY{+w}{ }\PY{l+s}{\PYZdq{}}\PY{l+s}{\PYZdq{}}\PY{p}{)}\PY{p}{)}\PY{o}{\PYZsq{}}\PY{+w}{ }\PY{k}{for}\PY{+w}{ }\PY{n}{m}\PY{+w}{ }\PY{k}{in}\PY{+w}{ }\PY{n}{messages}\PY{p}{]}\PY{p}{;}\PY{+w}{ }\PY{n+nd}{@show}\PY{+w}{ }\PY{n}{u̲}\PY{p}{;}\PY{+w}{  }\PY{c}{\PYZsh{} Convert to row vectors (1x4) u\PYZbs{}underbar}

\PY{c}{\PYZsh{} Generate all codewords using G}
\PY{n}{𝒳}\PY{+w}{ }\PY{o}{=}\PY{+w}{ }\PY{p}{[}\PY{n}{mod}\PY{o}{.}\PY{p}{(}\PY{n}{u}\PY{+w}{ }\PY{o}{*}\PY{+w}{ }\PY{n}{G}\PY{p}{,}\PY{+w}{ }\PY{l+m+mi}{2}\PY{p}{)}\PY{+w}{ }\PY{k}{for}\PY{+w}{ }\PY{n}{u}\PY{+w}{ }\PY{k}{in}\PY{+w}{ }\PY{n}{u̲}\PY{+w}{ }\PY{p}{]}\PY{p}{;}\PY{+w}{ }\PY{n+nd}{@show}\PY{+w}{ }\PY{n}{𝒳}

\PY{c}{\PYZsh{} Define a function to compute Hamming distance}
\PY{k}{function}\PY{+w}{ }\PY{n}{hamming\PYZus{}distance}\PY{p}{(}\PY{n}{v₁}\PY{+w}{ }\PY{p}{,}\PY{+w}{ }\PY{n}{v₂}\PY{p}{)}
\PY{+w}{    }\PY{n}{sum}\PY{p}{(}\PY{n}{v₁}\PY{+w}{ }\PY{o}{.!=}\PY{+w}{ }\PY{n}{v₂}\PY{p}{)}\PY{+w}{  }\PY{c}{\PYZsh{} Count differing elements}
\PY{k}{end}

\PY{c}{\PYZsh{} Compute the minimum Hamming distance using a comprehension}
\PY{n}{dₘᵢₙ}\PY{+w}{ }\PY{o}{=}\PY{+w}{ }\PY{n}{minimum}\PY{p}{(}
\PY{+w}{    }\PY{n}{hamming\PYZus{}distance}\PY{p}{(}\PY{n}{𝒳}\PY{p}{[}\PY{n}{i}\PY{p}{]}\PY{p}{,}\PY{+w}{ }\PY{n}{𝒳}\PY{p}{[}\PY{n}{j}\PY{p}{]}\PY{p}{)}\PY{+w}{ }\PY{k}{for}\PY{+w}{ }\PY{p}{(}\PY{n}{i}\PY{p}{,}\PY{+w}{ }\PY{n}{j}\PY{p}{)}\PY{+w}{ }
\PY{+w}{        }\PY{k}{in}\PY{+w}{ }\PY{n}{Iterators}\PY{o}{.}\PY{n}{product}\PY{p}{(}\PY{l+m+mi}{1}\PY{o}{:}\PY{n}{length}\PY{p}{(}\PY{n}{𝒳}\PY{p}{)}\PY{p}{,}\PY{+w}{ }\PY{l+m+mi}{1}\PY{o}{:}\PY{n}{length}\PY{p}{(}\PY{n}{𝒳}\PY{p}{)}\PY{p}{)}\PY{+w}{ }\PY{k}{if}\PY{+w}{ }\PY{n}{i}\PY{+w}{ }\PY{o}{\PYZlt{}}\PY{+w}{ }\PY{n}{j}
\PY{p}{)}

\PY{c}{\PYZsh{} Output the result}
\PY{n}{println}\PY{p}{(}\PY{l+s}{\PYZdq{}}\PY{l+s}{Minimum distance of the code: }\PY{l+s}{\PYZdq{}}\PY{p}{,}\PY{+w}{ }\PY{n}{dₘᵢₙ}\PY{p}{)}
\end{Verbatim}
\end{tcolorbox}

    \begin{Verbatim}[commandchars=\\\{\}]
u̲ = Adjoint\{Int64, Vector\{Int64\}\}[[0 0 0 0], [0 0 0 1], [0 0 1 0], [0 0 1 1],
[0 1 0 0], [0 1 0 1], [0 1 1 0], [0 1 1 1], [1 0 0 0], [1 0 0 1], [1 0 1 0], [1
0 1 1], [1 1 0 0], [1 1 0 1], [1 1 1 0], [1 1 1 1]]
𝒳 = [[0 0 0 0 0 0 0], [0 0 0 1 1 1 1], [0 0 1 0 0 1 1], [0 0 1 1 1 0 0], [0 1 0
0 1 0 1], [0 1 0 1 0 1 0], [0 1 1 0 1 1 0], [0 1 1 1 0 0 1], [1 0 0 0 1 1 0], [1
0 0 1 0 0 1], [1 0 1 0 1 0 1], [1 0 1 1 0 1 0], [1 1 0 0 0 1 1], [1 1 0 1 1 0
0], [1 1 1 0 0 0 0], [1 1 1 1 1 1 1]]
Minimum distance of the code: 3
    \end{Verbatim}


    % Add a bibliography block to the postdoc
    
    
    
\end{document}
