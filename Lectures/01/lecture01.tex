\documentclass[11pt]{article}

    \usepackage[breakable]{tcolorbox}
    \usepackage{parskip} % Stop auto-indenting (to mimic markdown behaviour)
    

    % Basic figure setup, for now with no caption control since it's done
    % automatically by Pandoc (which extracts ![](path) syntax from Markdown).
    \usepackage{graphicx}
    % Keep aspect ratio if custom image width or height is specified
    \setkeys{Gin}{keepaspectratio}
    % Maintain compatibility with old templates. Remove in nbconvert 6.0
    \let\Oldincludegraphics\includegraphics
    % Ensure that by default, figures have no caption (until we provide a
    % proper Figure object with a Caption API and a way to capture that
    % in the conversion process - todo).
    \usepackage{caption}
    \DeclareCaptionFormat{nocaption}{}
    \captionsetup{format=nocaption,aboveskip=0pt,belowskip=0pt}

    \usepackage{float}
    \floatplacement{figure}{H} % forces figures to be placed at the correct location
    \usepackage{xcolor} % Allow colors to be defined
    \usepackage{enumerate} % Needed for markdown enumerations to work
    \usepackage{geometry} % Used to adjust the document margins
    \usepackage{amsmath} % Equations
    \usepackage{amssymb} % Equations
    \usepackage{textcomp} % defines textquotesingle
    % Hack from http://tex.stackexchange.com/a/47451/13684:
    \AtBeginDocument{%
        \def\PYZsq{\textquotesingle}% Upright quotes in Pygmentized code
    }
    \usepackage{upquote} % Upright quotes for verbatim code
    \usepackage{eurosym} % defines \euro

    \usepackage{iftex}
    \ifPDFTeX
        \usepackage[T1]{fontenc}
        \IfFileExists{alphabeta.sty}{
              \usepackage{alphabeta}
          }{
              \usepackage[mathletters]{ucs}
              \usepackage[utf8x]{inputenc}
          }
    \else
        \usepackage{fontspec}
        \usepackage{unicode-math}
    \fi

    \usepackage{fancyvrb} % verbatim replacement that allows latex
    \usepackage{grffile} % extends the file name processing of package graphics
                         % to support a larger range
    \makeatletter % fix for old versions of grffile with XeLaTeX
    \@ifpackagelater{grffile}{2019/11/01}
    {
      % Do nothing on new versions
    }
    {
      \def\Gread@@xetex#1{%
        \IfFileExists{"\Gin@base".bb}%
        {\Gread@eps{\Gin@base.bb}}%
        {\Gread@@xetex@aux#1}%
      }
    }
    \makeatother
    \usepackage[Export]{adjustbox} % Used to constrain images to a maximum size
    \adjustboxset{max size={0.9\linewidth}{0.9\paperheight}}

    % The hyperref package gives us a pdf with properly built
    % internal navigation ('pdf bookmarks' for the table of contents,
    % internal cross-reference links, web links for URLs, etc.)
    \usepackage{hyperref}
    % The default LaTeX title has an obnoxious amount of whitespace. By default,
    % titling removes some of it. It also provides customization options.
    \usepackage{titling}
    \usepackage{longtable} % longtable support required by pandoc >1.10
    \usepackage{booktabs}  % table support for pandoc > 1.12.2
    \usepackage{array}     % table support for pandoc >= 2.11.3
    \usepackage{calc}      % table minipage width calculation for pandoc >= 2.11.1
    \usepackage[inline]{enumitem} % IRkernel/repr support (it uses the enumerate* environment)
    \usepackage[normalem]{ulem} % ulem is needed to support strikethroughs (\sout)
                                % normalem makes italics be italics, not underlines
    \usepackage{soul}      % strikethrough (\st) support for pandoc >= 3.0.0
    \usepackage{mathrsfs}
    

    
    % Colors for the hyperref package
    \definecolor{urlcolor}{rgb}{0,.145,.698}
    \definecolor{linkcolor}{rgb}{.71,0.21,0.01}
    \definecolor{citecolor}{rgb}{.12,.54,.11}

    % ANSI colors
    \definecolor{ansi-black}{HTML}{3E424D}
    \definecolor{ansi-black-intense}{HTML}{282C36}
    \definecolor{ansi-red}{HTML}{E75C58}
    \definecolor{ansi-red-intense}{HTML}{B22B31}
    \definecolor{ansi-green}{HTML}{00A250}
    \definecolor{ansi-green-intense}{HTML}{007427}
    \definecolor{ansi-yellow}{HTML}{DDB62B}
    \definecolor{ansi-yellow-intense}{HTML}{B27D12}
    \definecolor{ansi-blue}{HTML}{208FFB}
    \definecolor{ansi-blue-intense}{HTML}{0065CA}
    \definecolor{ansi-magenta}{HTML}{D160C4}
    \definecolor{ansi-magenta-intense}{HTML}{A03196}
    \definecolor{ansi-cyan}{HTML}{60C6C8}
    \definecolor{ansi-cyan-intense}{HTML}{258F8F}
    \definecolor{ansi-white}{HTML}{C5C1B4}
    \definecolor{ansi-white-intense}{HTML}{A1A6B2}
    \definecolor{ansi-default-inverse-fg}{HTML}{FFFFFF}
    \definecolor{ansi-default-inverse-bg}{HTML}{000000}

    % common color for the border for error outputs.
    \definecolor{outerrorbackground}{HTML}{FFDFDF}

    % commands and environments needed by pandoc snippets
    % extracted from the output of `pandoc -s`
    \providecommand{\tightlist}{%
      \setlength{\itemsep}{0pt}\setlength{\parskip}{0pt}}
    \DefineVerbatimEnvironment{Highlighting}{Verbatim}{commandchars=\\\{\}}
    % Add ',fontsize=\small' for more characters per line
    \newenvironment{Shaded}{}{}
    \newcommand{\KeywordTok}[1]{\textcolor[rgb]{0.00,0.44,0.13}{\textbf{{#1}}}}
    \newcommand{\DataTypeTok}[1]{\textcolor[rgb]{0.56,0.13,0.00}{{#1}}}
    \newcommand{\DecValTok}[1]{\textcolor[rgb]{0.25,0.63,0.44}{{#1}}}
    \newcommand{\BaseNTok}[1]{\textcolor[rgb]{0.25,0.63,0.44}{{#1}}}
    \newcommand{\FloatTok}[1]{\textcolor[rgb]{0.25,0.63,0.44}{{#1}}}
    \newcommand{\CharTok}[1]{\textcolor[rgb]{0.25,0.44,0.63}{{#1}}}
    \newcommand{\StringTok}[1]{\textcolor[rgb]{0.25,0.44,0.63}{{#1}}}
    \newcommand{\CommentTok}[1]{\textcolor[rgb]{0.38,0.63,0.69}{\textit{{#1}}}}
    \newcommand{\OtherTok}[1]{\textcolor[rgb]{0.00,0.44,0.13}{{#1}}}
    \newcommand{\AlertTok}[1]{\textcolor[rgb]{1.00,0.00,0.00}{\textbf{{#1}}}}
    \newcommand{\FunctionTok}[1]{\textcolor[rgb]{0.02,0.16,0.49}{{#1}}}
    \newcommand{\RegionMarkerTok}[1]{{#1}}
    \newcommand{\ErrorTok}[1]{\textcolor[rgb]{1.00,0.00,0.00}{\textbf{{#1}}}}
    \newcommand{\NormalTok}[1]{{#1}}

    % Additional commands for more recent versions of Pandoc
    \newcommand{\ConstantTok}[1]{\textcolor[rgb]{0.53,0.00,0.00}{{#1}}}
    \newcommand{\SpecialCharTok}[1]{\textcolor[rgb]{0.25,0.44,0.63}{{#1}}}
    \newcommand{\VerbatimStringTok}[1]{\textcolor[rgb]{0.25,0.44,0.63}{{#1}}}
    \newcommand{\SpecialStringTok}[1]{\textcolor[rgb]{0.73,0.40,0.53}{{#1}}}
    \newcommand{\ImportTok}[1]{{#1}}
    \newcommand{\DocumentationTok}[1]{\textcolor[rgb]{0.73,0.13,0.13}{\textit{{#1}}}}
    \newcommand{\AnnotationTok}[1]{\textcolor[rgb]{0.38,0.63,0.69}{\textbf{\textit{{#1}}}}}
    \newcommand{\CommentVarTok}[1]{\textcolor[rgb]{0.38,0.63,0.69}{\textbf{\textit{{#1}}}}}
    \newcommand{\VariableTok}[1]{\textcolor[rgb]{0.10,0.09,0.49}{{#1}}}
    \newcommand{\ControlFlowTok}[1]{\textcolor[rgb]{0.00,0.44,0.13}{\textbf{{#1}}}}
    \newcommand{\OperatorTok}[1]{\textcolor[rgb]{0.40,0.40,0.40}{{#1}}}
    \newcommand{\BuiltInTok}[1]{{#1}}
    \newcommand{\ExtensionTok}[1]{{#1}}
    \newcommand{\PreprocessorTok}[1]{\textcolor[rgb]{0.74,0.48,0.00}{{#1}}}
    \newcommand{\AttributeTok}[1]{\textcolor[rgb]{0.49,0.56,0.16}{{#1}}}
    \newcommand{\InformationTok}[1]{\textcolor[rgb]{0.38,0.63,0.69}{\textbf{\textit{{#1}}}}}
    \newcommand{\WarningTok}[1]{\textcolor[rgb]{0.38,0.63,0.69}{\textbf{\textit{{#1}}}}}


    % Define a nice break command that doesn't care if a line doesn't already
    % exist.
    \def\br{\hspace*{\fill} \\* }
    % Math Jax compatibility definitions
    \def\gt{>}
    \def\lt{<}
    \let\Oldtex\TeX
    \let\Oldlatex\LaTeX
    \renewcommand{\TeX}{\textrm{\Oldtex}}
    \renewcommand{\LaTeX}{\textrm{\Oldlatex}}
    % Document parameters
    % Document title
    \title{lecture01}
    
    
    
    
    
    
    
% Pygments definitions
\makeatletter
\def\PY@reset{\let\PY@it=\relax \let\PY@bf=\relax%
    \let\PY@ul=\relax \let\PY@tc=\relax%
    \let\PY@bc=\relax \let\PY@ff=\relax}
\def\PY@tok#1{\csname PY@tok@#1\endcsname}
\def\PY@toks#1+{\ifx\relax#1\empty\else%
    \PY@tok{#1}\expandafter\PY@toks\fi}
\def\PY@do#1{\PY@bc{\PY@tc{\PY@ul{%
    \PY@it{\PY@bf{\PY@ff{#1}}}}}}}
\def\PY#1#2{\PY@reset\PY@toks#1+\relax+\PY@do{#2}}

\@namedef{PY@tok@w}{\def\PY@tc##1{\textcolor[rgb]{0.73,0.73,0.73}{##1}}}
\@namedef{PY@tok@c}{\let\PY@it=\textit\def\PY@tc##1{\textcolor[rgb]{0.24,0.48,0.48}{##1}}}
\@namedef{PY@tok@cp}{\def\PY@tc##1{\textcolor[rgb]{0.61,0.40,0.00}{##1}}}
\@namedef{PY@tok@k}{\let\PY@bf=\textbf\def\PY@tc##1{\textcolor[rgb]{0.00,0.50,0.00}{##1}}}
\@namedef{PY@tok@kp}{\def\PY@tc##1{\textcolor[rgb]{0.00,0.50,0.00}{##1}}}
\@namedef{PY@tok@kt}{\def\PY@tc##1{\textcolor[rgb]{0.69,0.00,0.25}{##1}}}
\@namedef{PY@tok@o}{\def\PY@tc##1{\textcolor[rgb]{0.40,0.40,0.40}{##1}}}
\@namedef{PY@tok@ow}{\let\PY@bf=\textbf\def\PY@tc##1{\textcolor[rgb]{0.67,0.13,1.00}{##1}}}
\@namedef{PY@tok@nb}{\def\PY@tc##1{\textcolor[rgb]{0.00,0.50,0.00}{##1}}}
\@namedef{PY@tok@nf}{\def\PY@tc##1{\textcolor[rgb]{0.00,0.00,1.00}{##1}}}
\@namedef{PY@tok@nc}{\let\PY@bf=\textbf\def\PY@tc##1{\textcolor[rgb]{0.00,0.00,1.00}{##1}}}
\@namedef{PY@tok@nn}{\let\PY@bf=\textbf\def\PY@tc##1{\textcolor[rgb]{0.00,0.00,1.00}{##1}}}
\@namedef{PY@tok@ne}{\let\PY@bf=\textbf\def\PY@tc##1{\textcolor[rgb]{0.80,0.25,0.22}{##1}}}
\@namedef{PY@tok@nv}{\def\PY@tc##1{\textcolor[rgb]{0.10,0.09,0.49}{##1}}}
\@namedef{PY@tok@no}{\def\PY@tc##1{\textcolor[rgb]{0.53,0.00,0.00}{##1}}}
\@namedef{PY@tok@nl}{\def\PY@tc##1{\textcolor[rgb]{0.46,0.46,0.00}{##1}}}
\@namedef{PY@tok@ni}{\let\PY@bf=\textbf\def\PY@tc##1{\textcolor[rgb]{0.44,0.44,0.44}{##1}}}
\@namedef{PY@tok@na}{\def\PY@tc##1{\textcolor[rgb]{0.41,0.47,0.13}{##1}}}
\@namedef{PY@tok@nt}{\let\PY@bf=\textbf\def\PY@tc##1{\textcolor[rgb]{0.00,0.50,0.00}{##1}}}
\@namedef{PY@tok@nd}{\def\PY@tc##1{\textcolor[rgb]{0.67,0.13,1.00}{##1}}}
\@namedef{PY@tok@s}{\def\PY@tc##1{\textcolor[rgb]{0.73,0.13,0.13}{##1}}}
\@namedef{PY@tok@sd}{\let\PY@it=\textit\def\PY@tc##1{\textcolor[rgb]{0.73,0.13,0.13}{##1}}}
\@namedef{PY@tok@si}{\let\PY@bf=\textbf\def\PY@tc##1{\textcolor[rgb]{0.64,0.35,0.47}{##1}}}
\@namedef{PY@tok@se}{\let\PY@bf=\textbf\def\PY@tc##1{\textcolor[rgb]{0.67,0.36,0.12}{##1}}}
\@namedef{PY@tok@sr}{\def\PY@tc##1{\textcolor[rgb]{0.64,0.35,0.47}{##1}}}
\@namedef{PY@tok@ss}{\def\PY@tc##1{\textcolor[rgb]{0.10,0.09,0.49}{##1}}}
\@namedef{PY@tok@sx}{\def\PY@tc##1{\textcolor[rgb]{0.00,0.50,0.00}{##1}}}
\@namedef{PY@tok@m}{\def\PY@tc##1{\textcolor[rgb]{0.40,0.40,0.40}{##1}}}
\@namedef{PY@tok@gh}{\let\PY@bf=\textbf\def\PY@tc##1{\textcolor[rgb]{0.00,0.00,0.50}{##1}}}
\@namedef{PY@tok@gu}{\let\PY@bf=\textbf\def\PY@tc##1{\textcolor[rgb]{0.50,0.00,0.50}{##1}}}
\@namedef{PY@tok@gd}{\def\PY@tc##1{\textcolor[rgb]{0.63,0.00,0.00}{##1}}}
\@namedef{PY@tok@gi}{\def\PY@tc##1{\textcolor[rgb]{0.00,0.52,0.00}{##1}}}
\@namedef{PY@tok@gr}{\def\PY@tc##1{\textcolor[rgb]{0.89,0.00,0.00}{##1}}}
\@namedef{PY@tok@ge}{\let\PY@it=\textit}
\@namedef{PY@tok@gs}{\let\PY@bf=\textbf}
\@namedef{PY@tok@ges}{\let\PY@bf=\textbf\let\PY@it=\textit}
\@namedef{PY@tok@gp}{\let\PY@bf=\textbf\def\PY@tc##1{\textcolor[rgb]{0.00,0.00,0.50}{##1}}}
\@namedef{PY@tok@go}{\def\PY@tc##1{\textcolor[rgb]{0.44,0.44,0.44}{##1}}}
\@namedef{PY@tok@gt}{\def\PY@tc##1{\textcolor[rgb]{0.00,0.27,0.87}{##1}}}
\@namedef{PY@tok@err}{\def\PY@bc##1{{\setlength{\fboxsep}{\string -\fboxrule}\fcolorbox[rgb]{1.00,0.00,0.00}{1,1,1}{\strut ##1}}}}
\@namedef{PY@tok@kc}{\let\PY@bf=\textbf\def\PY@tc##1{\textcolor[rgb]{0.00,0.50,0.00}{##1}}}
\@namedef{PY@tok@kd}{\let\PY@bf=\textbf\def\PY@tc##1{\textcolor[rgb]{0.00,0.50,0.00}{##1}}}
\@namedef{PY@tok@kn}{\let\PY@bf=\textbf\def\PY@tc##1{\textcolor[rgb]{0.00,0.50,0.00}{##1}}}
\@namedef{PY@tok@kr}{\let\PY@bf=\textbf\def\PY@tc##1{\textcolor[rgb]{0.00,0.50,0.00}{##1}}}
\@namedef{PY@tok@bp}{\def\PY@tc##1{\textcolor[rgb]{0.00,0.50,0.00}{##1}}}
\@namedef{PY@tok@fm}{\def\PY@tc##1{\textcolor[rgb]{0.00,0.00,1.00}{##1}}}
\@namedef{PY@tok@vc}{\def\PY@tc##1{\textcolor[rgb]{0.10,0.09,0.49}{##1}}}
\@namedef{PY@tok@vg}{\def\PY@tc##1{\textcolor[rgb]{0.10,0.09,0.49}{##1}}}
\@namedef{PY@tok@vi}{\def\PY@tc##1{\textcolor[rgb]{0.10,0.09,0.49}{##1}}}
\@namedef{PY@tok@vm}{\def\PY@tc##1{\textcolor[rgb]{0.10,0.09,0.49}{##1}}}
\@namedef{PY@tok@sa}{\def\PY@tc##1{\textcolor[rgb]{0.73,0.13,0.13}{##1}}}
\@namedef{PY@tok@sb}{\def\PY@tc##1{\textcolor[rgb]{0.73,0.13,0.13}{##1}}}
\@namedef{PY@tok@sc}{\def\PY@tc##1{\textcolor[rgb]{0.73,0.13,0.13}{##1}}}
\@namedef{PY@tok@dl}{\def\PY@tc##1{\textcolor[rgb]{0.73,0.13,0.13}{##1}}}
\@namedef{PY@tok@s2}{\def\PY@tc##1{\textcolor[rgb]{0.73,0.13,0.13}{##1}}}
\@namedef{PY@tok@sh}{\def\PY@tc##1{\textcolor[rgb]{0.73,0.13,0.13}{##1}}}
\@namedef{PY@tok@s1}{\def\PY@tc##1{\textcolor[rgb]{0.73,0.13,0.13}{##1}}}
\@namedef{PY@tok@mb}{\def\PY@tc##1{\textcolor[rgb]{0.40,0.40,0.40}{##1}}}
\@namedef{PY@tok@mf}{\def\PY@tc##1{\textcolor[rgb]{0.40,0.40,0.40}{##1}}}
\@namedef{PY@tok@mh}{\def\PY@tc##1{\textcolor[rgb]{0.40,0.40,0.40}{##1}}}
\@namedef{PY@tok@mi}{\def\PY@tc##1{\textcolor[rgb]{0.40,0.40,0.40}{##1}}}
\@namedef{PY@tok@il}{\def\PY@tc##1{\textcolor[rgb]{0.40,0.40,0.40}{##1}}}
\@namedef{PY@tok@mo}{\def\PY@tc##1{\textcolor[rgb]{0.40,0.40,0.40}{##1}}}
\@namedef{PY@tok@ch}{\let\PY@it=\textit\def\PY@tc##1{\textcolor[rgb]{0.24,0.48,0.48}{##1}}}
\@namedef{PY@tok@cm}{\let\PY@it=\textit\def\PY@tc##1{\textcolor[rgb]{0.24,0.48,0.48}{##1}}}
\@namedef{PY@tok@cpf}{\let\PY@it=\textit\def\PY@tc##1{\textcolor[rgb]{0.24,0.48,0.48}{##1}}}
\@namedef{PY@tok@c1}{\let\PY@it=\textit\def\PY@tc##1{\textcolor[rgb]{0.24,0.48,0.48}{##1}}}
\@namedef{PY@tok@cs}{\let\PY@it=\textit\def\PY@tc##1{\textcolor[rgb]{0.24,0.48,0.48}{##1}}}

\def\PYZbs{\char`\\}
\def\PYZus{\char`\_}
\def\PYZob{\char`\{}
\def\PYZcb{\char`\}}
\def\PYZca{\char`\^}
\def\PYZam{\char`\&}
\def\PYZlt{\char`\<}
\def\PYZgt{\char`\>}
\def\PYZsh{\char`\#}
\def\PYZpc{\char`\%}
\def\PYZdl{\char`\$}
\def\PYZhy{\char`\-}
\def\PYZsq{\char`\'}
\def\PYZdq{\char`\"}
\def\PYZti{\char`\~}
% for compatibility with earlier versions
\def\PYZat{@}
\def\PYZlb{[}
\def\PYZrb{]}
\makeatother


    % For linebreaks inside Verbatim environment from package fancyvrb.
    \makeatletter
        \newbox\Wrappedcontinuationbox
        \newbox\Wrappedvisiblespacebox
        \newcommand*\Wrappedvisiblespace {\textcolor{red}{\textvisiblespace}}
        \newcommand*\Wrappedcontinuationsymbol {\textcolor{red}{\llap{\tiny$\m@th\hookrightarrow$}}}
        \newcommand*\Wrappedcontinuationindent {3ex }
        \newcommand*\Wrappedafterbreak {\kern\Wrappedcontinuationindent\copy\Wrappedcontinuationbox}
        % Take advantage of the already applied Pygments mark-up to insert
        % potential linebreaks for TeX processing.
        %        {, <, #, %, $, ' and ": go to next line.
        %        _, }, ^, &, >, - and ~: stay at end of broken line.
        % Use of \textquotesingle for straight quote.
        \newcommand*\Wrappedbreaksatspecials {%
            \def\PYGZus{\discretionary{\char`\_}{\Wrappedafterbreak}{\char`\_}}%
            \def\PYGZob{\discretionary{}{\Wrappedafterbreak\char`\{}{\char`\{}}%
            \def\PYGZcb{\discretionary{\char`\}}{\Wrappedafterbreak}{\char`\}}}%
            \def\PYGZca{\discretionary{\char`\^}{\Wrappedafterbreak}{\char`\^}}%
            \def\PYGZam{\discretionary{\char`\&}{\Wrappedafterbreak}{\char`\&}}%
            \def\PYGZlt{\discretionary{}{\Wrappedafterbreak\char`\<}{\char`\<}}%
            \def\PYGZgt{\discretionary{\char`\>}{\Wrappedafterbreak}{\char`\>}}%
            \def\PYGZsh{\discretionary{}{\Wrappedafterbreak\char`\#}{\char`\#}}%
            \def\PYGZpc{\discretionary{}{\Wrappedafterbreak\char`\%}{\char`\%}}%
            \def\PYGZdl{\discretionary{}{\Wrappedafterbreak\char`\$}{\char`\$}}%
            \def\PYGZhy{\discretionary{\char`\-}{\Wrappedafterbreak}{\char`\-}}%
            \def\PYGZsq{\discretionary{}{\Wrappedafterbreak\textquotesingle}{\textquotesingle}}%
            \def\PYGZdq{\discretionary{}{\Wrappedafterbreak\char`\"}{\char`\"}}%
            \def\PYGZti{\discretionary{\char`\~}{\Wrappedafterbreak}{\char`\~}}%
        }
        % Some characters . , ; ? ! / are not pygmentized.
        % This macro makes them "active" and they will insert potential linebreaks
        \newcommand*\Wrappedbreaksatpunct {%
            \lccode`\~`\.\lowercase{\def~}{\discretionary{\hbox{\char`\.}}{\Wrappedafterbreak}{\hbox{\char`\.}}}%
            \lccode`\~`\,\lowercase{\def~}{\discretionary{\hbox{\char`\,}}{\Wrappedafterbreak}{\hbox{\char`\,}}}%
            \lccode`\~`\;\lowercase{\def~}{\discretionary{\hbox{\char`\;}}{\Wrappedafterbreak}{\hbox{\char`\;}}}%
            \lccode`\~`\:\lowercase{\def~}{\discretionary{\hbox{\char`\:}}{\Wrappedafterbreak}{\hbox{\char`\:}}}%
            \lccode`\~`\?\lowercase{\def~}{\discretionary{\hbox{\char`\?}}{\Wrappedafterbreak}{\hbox{\char`\?}}}%
            \lccode`\~`\!\lowercase{\def~}{\discretionary{\hbox{\char`\!}}{\Wrappedafterbreak}{\hbox{\char`\!}}}%
            \lccode`\~`\/\lowercase{\def~}{\discretionary{\hbox{\char`\/}}{\Wrappedafterbreak}{\hbox{\char`\/}}}%
            \catcode`\.\active
            \catcode`\,\active
            \catcode`\;\active
            \catcode`\:\active
            \catcode`\?\active
            \catcode`\!\active
            \catcode`\/\active
            \lccode`\~`\~
        }
    \makeatother

    \let\OriginalVerbatim=\Verbatim
    \makeatletter
    \renewcommand{\Verbatim}[1][1]{%
        %\parskip\z@skip
        \sbox\Wrappedcontinuationbox {\Wrappedcontinuationsymbol}%
        \sbox\Wrappedvisiblespacebox {\FV@SetupFont\Wrappedvisiblespace}%
        \def\FancyVerbFormatLine ##1{\hsize\linewidth
            \vtop{\raggedright\hyphenpenalty\z@\exhyphenpenalty\z@
                \doublehyphendemerits\z@\finalhyphendemerits\z@
                \strut ##1\strut}%
        }%
        % If the linebreak is at a space, the latter will be displayed as visible
        % space at end of first line, and a continuation symbol starts next line.
        % Stretch/shrink are however usually zero for typewriter font.
        \def\FV@Space {%
            \nobreak\hskip\z@ plus\fontdimen3\font minus\fontdimen4\font
            \discretionary{\copy\Wrappedvisiblespacebox}{\Wrappedafterbreak}
            {\kern\fontdimen2\font}%
        }%

        % Allow breaks at special characters using \PYG... macros.
        \Wrappedbreaksatspecials
        % Breaks at punctuation characters . , ; ? ! and / need catcode=\active
        \OriginalVerbatim[#1,codes*=\Wrappedbreaksatpunct]%
    }
    \makeatother

    % Exact colors from NB
    \definecolor{incolor}{HTML}{303F9F}
    \definecolor{outcolor}{HTML}{D84315}
    \definecolor{cellborder}{HTML}{CFCFCF}
    \definecolor{cellbackground}{HTML}{F7F7F7}

    % prompt
    \makeatletter
    \newcommand{\boxspacing}{\kern\kvtcb@left@rule\kern\kvtcb@boxsep}
    \makeatother
    \newcommand{\prompt}[4]{
        {\ttfamily\llap{{\color{#2}[#3]:\hspace{3pt}#4}}\vspace{-\baselineskip}}
    }
    

    
    % Prevent overflowing lines due to hard-to-break entities
    \sloppy
    % Setup hyperref package
    \hypersetup{
      breaklinks=true,  % so long urls are correctly broken across lines
      colorlinks=true,
      urlcolor=urlcolor,
      linkcolor=linkcolor,
      citecolor=citecolor,
      }
    % Slightly bigger margins than the latex defaults
    
    \geometry{verbose,tmargin=1in,bmargin=1in,lmargin=1in,rmargin=1in}
    
    

\begin{document}
    
    \maketitle
    
    

    
    \section{📜 Lecture 1}\label{lecture-1}

Small-scale multipath fading is more relevant to the design of reliable
and efficient communication systems - The focus of the book (Fundamental
of WIRELESS COMMUNICATION)

    \subsection{❍ Communication (Simplistic
View)}\label{communication-simplistic-view}

Transmitted sinusoid: \(x(t) = \cos(2 \pi \mathcal{f} t)\) where : -
\(\mathcal{f}\) is frequency \(\approx 1 \to 4\) Ghz - \(t\) is time

    \begin{equation}
\text{Rx: } E(f, t, r, \theta, \psi) = \frac{1}{r} \cdot {\color{yellow}\alpha}_s(\theta, \psi, f) \cdot \cos\left( 2 \pi f \left( t - {\color{orange}\frac{r}{c}} \right) \right)
\end{equation}

\begin{itemize}
\tightlist
\item
  \({\color{yellow}\alpha}\): antenna losses
\item
  \({\color{orange}\frac{r}{c}}\) : delay due to signal
\end{itemize}

\textbf{Intuition why \(E \propto \frac{1}{r}\):}

\begin{itemize}
\tightlist
\item
  Tx in 3-D space.
\item
  Power preserved in surface of sphere: (radius \(r\), center is
  antenna).
\item
  Fixed \(\frac{\text{Total Power}}{\text{ Area }}\) → Rx power
  \(\propto \frac{1}{4 \pi r^2}\) → \(\propto \frac{1}{r^2}\).
\end{itemize}

    \subsubsection{❍ Consider Movement}\label{consider-movement}

\begin{align}
E(f, t, r(t)) &= \frac{\alpha_s}{r(t)} \cdot \cos 2 \pi f \left( t - \frac{r(t)}{c} \right) \qquad \text{like before: } r \to r(t) \\
&= \frac{\alpha_s}{r_0 + v \cdot t} \cdot \cos 2 \pi f \left( t - \frac{r_0}{c} - \frac{v \cdot t}{c} \right)\\
&= \frac{\alpha_s}{r_0 + v \cdot t} \cdot \cos 2 \pi f \left[t \cdot \left( 1 - \frac{v}{c} \right) - \frac{r_0}{c} \right]
\end{align}

\begin{itemize}
\item
  Effective frequency change
  \(f t \to f \cdot \left( 1 - \frac{v}{c} \right) \cdot t\) :
  \(f \to f \cdot \left( 1 - \frac{v}{c} \right)\)

  \begin{itemize}
  \tightlist
  \item
    frequency reduction
    \(f t \to f \cdot \left( 1 - \frac{v}{c} \right) \cdot t\)
  \item
    \textbf{Doppler shift}: \(D = -f \frac{v}{c}\)
  \end{itemize}
\item
  \[
  E(f, t, r(t)) = {\color{yellow}\frac{\alpha_s}{r_0 + v \cdot t}} \cdot \cos 2 \pi f \left( t \cdot \left( 1 - \frac{v}{c} \right) - \frac{r_0}{c} \right)
  \]

  \textbf{Note}: \({\color{yellow}\frac{\alpha_s}{r_0 + v \cdot t}}\)
  ``Not LTI now.''
\end{itemize}

    \subsubsection{❍ Fixed Tx, Fixed Rx, Perfectly Reflecting
Wall}\label{fixed-tx-fixed-rx-perfectly-reflecting-wall}

\textbf{Use method of ``ray tracing'': consider dominant (main) paths.}

\[
E_r(f, t) = \frac{\alpha}{r} \cos 2 \pi f \left( t - \frac{r}{c} \right) - \frac{\alpha}{2d - r} \cos 2 \pi f \left( t - \frac{2d - r}{c} \right)
\]

\$ \qquad \qquad \qquad r \$ is \textbf{distance 1}, \$ \qquad 2d - r \$
= \textbf{distance 2}.

\textbf{Superposition (addition of two sinusoids, with different
phases):}

\begin{itemize}
\tightlist
\item
  \textbf{\(\color{red}\text{Possible effect of cancelling out}\)}.
\end{itemize}

\textbf{Phase difference}: \$ \Delta \phi = \phi\_2 + \{\color{orange}
\pi \} - \phi\_1 \$

\textbf{Interpretation}: - ( \textbf{\({\color{orange} \pi }\) phase
shift}: in the electric field of an EM wave when reflected from an
optically denser medium. ) - ( This causes \textbf{essentially
cancelling} of the signals close to the wall. )

\[
\Delta \phi = 2 \pi f \frac{2d - r}{c} + \pi - 2 \pi f \frac{r}{c}
\]

    \subsubsection{\texorpdfstring{❍ \textbf{Coherence
Bandwidth:}}{❍ Coherence Bandwidth:}}\label{coherence-bandwidth}

Change in frequency \$ f \$ such that the channel (i.e.~magnitude of
received signal) remains relatively the same.

\begin{align}
\Delta \phi &= \frac{4 \pi f }{c} (d - r) + \pi \qquad \to \text{change} f \text{ so that phase shift changes} \approx \frac{\pi}{2} \\
&= \frac{4 \pi f}{c} ( d - r ) \qquad \text{change by } \approx \frac{\pi}{2} \\
&=2 \pi f \left( \frac{2d - r \; - r}{c} \right) \approx \frac{\pi}{2} \impliedby \text{rewrite to help us later.}
\end{align}

\textbf{Note}: \(\frac{2d - r}{c} - \frac{r}{c} \Rightarrow T_d\) is
called the \textbf{delay spread}, (\textbf{difference in delay} of the
paths).

\(\qquad \Rightarrow 2 \pi f T_d \approx \frac{\pi}{2}  \qquad \qquad  \implies f \approx \frac{1}{4 T_d} \Rightarrow W_c = \frac{1}{4 T_d}\)

\textbf{(note:} if set
\(2\pi f \frac{2d-r-r}{c} \approx {\color{green}\pi } \text{[substantial phase shift]} \Rightarrow W_c = \frac{1}{2T_d}\)\\
- the constant factor does not really matter).
\(( \frac{\pi}{2} \to \frac{1}{4T_d} \text{ or } \pi \to \frac{1}{2T_d}).\)

\textbf{In fact,
\href{https://en.wikipedia.org/wiki/Coherence_bandwidth}{wiki}
\(W_c \approx \frac{2 \pi {T_d} \quad B_c \approx \frac{1}{T_d}\).}

    \subsubsection{❍ Coherence Distance: same
scenario}\label{coherence-distance-same-scenario}

\$ \Delta \phi \approx 2\pi f \left( \frac{2d - r}{c} -
\frac{r}{c}\right) = \frac{2\pi f}{c} (2d - 2r) = \frac{4\pi f}{c} (d -
r) \$

Change in \(r\) prop to change in \$ d - r \$

What change in \$ r \$ will result in phase shift of \$ \frac{\pi}{2}
\$.

\$ \frac{4\pi f}{c} \cdot x \approx \frac{\pi}{2} \Rightarrow x
\approx \frac{\frac{\pi}{2}}{4\pi \frac{f}{c}} = \frac{c}{f . 8}
\approx \frac{\lambda}{8}. \$

\$ \Rightarrow \text{coherence distance } \Delta x\_c
\approx \frac{\lambda}{8}. \$

(If I set ``substantial phase shift'' \$
\Delta \phi \approx \frac{4\pi f}{c} \overbrace{(d - r)}\^{}\{x\}
\approx \pi \$)

\$ \Rightarrow x = \frac{c}{f} \cdot \frac{1}{4} = \frac{\lambda}{4} =
\Delta x\_c \$

\$ \Delta x\_c \approx \frac{\lambda}{8} \to \frac{\lambda}{4} ;
(\text{depending on convention}). \$

    \subsubsection{❍ Reflecting wall, moving
Antenna}\label{reflecting-wall-moving-antenna}

\begin{itemize}
\tightlist
\item
  \textbf{Channel strength changes as you move through constructive \&
  destructive interference (``multipath fading'').}
\end{itemize}

\$ \begin{align}
    E_r(f, t) &= \frac{\alpha}{r_1(t)} \cdot \cos 2 \pi f \left[ t - \overbrace{\frac{r(t)}{c}}^{\text distance 1} \right] + \frac{\alpha}{r_2(t)} \cdot \cos \left[ 2 \pi f \left[ t - \frac{2d_1 - r(t)}{c} \right] + \pi \right] \\
    &= \frac{\alpha}{r_0 + v \cdot t} \cdot \cos  2 \pi f \left[ t - \frac{v \cdot t}{c} - \underbrace{\frac{r_0}{c}}_{\phi 1} \right] + \frac{\alpha}{2d - r_0 - v \cdot t} \cdot \cos \left[ 2 \pi f \left[ t - \frac{2d - r_0 - v \cdot t}{c} \right] + \overbrace{\pi}^{\text{because reflection}} \right] \\
    &= \frac{\alpha}{r_0 + v \cdot t} \cdot \cos 2 \pi f \left[ t \left( 1 - \frac{v}{c} \right) - \underbrace{\frac{r_0}{c}}_{\phi 1} \right] \overbrace{-}^{(\pi \text{ taken here})} \frac{\alpha}{2d - r_0 - v \cdot t} \cdot \cos 2 \pi f \left[ t \left( 1 + \frac{v}{c} \right) - \underbrace{\frac{2d - r_0}{c}}_{\phi 2} \right]
\end{align} \$

    \textbf{Now assume (for simplicity) that we are close to the wall}\\
\$ \Rightarrow \qquad \qquad \qquad \qquad \Rightarrow r(t) \approx d
\quad \Rightarrow \quad r(t) \approx 2d - r(t) \approx d \$

\$ E\_r(f, t) \approx \frac{\alpha}{r(t)}
\Big[ \cos \underbrace{2 \pi f \left[ t \left(1 - \frac{v}{c} \right) - \phi_1 \right]}_{A} - \cos \underbrace{ 2 \pi f \left[ t \left(1 + \frac{v}{c} \right) - \phi_2 \right]}_{B} \Big]\$

\textbf{Recall} \(\qquad
\cos A - \cos B = -2 \sin \left( \frac{A + B}{2} \right) \sin \left( \frac{A - B}{2} \right)\)

\$ \frac{A + B}{2} = 2 \pi f
\left[ \frac{1}{2} \left[ t \left( 1 - \frac{v}{c} \right) - \frac{r_0}{c} + t \left( 1 + \frac{v}{c} \right) - \frac{2d - r_0}{c} \right] \right{]}
= 2 \pi f \left[ t - \frac{d}{c} \right]\$

\$ \frac{A - B}{2} = 2 \pi f
\left[ \frac{1}{2} \left[ \cancel{t} - \frac{t \cdot v}{c} - \frac{r_0}{c} - \cancel{t} - \frac{t \cdot v}{c} + \frac{2d - r_0}{c} \right] \right{]}
= 2 \pi f \left[ - \frac{t \cdot v}{c} + \frac{d - r_0}{c} \right]\$

\$ E\_r(f, t) = \frac{2 \alpha}{d}
\cdot \underbrace{ \sin \left[ 2 \pi f \left(\frac{t \cdot v}{c} - \frac{d - r_0}{c} \right) \right] }\emph{\{\text{very slow }
( \text{freq } f \frac{v}{c})\}
\cdot \underbrace{ \sin 2 \pi f \left[ t - \frac{d}{c} \right]}}\{\text{very fast (freq }f)\}
\$

    \$ E\_r(f, t) = \frac{2 \alpha}{d}
\cdot \sin \left[ 2 \pi f \left(\frac{t \cdot v}{c} - \frac{d - r_0}{c} \right) \right] \cdot \sin 2
\pi f \left[ t - \frac{d}{c} \right]\$

\$ f' = f \cdot \frac{v}{c} = \frac{1}{T'} \qquad \qquad T' =
\frac{1}{f'} \$

\textbf{- period of slow sinusoid,}\\
\textbf{- period of major ups \& downs}

\$ \Rightarrow \text{Coherence period: } \frac{T'}{4} = \frac{1}{4f'} =
\frac{1}{4} \cdot \frac{c}{f \cdot v} = \frac{1}{4}
\lambda \cdot \frac{1}{v}. \$ \textbf{(recall \$ c = f
\cdot \lambda \$).}

Another way to see: \textbf{coherence distance} \$
\approx \frac{\lambda}{4} = \Delta x\_c \$ (Can call it distance because
next to it we have \$ v \$.)

\textbf{Generally note:} \$ \frac{\lambda}{4} \ll r \$. (In process, it
will change many times.)

    \textbf{Also write \$ \Delta x\_c \$ or \$ T\_c \$ in terms of Doppler
Spread.}

\textbf{Doppler Spread here:} \$ D\_s \triangleq D\_\{\text{max}\} -
D\_\{\text{min}\} = f \frac{v}{c} - (-f \frac{v}{c}) = 2f \frac{v}{c} =
D\_s. \$

\$ T\_c = \frac{T'}{4} = \frac{1}{4} \frac{c}{f v} = \frac{1}{2}
\frac{c}{2 f v} = \boxed{ \frac{1}{2 \cdot D_s} \approx T_c }. \$

\$ v = \frac{ \Delta x_c }{ T_c } \qquad \Delta x\_c = T\_c \cdot v =
\frac{v}{2 D_s}. \$

\begin{center}\rule{0.5\linewidth}{0.5pt}\end{center}

    \textbf{Example}: \$ v = 60 , \text{km/h}, , f = 1 , \text{GHz} ,
(\text{typical freq}) \$.

\$ v = \frac{60 \cdot 10^3 \text{ m}}{3600 \text{ s}} = \frac{50}{3} ,
ms\^{}\{-1\} \Rightarrow D\_s \approx 2 \cdot f \cdot \frac{v}{c} =
\frac{2 \cdot 10^9 \cdot 50}{3 \cdot 3 \cdot 10^8}
\approx \frac{2.5 \cdot 100}{9} \approx 110 , \text{Hz}. \$

\$ \Rightarrow T\_c \approx \frac{1}{2 D_s} = \frac{1}{2 \cdot 110} ,
\text{s} \approx 5 , \text{ms}; , \Delta x\_c \approx v \cdot T\_c =
\frac{50}{3} \cdot 5 \cdot 10\^{}\{-3\} \approx 75 , \text{cm}
\approx \text{few cm}. \$

    \subsection{❍ Input - Output Model}\label{input---output-model}

    Consider many paths (many reflectors)

~ ~ Difference paths \(i\) (many \ldots)

\textbf{Input}\\
\$ x(t) = \cos 2\pi f t \$

\textbf{Recall our attenuation examples before:}

\$ \boxed{
y(t) = \sum_i a_i(f, t) \cdot x\big(t - \tau_i(f, t)\big)
} \$

\textbf{Propagation delays} \$ \tau\_i(f, t) \$: How long it takes for
each path to travel from \$ t \to r \$.

    \textbf{Path attenuations \$ a\_i(f, t) \$:}

\begin{itemize}
\tightlist
\item
  Note that \$ \tau\_i(f, t) \$ and \$ a\_i(f, t) \$ are functions of
  both \$ f \$ and \$ t \$.\\
\item
  But recall :
\end{itemize}

\textbf{Bandwidth} is small compared to \$ f\_c = 1, \text{GHz} \$
(central frequency).\\
So all frequencies that we use are (in relative terms) very close to \$
f\_c = 1 , \text{GHz} \$.\\
(Bandwidth \$ \text{BW} \$ typically \$ 1 , \text{MHz} \$ or so.)

\$ \Rightarrow f
\in [0.999 \, \text{GHz} \rightarrow 1.001 \, \text{GHz} ] \approx 1 ,
\text{GHz} \$

\$ \Rightarrow\$ \textbf{We can nicely assume:} \$ \tau\_i \$ and \$
a\_i \$ are independent of \$\color{orange}f \$.

\textbf{Be careful}: Channel response is still a function of
\$\color{orange}f \$. (Just like in previous examples.)

\$ \Rightarrow\$ \$ \tau\_i(t), , a\_i(t) \$

\$ \Rightarrow\$ \$ y(t) = \sum\_i a\_i(t) \cdot x(t - \tau\_i(t)) \$

    \textbf{Example:} Perfectly reflecting wall with movement.

\textbf{Recall:}

\$ \begin{align}
E_r(f, t) &= \frac{\alpha}{r_1(t)} \cos 2 \pi f \left(t - \tau_1(t)\right) + \frac{\alpha}{r_2(t)} \cos 2 \pi f \left(t - \tau_2(t) - \pi\right) \\
&= {\color{yellow}\frac{\alpha}{r_1(t)}} \cos 2 \pi f \left(t - {\color{orange}\frac{v t + r_0}{c}}\right) + {\color{yellow}\frac{\alpha}{r_2(t)}} \cos 2 \pi f \left(t - {\color{orange}\frac{2d - r_0 - v t}{c}}\right).
\end{align} \$

Where: - \$ a\_1(t) = \{\color{yellow}\frac{\alpha}{r_1(t)}\} ,
\tau\_1(t) = \{\color{orange}\frac{v t + r_0}{c}\} \$, - \$ a\_2(t) =
\{\color{yellow}\frac{\alpha}{r_2(t)}\} , \tau\_2(t) =
\{\color{orange}\frac{2d - r_0 - v t}{c}\} \$.

\textbf{Recall:} We have accepted linearity.

\$ y(t) = \int\_\{-\infty\}\^{}\{\infty\} h(t, \tau) \cdot x(t - \tau) ,
d\tau \$

\$ \implies \$ I/O relationship via convolution:\\
\$\qquad \$ \$ y(t) = h(t, \tau) \ast x(t) \$: For \$ h(t, \tau) \$, a
T.V. (time-variant) channel impulse response.

\$ y(t) = \int\limits\_\{-\infty\}\^{}\{\infty\} h(t, \tau) \cdot x(t -
\tau) , d\tau  \$

\(\qquad \qquad \downarrow \quad\)\textbf{But also (from before):}

\$ y(t) = \sum\_i a\_i(t) \cdot x(t - \tau\_i(t)) \$

\textbf{Put two together:}\\
\$ \implies \$ \(\boxed{
h(t, \tau) = \sum\limits_{i}  a_i(t) \cdot \delta(\tau - \tau_i(t))
}\)

\textbf{Verify:}

\$ \begin{align}
y(t) &= \int\limits_{-\infty}^{\infty} h(t, \tau) \cdot x(t - \tau) \, d\tau = \int\limits_{-\infty}^{\infty} \sum_i a_i(t) \cdot \delta(\tau - \tau_i(t)) \cdot x(t - \tau) \, d\tau \\
&= \sum\limits_{i} a_i(t) \underbrace{ \int\limits_{-\infty}^{\infty} \delta(\tau - \tau_i(t)) \cdot x(t - \tau) \, d\tau }_{x(t - \tau_i(t))}= \sum\limits_{i} a_i(t) \cdot x(t - \tau_i(t))
\end{align} \$

\$ \implies \$ \textbf{LTV impulse response of the channel:} \(\boxed{
y(t) = h(t, \tau) \ast x(t)
}\)

    \$ x(t) \xrightarrow{\qquad } \underset{\hat{
\begin{matrix}
  | \\
  \text{channel}
\end{matrix}}
}{\boxed{\qquad h(t, \tau) \qquad}} \xrightarrow{\qquad } y(t) \$ (we
have seen)

Our task now will be to extend the concept of this \textbf{``channel''}
to include important processes of communication.

Thus, let us sketch this process.

    \subsubsection{\texorpdfstring{❍ \textbf{Communication process in a
glimpse: from \$ m(t) \$ to \$ x(t)
\$:}}{❍ Communication process in a glimpse: from \$ m(t) \$ to \$ x(t) \$:}}\label{communication-process-in-a-glimpse-from-mt-to-xt}

\begin{enumerate}
\def\labelenumi{\arabic{enumi}.}
\tightlist
\item
  \textbf{Message:} \texttt{"Meet\ me\ at\ 4pm"} → \$ m\_1 \$
\end{enumerate}

\$ \begin{align*}
m_1 &= \, \underbrace{11}, \quad \underbrace{01}, \quad \underbrace{11}, \quad \underbrace{01}, \quad \underbrace{01} \qquad \qquad = b_1 \\
b_1 &= (1 - i, \, -1 + i, \, 1 - i, \, -1 + i, \, 1 + i ) \qquad \qquad = [x[1]  x[2]  \ldots  x[5]] \qquad  \qquad  = \underline{x} \\
\mathcal{Re}\{x\} &= [1, -1, 1, -1, 1] \qquad \qquad \qquad \qquad \qquad \mathcal{Im}\{x\} = [-1, 1, -1, 1]
\end{align*} \$

\begin{itemize}
\tightlist
\item
  \textbf{this requires infinite bandwidth.} \(\qquad \qquad\)
  \(\to \infty \, \text{BW}\) (because \(F T\) of \(exp\) is sync).
\item
  \$ \mathcal{Re}\{x{[}1{]}\} = 1 \$, \$ \mathcal{Re}\{x{[}2{]}\} = -1
  \$
\end{itemize}

\begin{enumerate}
\def\labelenumi{\arabic{enumi}.}
\setcounter{enumi}{4}
\tightlist
\item
  \textbf{Solution:}

  \begin{itemize}
  \tightlist
  \item
    \textbf{Need to smooth signals.}
  \item
    \textbf{Modulate} with a sequence of smooth sync functions:

    \begin{itemize}
    \tightlist
    \item
      \(\underline{x} \, \ast \, \left\{ \text{sinc}\{w_t  -n\}\right\}_n\)
      to give \$ x\_b(t) \$ that has finite bandwidth.
    \item
      \(P.A\) Convolution (Pulse Amplitude)
    \item
    \end{itemize}
  \end{itemize}
\end{enumerate}

\begin{itemize}
\tightlist
\item
  High Frequency Modulation\\
  Low frequency signals suffer from rapid attenuation\\
  (easily absorbed by walls, etc.).
\end{itemize}

\$ x\_b(t) \xrightarrow{\qquad \qquad } \underset{\hat{
\begin{matrix}
  \approx & | \\
  \cos & 
\end{matrix}2\pi f_c t}
}{\bigodot} \xrightarrow{\qquad } \boxed{\qquad x(t) \qquad } \$
\(\quad f_c \approx 1 \to 2 \text{GHz}\)

\begin{itemize}
\tightlist
\item
  (Diagram of modulation process with \$ x\_b(t) \$ and \$ x\_c(t) \$)\\
  → High Frequency spectrum \$ f\_c \pm w \$.
\end{itemize}

→ Then demodulate (scaled by \$ \approx \cos 2 \pi f\_c t \$).\\
→ Then sample.

(Sample points above in the waveform):
\(1.3 + 2.5i, -1.9 + 0.7i \cdots\)

    \subsubsection{❍ Establishing relationship between Baseband input \&
output}\label{establishing-relationship-between-baseband-input-output}

\[
x_b(t) \xrightarrow{\qquad } 
\underset{\hat{
\begin{matrix}
  & | \\
  \cos \\
  \sin
\end{matrix} 2\pi f_c t}
}{\bigotimes}
\xrightarrow{\quad x(t) \quad }  \boxed{\qquad h(t, \tau) \qquad } \xrightarrow{ \quad y(t) \quad } 
\underset{\hat{
\begin{matrix}
  & | \\
  \cos \\
  \sin
\end{matrix} 2\pi f_c t}
}{\bigotimes}
\xrightarrow{\qquad } y(t)
\]

\$ x\_b(t) \xrightarrow{\qquad \text{F.T} \qquad \qquad} X\_b(f)
\xrightarrow{\qquad \qquad \qquad \qquad \qquad \qquad \qquad \qquad}
X(f) \$

\textbf{In the time domain:}

\$ \qquad \qquad x(t) = \Re \left\{ x\_b(t) \cdot e\^{}\{j2\pi f\_c t\}
\right\} \$

Similarly: \$ ; y(t) = \Re \left\{ y\_b(t) \cdot e\^{}\{j2\pi f\_c t\}
\right\} \$

\textbf{From reverse action of demodulation:}

\$ Y(f) \quad \text{(plots on the left: symmetric about } -f\_c
\text{ and } f\_c\text{)} \quad \to \quad Y\_b(f)
\quad \text{(filtered centered at } f\_c\text{)}. \$

    \textbf{In practice: how is this achieved?}

\$ \begin{align}
x(t) &= \Re \{ \underbrace{\bar{x}_b(t)}_{\text{complex}} \cdot e^{j 2 \pi f_c t} \} \\
\Rightarrow \text{write} \quad \bar{x}_b(t) &= x_I(t) + j x_Q(t) \quad \& \quad \text{recall} \quad e^{j 2 \pi f_c t} = \cos 2 \pi f_c t + j \sin 2 \pi f_c t \\
\Rightarrow \qquad \quad \; \; x(t) &= \Re \{ x_I(t) (\cos 2 \pi f_c t + j \sin 2 \pi f_c t) + j x_Q(t) (\cos 2 \pi f_c t + j \sin 2 \pi f_c t) \} \\
\Rightarrow \qquad \qquad \qquad \! & \boxed{x(t) = x_I(t) \cos 2 \pi f_c t - x_Q(t) \sin 2 \pi f_c t} \qquad \text{(xxx)}
\end{align} \$

\textbf{Recap}

\begin{longtable}[]{@{}ll@{}}
\toprule\noalign{}
From the action of modulation & \\
\midrule\noalign{}
\endhead
\bottomrule\noalign{}
\endlastfoot
& \\
\end{longtable}

\$ \boxed{
\begin{align}
\Rightarrow \text{For our signals. } & \quad \bar{x}_b(t) \qquad \& \qquad x(t) \\
\Rightarrow \qquad \qquad \quad x(t) &= \Re \{ \bar{x}_b(t) \cdot e^{j 2 \pi f_c t} \} \\
\Rightarrow \qquad \qquad \quad x(t) &= x_I(t) \cos 2 \pi f_c t - x_Q(t) \sin 2 \pi f_c t
\end{align}} \$ \textbar{} \textbar{}

\begin{longtable}[]{@{}ll@{}}
\toprule\noalign{}
From the reverse action of demodulation & \\
\midrule\noalign{}
\endhead
\bottomrule\noalign{}
\endlastfoot
& \\
\end{longtable}

\$ \boxed{
\Rightarrow \text{Similarly. } \quad \; y(t) = \Re \left\{ y_b(t) \cdot e^{j2\pi f_c t} \right\}
} \$ \textbar{} \textbar{}

    Now ready to establish relationship

\$ x\_b(t) \xrightarrow{\qquad } \{\boxed{\qquad h_b(t, \tau) \qquad}\}
\xrightarrow{\qquad } y\_b(t) \$

\subsubsection{Recall:}\label{recall}

\$ \begin{align}
y(t) &= \sum\limits_i a_i(t) \cdot x(t - \tau_i(t)) \qquad \qquad a_i(t) \qquad \text{path attenuation} \\
& \qquad \qquad \qquad \qquad  \qquad \qquad \qquad \quad \tau_i(t) \qquad \text{"} \qquad  \text{delay} \\
&= \Re \{ y_b(t) \cdot e^{j2\pi f_c t} \}   \qquad \quad \qquad \quad x(t) = \Re \{ x_b(t) \cdot e^{j2\pi f_c t} \}
\end{align} \$

\$ \Rightarrow \Re \{ y\_b(t) \cdot \not{e^{j2\pi f_c t}} \} =
\sum\limits\_i a\_i(t) \cdot \Re \Big\{ x\_b(t - \tau\_i(t))
\cdot e\^{}\{j2\pi f\_c(t - \tau\_i(t))\} \Big\} = \Re \left\{
\sum\limits\emph{i \underbrace{a_i(t)}}\{\text{real}\} \cdot x\_b(t -
\tau\_i(t)) \cdot \not{e}\^{}\{j2\pi f\_c t\} \cdot e\^{}\{-j2\pi f\_c
\tau\_i(t)\} \right\} \$

\$ \Rightarrow \Re \{ y\_b(t) \} = \Re \left\{ \sum\limits\_i a\_i(t)
\cdot x\_b(t - \tau\_i(t)) \cdot e\^{}\{-j2\pi f\_c \tau\_i(t)\}
\right\} \quad \text{(easy exercise to show)} \$

Similarly,

\$ \Im \{ y\_b(t) \cdot \not{e^{j2\pi f_c t}} \} = \Im \left\{
\sum\limits\_i a\_i(t) \cdot x\_b(t - \tau\_i(t))
\cdot \not{e^{j2\pi f_c t}} \cdot e\^{}\{-j2\pi f\_c \tau\_i(t)\}
\right\} \$

\$
\boxed{ y_b(t) = \sum_i a_i(t) \cdot e^{-j2\pi f_c \tau_i(t)} \cdot x_b(t - \tau_i(t)) }
\qquad y\_b(t) = \sum\_i a\_i\^{}b(t) \cdot x\_b(t - \tau\_i(t)) ;
\boxed{  \text{where} \; a_i^b(t) = a_i(t) \cdot e^{-j2\pi f_c \tau_i(t)} }
\$

    \textbf{Recall:}

\$ \text{Linearity still holds since}
\qquad \qquad \qquad \qquad \text{all we did is multiply linear functions}
\$

\$ \Rightarrow y\_b(t) = \int\limits\_\{-\infty\}\^{}\{\infty\}
h\_b(\tau, t) ; x\_b(t - \tau) d\tau ;
\implies \boxed{h_b(\tau, t) = \sum\limits_i \underbrace{a_i^b(t)}_{\color{orange}(Note_1)} \; \delta(\tau - \tau_i(t))}
\implies\$ \textbf{LTV complex continuous-time baseband I/O channel
representation}

\begin{itemize}
\tightlist
\item
  \(\color{orange}(Note_1):\) \$ a\_i\^{}b(t) =
  \underbrace{a_i(t)}\emph{\{\color{green}\text{slow}\} \cdot e\^{}\{-j
  2 \pi f\_c
  \cdot \underbrace{\tau_i (t)}}\{\color{cyan}\text{fast}\}\}\$

  \begin{itemize}
  \tightlist
  \item
    \(\color{green}\text{slow}\): in minutes
  \item
    \(\color{cyan}\text{fast}\): rate of change defines rate of change
    of \(a_i(t)\) and thus of \(h(\tau, t)\)
  \end{itemize}
\end{itemize}

\subparagraph{\texorpdfstring{- Say not so big change in \$ a\_i\^{}b(t)
\$ (coherence in \$ a\_i\^{}b(t) \$) corresponding to \textbf{phase
shift} of \$ \pi/2 \(:
        - (Note: here there’s a model of having one path.)\)}{- Say not so big change in \$ a\_i\^{}b(t) \$ (coherence in \$ a\_i\^{}b(t) \$) corresponding to phase shift of \$ /2 :
        - (Note: here there's a model of having one path.)}}\label{say-not-so-big-change-in-a_ibt-coherence-in-a_ibt-corresponding-to-phase-shift-of-2---note-here-theres-a-model-of-having-one-path.}

2\pi f\_c \tau\_i(t) \approx \frac{\pi}{2} \implies \tau\_i
\approx \frac{1}{4f_c} \$ \(\quad \frac{\pi}{2}\) change in phase when
delay of path changes by up to \$ \approx \frac{1}{4f_c} \$.

At speed of light: \$ \Delta x\_c \approx \frac{\lambda}{4} \Rightarrow
\$ (Recall \$ c = \lambda \cdot f \$). ( make note that
\(f \approx f_c\) )

\$ \Rightarrow T\_c = \frac{\Delta X_c}{v} = \frac{\lambda}{4v}
\approx \frac{c}{4 \cdot f_c \cdot v} \approx \frac{1}{4D_s} \$ (Recall
Doppler spread \$ D\_s \approx \frac{f_c V}{c} \$).

    \$ \Rightarrow h\_b(t, \tau) \text{ changes } \Delta x\_c
\approx \frac{\lambda}{4} \qquad \left( \frac{\lambda}{4}
\approx \frac{1}{4} \qquad \frac{c}{f} \approx \frac{1}{4}
\frac{3 \cdot 10^8 m \not{s^{-1}}}{10^9 \not{s^{-1}}} \right) \approx 7
\text{cm}. \$ (few cm)

\& reasonable at reasonable speeds (say \$ \approx 60 , \text{km/s} \$)

\$ \frac{c}{4 \cdot f_c \cdot v}
\approx \frac{3 \cdot \not{10^8}}{4 \cdot \not{10^9} \cdot \frac{60 \cdot 10^3}{3600}}
= \frac{\overset{6}{3.\not{36}\not{00}}}{4\not{0}.\not{60}.10^3} =
\frac{9}{1000} \approx 4 \rightarrow 5 , \text{ms} ; (\text{few} ,
\text{ms})\$

\begin{itemize}
\item
  \textbf{Interesting observation on frequency response of \$ h\_b(t,
  \tau) \$:}\\
  \$ T\_c
  \gg \underbrace{\text{memory of channel} \approx T_d}\_\{\text{(a bit tricky)}\}
  \$

  \textbf{By definition of impulse response}\\
  \$ \delta(t) \to \boxed{ h_b(t, \tau)} \to y\_b(t) = h\_b(t, \tau) \$.
\item
  \textbf{Send an impulse through channel.}
\item
  \textbf{All signals will be collected (from all paths) in the order of
  what we will call ``\,``Delay Spread''\,``}:\\
  \$ T\_d \overset{\Delta}{=} \max\limits\_\{i,j\} \textbar{}\tau\_i -
  \tau\_j\textbar{} \qquad \tau\_i =
  \frac{\overbrace{d_i}^{\text{distance}}}{c}, ; \tau\_j =
  \frac{\overbrace{d_j}^{\text{distance}}}{c}. \$
\end{itemize}

    \begin{itemize}
\tightlist
\item
  In cellular nets (mainly urban), \textbf{typical} distance
  \textbf{differences} \$ d\_i - d\_j \approx \text{few hundred meters}
  \$
\end{itemize}

\$ \Rightarrow \tau\_i - \tau\_j = \frac{d_i - d_j}{c}
\approx \frac{\text{few hundred m}}{3 \cdot 10^8 \, m s^{-1}}
\approx \text{few} , \mu s , , (\approx 1 \to 10 \cdot 10\^{}\{-6\} ,
\text{s}). \$

\(\Rightarrow\) ``Memory of channel'' \$\qquad h\_b(\tau, t)
\approx \text{few} , \mu s = T\_d \$

\$ \qquad \text{But } T\_d \ll T\_c \approx \text{few ms.} \$

\$ \Rightarrow \$ can consider channel impulse response to be almost
\textbf{time-invariant} \emph{(But clarify what is meant by this.)}

\begin{verbatim}
(since it is only after many impulses that the channel will change).  
\end{verbatim}

    \begin{tcolorbox}[breakable, size=fbox, boxrule=1pt, pad at break*=1mm,colback=cellbackground, colframe=cellborder]
\prompt{In}{incolor}{ }{\boxspacing}
\begin{Verbatim}[commandchars=\\\{\}]

\end{Verbatim}
\end{tcolorbox}


    % Add a bibliography block to the postdoc
    
    
    
\end{document}
