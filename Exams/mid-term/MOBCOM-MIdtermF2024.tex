\documentclass[11pt]{article}

    \usepackage[breakable]{tcolorbox}
    \usepackage{parskip} % Stop auto-indenting (to mimic markdown behaviour)
    

    % Basic figure setup, for now with no caption control since it's done
    % automatically by Pandoc (which extracts ![](path) syntax from Markdown).
    \usepackage{graphicx}
    % Keep aspect ratio if custom image width or height is specified
    \setkeys{Gin}{keepaspectratio}
    % Maintain compatibility with old templates. Remove in nbconvert 6.0
    \let\Oldincludegraphics\includegraphics
    % Ensure that by default, figures have no caption (until we provide a
    % proper Figure object with a Caption API and a way to capture that
    % in the conversion process - todo).
    \usepackage{caption}
    \DeclareCaptionFormat{nocaption}{}
    \captionsetup{format=nocaption,aboveskip=0pt,belowskip=0pt}

    \usepackage{float}
    \floatplacement{figure}{H} % forces figures to be placed at the correct location
    \usepackage{xcolor} % Allow colors to be defined
    \usepackage{enumerate} % Needed for markdown enumerations to work
    \usepackage{geometry} % Used to adjust the document margins
    \usepackage{amsmath} % Equations
    \usepackage{amssymb} % Equations
    \usepackage{textcomp} % defines textquotesingle
    % Hack from http://tex.stackexchange.com/a/47451/13684:
    \AtBeginDocument{%
        \def\PYZsq{\textquotesingle}% Upright quotes in Pygmentized code
    }
    \usepackage{upquote} % Upright quotes for verbatim code
    \usepackage{eurosym} % defines \euro

    \usepackage{iftex}
    \ifPDFTeX
        \usepackage[T1]{fontenc}
        \IfFileExists{alphabeta.sty}{
              \usepackage{alphabeta}
          }{
              \usepackage[mathletters]{ucs}
              \usepackage[utf8x]{inputenc}
          }
    \else
        \usepackage{fontspec}
        \usepackage{unicode-math}
    \fi

    \usepackage{fancyvrb} % verbatim replacement that allows latex
    \usepackage{grffile} % extends the file name processing of package graphics
                         % to support a larger range
    \makeatletter % fix for old versions of grffile with XeLaTeX
    \@ifpackagelater{grffile}{2019/11/01}
    {
      % Do nothing on new versions
    }
    {
      \def\Gread@@xetex#1{%
        \IfFileExists{"\Gin@base".bb}%
        {\Gread@eps{\Gin@base.bb}}%
        {\Gread@@xetex@aux#1}%
      }
    }
    \makeatother
    \usepackage[Export]{adjustbox} % Used to constrain images to a maximum size
    \adjustboxset{max size={0.9\linewidth}{0.9\paperheight}}

    % The hyperref package gives us a pdf with properly built
    % internal navigation ('pdf bookmarks' for the table of contents,
    % internal cross-reference links, web links for URLs, etc.)
    \usepackage{hyperref}
    % The default LaTeX title has an obnoxious amount of whitespace. By default,
    % titling removes some of it. It also provides customization options.
    \usepackage{titling}
    \usepackage{longtable} % longtable support required by pandoc >1.10
    \usepackage{booktabs}  % table support for pandoc > 1.12.2
    \usepackage{array}     % table support for pandoc >= 2.11.3
    \usepackage{calc}      % table minipage width calculation for pandoc >= 2.11.1
    \usepackage[inline]{enumitem} % IRkernel/repr support (it uses the enumerate* environment)
    \usepackage[normalem]{ulem} % ulem is needed to support strikethroughs (\sout)
                                % normalem makes italics be italics, not underlines
    \usepackage{soul}      % strikethrough (\st) support for pandoc >= 3.0.0
    \usepackage{mathrsfs}
    

    
    % Colors for the hyperref package
    \definecolor{urlcolor}{rgb}{0,.145,.698}
    \definecolor{linkcolor}{rgb}{.71,0.21,0.01}
    \definecolor{citecolor}{rgb}{.12,.54,.11}

    % ANSI colors
    \definecolor{ansi-black}{HTML}{3E424D}
    \definecolor{ansi-black-intense}{HTML}{282C36}
    \definecolor{ansi-red}{HTML}{E75C58}
    \definecolor{ansi-red-intense}{HTML}{B22B31}
    \definecolor{ansi-green}{HTML}{00A250}
    \definecolor{ansi-green-intense}{HTML}{007427}
    \definecolor{ansi-yellow}{HTML}{DDB62B}
    \definecolor{ansi-yellow-intense}{HTML}{B27D12}
    \definecolor{ansi-blue}{HTML}{208FFB}
    \definecolor{ansi-blue-intense}{HTML}{0065CA}
    \definecolor{ansi-magenta}{HTML}{D160C4}
    \definecolor{ansi-magenta-intense}{HTML}{A03196}
    \definecolor{ansi-cyan}{HTML}{60C6C8}
    \definecolor{ansi-cyan-intense}{HTML}{258F8F}
    \definecolor{ansi-white}{HTML}{C5C1B4}
    \definecolor{ansi-white-intense}{HTML}{A1A6B2}
    \definecolor{ansi-default-inverse-fg}{HTML}{FFFFFF}
    \definecolor{ansi-default-inverse-bg}{HTML}{000000}

    % common color for the border for error outputs.
    \definecolor{outerrorbackground}{HTML}{FFDFDF}

    % commands and environments needed by pandoc snippets
    % extracted from the output of `pandoc -s`
    \providecommand{\tightlist}{%
      \setlength{\itemsep}{0pt}\setlength{\parskip}{0pt}}
    \DefineVerbatimEnvironment{Highlighting}{Verbatim}{commandchars=\\\{\}}
    % Add ',fontsize=\small' for more characters per line
    \newenvironment{Shaded}{}{}
    \newcommand{\KeywordTok}[1]{\textcolor[rgb]{0.00,0.44,0.13}{\textbf{{#1}}}}
    \newcommand{\DataTypeTok}[1]{\textcolor[rgb]{0.56,0.13,0.00}{{#1}}}
    \newcommand{\DecValTok}[1]{\textcolor[rgb]{0.25,0.63,0.44}{{#1}}}
    \newcommand{\BaseNTok}[1]{\textcolor[rgb]{0.25,0.63,0.44}{{#1}}}
    \newcommand{\FloatTok}[1]{\textcolor[rgb]{0.25,0.63,0.44}{{#1}}}
    \newcommand{\CharTok}[1]{\textcolor[rgb]{0.25,0.44,0.63}{{#1}}}
    \newcommand{\StringTok}[1]{\textcolor[rgb]{0.25,0.44,0.63}{{#1}}}
    \newcommand{\CommentTok}[1]{\textcolor[rgb]{0.38,0.63,0.69}{\textit{{#1}}}}
    \newcommand{\OtherTok}[1]{\textcolor[rgb]{0.00,0.44,0.13}{{#1}}}
    \newcommand{\AlertTok}[1]{\textcolor[rgb]{1.00,0.00,0.00}{\textbf{{#1}}}}
    \newcommand{\FunctionTok}[1]{\textcolor[rgb]{0.02,0.16,0.49}{{#1}}}
    \newcommand{\RegionMarkerTok}[1]{{#1}}
    \newcommand{\ErrorTok}[1]{\textcolor[rgb]{1.00,0.00,0.00}{\textbf{{#1}}}}
    \newcommand{\NormalTok}[1]{{#1}}

    % Additional commands for more recent versions of Pandoc
    \newcommand{\ConstantTok}[1]{\textcolor[rgb]{0.53,0.00,0.00}{{#1}}}
    \newcommand{\SpecialCharTok}[1]{\textcolor[rgb]{0.25,0.44,0.63}{{#1}}}
    \newcommand{\VerbatimStringTok}[1]{\textcolor[rgb]{0.25,0.44,0.63}{{#1}}}
    \newcommand{\SpecialStringTok}[1]{\textcolor[rgb]{0.73,0.40,0.53}{{#1}}}
    \newcommand{\ImportTok}[1]{{#1}}
    \newcommand{\DocumentationTok}[1]{\textcolor[rgb]{0.73,0.13,0.13}{\textit{{#1}}}}
    \newcommand{\AnnotationTok}[1]{\textcolor[rgb]{0.38,0.63,0.69}{\textbf{\textit{{#1}}}}}
    \newcommand{\CommentVarTok}[1]{\textcolor[rgb]{0.38,0.63,0.69}{\textbf{\textit{{#1}}}}}
    \newcommand{\VariableTok}[1]{\textcolor[rgb]{0.10,0.09,0.49}{{#1}}}
    \newcommand{\ControlFlowTok}[1]{\textcolor[rgb]{0.00,0.44,0.13}{\textbf{{#1}}}}
    \newcommand{\OperatorTok}[1]{\textcolor[rgb]{0.40,0.40,0.40}{{#1}}}
    \newcommand{\BuiltInTok}[1]{{#1}}
    \newcommand{\ExtensionTok}[1]{{#1}}
    \newcommand{\PreprocessorTok}[1]{\textcolor[rgb]{0.74,0.48,0.00}{{#1}}}
    \newcommand{\AttributeTok}[1]{\textcolor[rgb]{0.49,0.56,0.16}{{#1}}}
    \newcommand{\InformationTok}[1]{\textcolor[rgb]{0.38,0.63,0.69}{\textbf{\textit{{#1}}}}}
    \newcommand{\WarningTok}[1]{\textcolor[rgb]{0.38,0.63,0.69}{\textbf{\textit{{#1}}}}}


    % Define a nice break command that doesn't care if a line doesn't already
    % exist.
    \def\br{\hspace*{\fill} \\* }
    % Math Jax compatibility definitions
    \def\gt{>}
    \def\lt{<}
    \let\Oldtex\TeX
    \let\Oldlatex\LaTeX
    \renewcommand{\TeX}{\textrm{\Oldtex}}
    \renewcommand{\LaTeX}{\textrm{\Oldlatex}}
    % Document parameters
    % Document title
    \title{MOBCOM-MIdtermF2024}
    
    
    
    
    
    
    
% Pygments definitions
\makeatletter
\def\PY@reset{\let\PY@it=\relax \let\PY@bf=\relax%
    \let\PY@ul=\relax \let\PY@tc=\relax%
    \let\PY@bc=\relax \let\PY@ff=\relax}
\def\PY@tok#1{\csname PY@tok@#1\endcsname}
\def\PY@toks#1+{\ifx\relax#1\empty\else%
    \PY@tok{#1}\expandafter\PY@toks\fi}
\def\PY@do#1{\PY@bc{\PY@tc{\PY@ul{%
    \PY@it{\PY@bf{\PY@ff{#1}}}}}}}
\def\PY#1#2{\PY@reset\PY@toks#1+\relax+\PY@do{#2}}

\@namedef{PY@tok@w}{\def\PY@tc##1{\textcolor[rgb]{0.73,0.73,0.73}{##1}}}
\@namedef{PY@tok@c}{\let\PY@it=\textit\def\PY@tc##1{\textcolor[rgb]{0.24,0.48,0.48}{##1}}}
\@namedef{PY@tok@cp}{\def\PY@tc##1{\textcolor[rgb]{0.61,0.40,0.00}{##1}}}
\@namedef{PY@tok@k}{\let\PY@bf=\textbf\def\PY@tc##1{\textcolor[rgb]{0.00,0.50,0.00}{##1}}}
\@namedef{PY@tok@kp}{\def\PY@tc##1{\textcolor[rgb]{0.00,0.50,0.00}{##1}}}
\@namedef{PY@tok@kt}{\def\PY@tc##1{\textcolor[rgb]{0.69,0.00,0.25}{##1}}}
\@namedef{PY@tok@o}{\def\PY@tc##1{\textcolor[rgb]{0.40,0.40,0.40}{##1}}}
\@namedef{PY@tok@ow}{\let\PY@bf=\textbf\def\PY@tc##1{\textcolor[rgb]{0.67,0.13,1.00}{##1}}}
\@namedef{PY@tok@nb}{\def\PY@tc##1{\textcolor[rgb]{0.00,0.50,0.00}{##1}}}
\@namedef{PY@tok@nf}{\def\PY@tc##1{\textcolor[rgb]{0.00,0.00,1.00}{##1}}}
\@namedef{PY@tok@nc}{\let\PY@bf=\textbf\def\PY@tc##1{\textcolor[rgb]{0.00,0.00,1.00}{##1}}}
\@namedef{PY@tok@nn}{\let\PY@bf=\textbf\def\PY@tc##1{\textcolor[rgb]{0.00,0.00,1.00}{##1}}}
\@namedef{PY@tok@ne}{\let\PY@bf=\textbf\def\PY@tc##1{\textcolor[rgb]{0.80,0.25,0.22}{##1}}}
\@namedef{PY@tok@nv}{\def\PY@tc##1{\textcolor[rgb]{0.10,0.09,0.49}{##1}}}
\@namedef{PY@tok@no}{\def\PY@tc##1{\textcolor[rgb]{0.53,0.00,0.00}{##1}}}
\@namedef{PY@tok@nl}{\def\PY@tc##1{\textcolor[rgb]{0.46,0.46,0.00}{##1}}}
\@namedef{PY@tok@ni}{\let\PY@bf=\textbf\def\PY@tc##1{\textcolor[rgb]{0.44,0.44,0.44}{##1}}}
\@namedef{PY@tok@na}{\def\PY@tc##1{\textcolor[rgb]{0.41,0.47,0.13}{##1}}}
\@namedef{PY@tok@nt}{\let\PY@bf=\textbf\def\PY@tc##1{\textcolor[rgb]{0.00,0.50,0.00}{##1}}}
\@namedef{PY@tok@nd}{\def\PY@tc##1{\textcolor[rgb]{0.67,0.13,1.00}{##1}}}
\@namedef{PY@tok@s}{\def\PY@tc##1{\textcolor[rgb]{0.73,0.13,0.13}{##1}}}
\@namedef{PY@tok@sd}{\let\PY@it=\textit\def\PY@tc##1{\textcolor[rgb]{0.73,0.13,0.13}{##1}}}
\@namedef{PY@tok@si}{\let\PY@bf=\textbf\def\PY@tc##1{\textcolor[rgb]{0.64,0.35,0.47}{##1}}}
\@namedef{PY@tok@se}{\let\PY@bf=\textbf\def\PY@tc##1{\textcolor[rgb]{0.67,0.36,0.12}{##1}}}
\@namedef{PY@tok@sr}{\def\PY@tc##1{\textcolor[rgb]{0.64,0.35,0.47}{##1}}}
\@namedef{PY@tok@ss}{\def\PY@tc##1{\textcolor[rgb]{0.10,0.09,0.49}{##1}}}
\@namedef{PY@tok@sx}{\def\PY@tc##1{\textcolor[rgb]{0.00,0.50,0.00}{##1}}}
\@namedef{PY@tok@m}{\def\PY@tc##1{\textcolor[rgb]{0.40,0.40,0.40}{##1}}}
\@namedef{PY@tok@gh}{\let\PY@bf=\textbf\def\PY@tc##1{\textcolor[rgb]{0.00,0.00,0.50}{##1}}}
\@namedef{PY@tok@gu}{\let\PY@bf=\textbf\def\PY@tc##1{\textcolor[rgb]{0.50,0.00,0.50}{##1}}}
\@namedef{PY@tok@gd}{\def\PY@tc##1{\textcolor[rgb]{0.63,0.00,0.00}{##1}}}
\@namedef{PY@tok@gi}{\def\PY@tc##1{\textcolor[rgb]{0.00,0.52,0.00}{##1}}}
\@namedef{PY@tok@gr}{\def\PY@tc##1{\textcolor[rgb]{0.89,0.00,0.00}{##1}}}
\@namedef{PY@tok@ge}{\let\PY@it=\textit}
\@namedef{PY@tok@gs}{\let\PY@bf=\textbf}
\@namedef{PY@tok@ges}{\let\PY@bf=\textbf\let\PY@it=\textit}
\@namedef{PY@tok@gp}{\let\PY@bf=\textbf\def\PY@tc##1{\textcolor[rgb]{0.00,0.00,0.50}{##1}}}
\@namedef{PY@tok@go}{\def\PY@tc##1{\textcolor[rgb]{0.44,0.44,0.44}{##1}}}
\@namedef{PY@tok@gt}{\def\PY@tc##1{\textcolor[rgb]{0.00,0.27,0.87}{##1}}}
\@namedef{PY@tok@err}{\def\PY@bc##1{{\setlength{\fboxsep}{\string -\fboxrule}\fcolorbox[rgb]{1.00,0.00,0.00}{1,1,1}{\strut ##1}}}}
\@namedef{PY@tok@kc}{\let\PY@bf=\textbf\def\PY@tc##1{\textcolor[rgb]{0.00,0.50,0.00}{##1}}}
\@namedef{PY@tok@kd}{\let\PY@bf=\textbf\def\PY@tc##1{\textcolor[rgb]{0.00,0.50,0.00}{##1}}}
\@namedef{PY@tok@kn}{\let\PY@bf=\textbf\def\PY@tc##1{\textcolor[rgb]{0.00,0.50,0.00}{##1}}}
\@namedef{PY@tok@kr}{\let\PY@bf=\textbf\def\PY@tc##1{\textcolor[rgb]{0.00,0.50,0.00}{##1}}}
\@namedef{PY@tok@bp}{\def\PY@tc##1{\textcolor[rgb]{0.00,0.50,0.00}{##1}}}
\@namedef{PY@tok@fm}{\def\PY@tc##1{\textcolor[rgb]{0.00,0.00,1.00}{##1}}}
\@namedef{PY@tok@vc}{\def\PY@tc##1{\textcolor[rgb]{0.10,0.09,0.49}{##1}}}
\@namedef{PY@tok@vg}{\def\PY@tc##1{\textcolor[rgb]{0.10,0.09,0.49}{##1}}}
\@namedef{PY@tok@vi}{\def\PY@tc##1{\textcolor[rgb]{0.10,0.09,0.49}{##1}}}
\@namedef{PY@tok@vm}{\def\PY@tc##1{\textcolor[rgb]{0.10,0.09,0.49}{##1}}}
\@namedef{PY@tok@sa}{\def\PY@tc##1{\textcolor[rgb]{0.73,0.13,0.13}{##1}}}
\@namedef{PY@tok@sb}{\def\PY@tc##1{\textcolor[rgb]{0.73,0.13,0.13}{##1}}}
\@namedef{PY@tok@sc}{\def\PY@tc##1{\textcolor[rgb]{0.73,0.13,0.13}{##1}}}
\@namedef{PY@tok@dl}{\def\PY@tc##1{\textcolor[rgb]{0.73,0.13,0.13}{##1}}}
\@namedef{PY@tok@s2}{\def\PY@tc##1{\textcolor[rgb]{0.73,0.13,0.13}{##1}}}
\@namedef{PY@tok@sh}{\def\PY@tc##1{\textcolor[rgb]{0.73,0.13,0.13}{##1}}}
\@namedef{PY@tok@s1}{\def\PY@tc##1{\textcolor[rgb]{0.73,0.13,0.13}{##1}}}
\@namedef{PY@tok@mb}{\def\PY@tc##1{\textcolor[rgb]{0.40,0.40,0.40}{##1}}}
\@namedef{PY@tok@mf}{\def\PY@tc##1{\textcolor[rgb]{0.40,0.40,0.40}{##1}}}
\@namedef{PY@tok@mh}{\def\PY@tc##1{\textcolor[rgb]{0.40,0.40,0.40}{##1}}}
\@namedef{PY@tok@mi}{\def\PY@tc##1{\textcolor[rgb]{0.40,0.40,0.40}{##1}}}
\@namedef{PY@tok@il}{\def\PY@tc##1{\textcolor[rgb]{0.40,0.40,0.40}{##1}}}
\@namedef{PY@tok@mo}{\def\PY@tc##1{\textcolor[rgb]{0.40,0.40,0.40}{##1}}}
\@namedef{PY@tok@ch}{\let\PY@it=\textit\def\PY@tc##1{\textcolor[rgb]{0.24,0.48,0.48}{##1}}}
\@namedef{PY@tok@cm}{\let\PY@it=\textit\def\PY@tc##1{\textcolor[rgb]{0.24,0.48,0.48}{##1}}}
\@namedef{PY@tok@cpf}{\let\PY@it=\textit\def\PY@tc##1{\textcolor[rgb]{0.24,0.48,0.48}{##1}}}
\@namedef{PY@tok@c1}{\let\PY@it=\textit\def\PY@tc##1{\textcolor[rgb]{0.24,0.48,0.48}{##1}}}
\@namedef{PY@tok@cs}{\let\PY@it=\textit\def\PY@tc##1{\textcolor[rgb]{0.24,0.48,0.48}{##1}}}

\def\PYZbs{\char`\\}
\def\PYZus{\char`\_}
\def\PYZob{\char`\{}
\def\PYZcb{\char`\}}
\def\PYZca{\char`\^}
\def\PYZam{\char`\&}
\def\PYZlt{\char`\<}
\def\PYZgt{\char`\>}
\def\PYZsh{\char`\#}
\def\PYZpc{\char`\%}
\def\PYZdl{\char`\$}
\def\PYZhy{\char`\-}
\def\PYZsq{\char`\'}
\def\PYZdq{\char`\"}
\def\PYZti{\char`\~}
% for compatibility with earlier versions
\def\PYZat{@}
\def\PYZlb{[}
\def\PYZrb{]}
\makeatother


    % For linebreaks inside Verbatim environment from package fancyvrb.
    \makeatletter
        \newbox\Wrappedcontinuationbox
        \newbox\Wrappedvisiblespacebox
        \newcommand*\Wrappedvisiblespace {\textcolor{red}{\textvisiblespace}}
        \newcommand*\Wrappedcontinuationsymbol {\textcolor{red}{\llap{\tiny$\m@th\hookrightarrow$}}}
        \newcommand*\Wrappedcontinuationindent {3ex }
        \newcommand*\Wrappedafterbreak {\kern\Wrappedcontinuationindent\copy\Wrappedcontinuationbox}
        % Take advantage of the already applied Pygments mark-up to insert
        % potential linebreaks for TeX processing.
        %        {, <, #, %, $, ' and ": go to next line.
        %        _, }, ^, &, >, - and ~: stay at end of broken line.
        % Use of \textquotesingle for straight quote.
        \newcommand*\Wrappedbreaksatspecials {%
            \def\PYGZus{\discretionary{\char`\_}{\Wrappedafterbreak}{\char`\_}}%
            \def\PYGZob{\discretionary{}{\Wrappedafterbreak\char`\{}{\char`\{}}%
            \def\PYGZcb{\discretionary{\char`\}}{\Wrappedafterbreak}{\char`\}}}%
            \def\PYGZca{\discretionary{\char`\^}{\Wrappedafterbreak}{\char`\^}}%
            \def\PYGZam{\discretionary{\char`\&}{\Wrappedafterbreak}{\char`\&}}%
            \def\PYGZlt{\discretionary{}{\Wrappedafterbreak\char`\<}{\char`\<}}%
            \def\PYGZgt{\discretionary{\char`\>}{\Wrappedafterbreak}{\char`\>}}%
            \def\PYGZsh{\discretionary{}{\Wrappedafterbreak\char`\#}{\char`\#}}%
            \def\PYGZpc{\discretionary{}{\Wrappedafterbreak\char`\%}{\char`\%}}%
            \def\PYGZdl{\discretionary{}{\Wrappedafterbreak\char`\$}{\char`\$}}%
            \def\PYGZhy{\discretionary{\char`\-}{\Wrappedafterbreak}{\char`\-}}%
            \def\PYGZsq{\discretionary{}{\Wrappedafterbreak\textquotesingle}{\textquotesingle}}%
            \def\PYGZdq{\discretionary{}{\Wrappedafterbreak\char`\"}{\char`\"}}%
            \def\PYGZti{\discretionary{\char`\~}{\Wrappedafterbreak}{\char`\~}}%
        }
        % Some characters . , ; ? ! / are not pygmentized.
        % This macro makes them "active" and they will insert potential linebreaks
        \newcommand*\Wrappedbreaksatpunct {%
            \lccode`\~`\.\lowercase{\def~}{\discretionary{\hbox{\char`\.}}{\Wrappedafterbreak}{\hbox{\char`\.}}}%
            \lccode`\~`\,\lowercase{\def~}{\discretionary{\hbox{\char`\,}}{\Wrappedafterbreak}{\hbox{\char`\,}}}%
            \lccode`\~`\;\lowercase{\def~}{\discretionary{\hbox{\char`\;}}{\Wrappedafterbreak}{\hbox{\char`\;}}}%
            \lccode`\~`\:\lowercase{\def~}{\discretionary{\hbox{\char`\:}}{\Wrappedafterbreak}{\hbox{\char`\:}}}%
            \lccode`\~`\?\lowercase{\def~}{\discretionary{\hbox{\char`\?}}{\Wrappedafterbreak}{\hbox{\char`\?}}}%
            \lccode`\~`\!\lowercase{\def~}{\discretionary{\hbox{\char`\!}}{\Wrappedafterbreak}{\hbox{\char`\!}}}%
            \lccode`\~`\/\lowercase{\def~}{\discretionary{\hbox{\char`\/}}{\Wrappedafterbreak}{\hbox{\char`\/}}}%
            \catcode`\.\active
            \catcode`\,\active
            \catcode`\;\active
            \catcode`\:\active
            \catcode`\?\active
            \catcode`\!\active
            \catcode`\/\active
            \lccode`\~`\~
        }
    \makeatother

    \let\OriginalVerbatim=\Verbatim
    \makeatletter
    \renewcommand{\Verbatim}[1][1]{%
        %\parskip\z@skip
        \sbox\Wrappedcontinuationbox {\Wrappedcontinuationsymbol}%
        \sbox\Wrappedvisiblespacebox {\FV@SetupFont\Wrappedvisiblespace}%
        \def\FancyVerbFormatLine ##1{\hsize\linewidth
            \vtop{\raggedright\hyphenpenalty\z@\exhyphenpenalty\z@
                \doublehyphendemerits\z@\finalhyphendemerits\z@
                \strut ##1\strut}%
        }%
        % If the linebreak is at a space, the latter will be displayed as visible
        % space at end of first line, and a continuation symbol starts next line.
        % Stretch/shrink are however usually zero for typewriter font.
        \def\FV@Space {%
            \nobreak\hskip\z@ plus\fontdimen3\font minus\fontdimen4\font
            \discretionary{\copy\Wrappedvisiblespacebox}{\Wrappedafterbreak}
            {\kern\fontdimen2\font}%
        }%

        % Allow breaks at special characters using \PYG... macros.
        \Wrappedbreaksatspecials
        % Breaks at punctuation characters . , ; ? ! and / need catcode=\active
        \OriginalVerbatim[#1,codes*=\Wrappedbreaksatpunct]%
    }
    \makeatother

    % Exact colors from NB
    \definecolor{incolor}{HTML}{303F9F}
    \definecolor{outcolor}{HTML}{D84315}
    \definecolor{cellborder}{HTML}{CFCFCF}
    \definecolor{cellbackground}{HTML}{F7F7F7}

    % prompt
    \makeatletter
    \newcommand{\boxspacing}{\kern\kvtcb@left@rule\kern\kvtcb@boxsep}
    \makeatother
    \newcommand{\prompt}[4]{
        {\ttfamily\llap{{\color{#2}[#3]:\hspace{3pt}#4}}\vspace{-\baselineskip}}
    }
    

    
    % Prevent overflowing lines due to hard-to-break entities
    \sloppy
    % Setup hyperref package
    \hypersetup{
      breaklinks=true,  % so long urls are correctly broken across lines
      colorlinks=true,
      urlcolor=urlcolor,
      linkcolor=linkcolor,
      citecolor=citecolor,
      }
    % Slightly bigger margins than the latex defaults
    
    \geometry{verbose,tmargin=1in,bmargin=1in,lmargin=1in,rmargin=1in}
    
    

\begin{document}
    
    \maketitle
    
    

    
    \begin{longtable}[]{@{}c@{}}
\toprule\noalign{}
Mobile Communication Techniques \\
\midrule\noalign{}
\endhead
\bottomrule\noalign{}
\endlastfoot
Petros Elia, elia@eurecom.fr \\
Midterm Exam \\
November 21st, 2024 \\
Time: 9:00-10:00 \\
\end{longtable}

    \[\text{Instructions}\]

\begin{itemize}
\item
  Exercises fall in categories of 1-point and 2-point exercises.
\item
  Total of 11 ×1 + 2 ×2 = 15 points.
\item
  NOTE!!! The exam will be evaluated, out of 13 points. Any points you
  get beyond 13 points, will be offered as extra bonus.
\item
  Each answer should be clearly written, and the solution should be
  developed in detail.
\item
  Mathematical derivations need to show all steps that lead to the
  answer.
\item
  Complete as many exercises as you can. Don't spend too much time on an
  individual question.
\item
  There is NO penalty for incorrect solutions.
\item
  If in certain cases you are unable to provide rigorous mathematical
  proofs, go ahead and provide intuitive justification of your answers.
  Partial credit will be given.
\item
  Calculators are not allowed.
\item
  You are allowed your class notes and class book.
\end{itemize}

\[\text{Hints - equations - conventions:}\]

\begin{itemize}
\tightlist
\item
  Notation

  \begin{itemize}
  \tightlist
  \item
    SISO = single-input single-output, MISO = multiple-input
    single-output, SIMO = single-input multiple- output, MIMO =
    single-input multiple-output,
  \item
    \(R\) represents the rate of communication in bits per channel use
    (b.p.c.u),
  \item
    \(\rho\) represents the SNR (signal to noise ratio),
  \item
    \(w\) will denote additive noise which will be distributed as a
    circularly symmetric Gaussian random variable
    \(\mathbb{C}\mathcal{N}(0,N_0)\). If \(N_0\) is not specified, then
    set \(N_0 = 1\),
  \item
    \(h_i\) will denote independent fading scalar coefficients which
    will be distributed as circularly symmetric Gaussian random
    variables \(\mathbb{C}\mathcal{N}(0,1)\).
  \end{itemize}
\item
  GOOD LUCK!!
\end{itemize}

    \(\color{orange} \large \textbf{1)}\) (1 point). In a multi-path fading
scenario with delay spread \(6\mu s\) and \(L = 3\) channel taps, what
is the operational bandwidth \(W\)?

    \subsubsection{\texorpdfstring{\textbf{Answer}:}{Answer:}}\label{answer}

Given \(\tau_d = 6 \, \mu s\) and \(L = 3\): - Coherence bandwidth:
\(B_c \approx \frac{1}{\tau_d} = \frac{1}{6 \times 10^{-6}} \approx 166.67 \, \text{kHz}\).
- Operational bandwidth:
\(W \approx L \cdot B_c  = 3 \cdot 166.67 = 500 \, \text{kHz}\).

\(\boxed{W = 500 \, \text{kHz}}\)

    \(\color{orange} \large \textbf{2)}\) (1 point). Imagine a given SNR
equal to \(\rho\), and imagine that we are operating over a
(quasi-static) Rayleigh fading SISO channel. Can you describe a code
that achieves probability of error approximately equal to
\(P_e \approx \rho^{−4}\), and rate equal to \(R = 2\) bpcu.

    \begin{center}\rule{0.5\linewidth}{0.5pt}\end{center}

\subparagraph{\texorpdfstring{\textbf{Solution 1: Repetition Code
(256-QAM)}}{Solution 1: Repetition Code (256-QAM)}}\label{solution-1-repetition-code-256-qam}

\begin{itemize}
\tightlist
\item
  \textbf{Modulation}: 256-QAM (8 bits/symbol)\\
\item
  \textbf{Code}: Repetition factor = 4 (diversity order = 4)\\
\item
  \textbf{Rate}: \(R = \frac{8}{4} = 2 \, \text{bpcu}\)\\
\item
  \textbf{Error Probability}: \(P_e \approx \rho^{-4}\)\\
\item
  \textbf{Complexity}: High due to 256-QAM
\end{itemize}

\subparagraph{\texorpdfstring{\textbf{Solution 2: Rotated Code
(4-QAM)}}{Solution 2: Rotated Code (4-QAM)}}\label{solution-2-rotated-code-4-qam}

\begin{itemize}
\tightlist
\item
  \textbf{Modulation}: 4-QAM (2 bits/symbol)\\
\item
  \textbf{Code}: Rotated constellation across 4 time slots (diversity
  order = 4)\\
\item
  \textbf{Rate}: \(R = 2 \, \text{bpcu}\)\\
\item
  \textbf{Error Probability}: \(P_e \approx \rho^{-4}\)\\
\item
  \textbf{Complexity}: Lower due to 4-QAM
\end{itemize}

\subparagraph{\texorpdfstring{\textbf{Comparison}}{Comparison}}\label{comparison}

\begin{longtable}[]{@{}
  >{\raggedright\arraybackslash}p{(\linewidth - 4\tabcolsep) * \real{0.1942}}
  >{\raggedright\arraybackslash}p{(\linewidth - 4\tabcolsep) * \real{0.3883}}
  >{\raggedright\arraybackslash}p{(\linewidth - 4\tabcolsep) * \real{0.4175}}@{}}
\toprule\noalign{}
\begin{minipage}[b]{\linewidth}\raggedright
Feature
\end{minipage} & \begin{minipage}[b]{\linewidth}\raggedright
Solution 1: Repetition Code (256-QAM)
\end{minipage} & \begin{minipage}[b]{\linewidth}\raggedright
Solution 2: Rotated Code (4-QAM)
\end{minipage} \\
\midrule\noalign{}
\endhead
\bottomrule\noalign{}
\endlastfoot
Modulation & 256-QAM (8 bits/symbol) & 4-QAM (2 bits/symbol) \\
Code Type & Repetition code & Rotated time-diversity code \\
Diversity Order & 4 & 4 \\
Rate & 2 bpcu & 2 bpcu \\
Error Probability & \(P_e \approx \rho^{-4}\) &
\(P_e \approx \rho^{-4}\) \\
Complexity & Higher decoding complexity & Lower decoding complexity \\
\end{longtable}

Both solutions achieve \textbf{rate = 2 bpcu} and \textbf{diversity
order 4}, but the rotated 4-QAM code offers \textbf{lower complexity}.

\paragraph{Note:}\label{note}

\begin{itemize}
\tightlist
\item
  Alamouti Code: Closely related to Solution 2 but typically designed
  for MIMO (2 transmit antennas).
\item
  4-Dimensional Lattice Code: Matches Solution 2 with rotation and
  symbol spreading across multiple time slots. This solution achieves
  diversity order 4, fitting the requirement perfectly.
\end{itemize}

\subparagraph{\texorpdfstring{\textbf{Diversity Order
Overview}}{Diversity Order Overview}}\label{diversity-order-overview}

\begin{itemize}
\tightlist
\item
  \textbf{Definition}:\\
  The number of independent signal paths used to combat fading. Error
  probability decreases as \(P_e \approx \rho^{-d}\), where \(d\) is the
  \textbf{diversity order}.
\end{itemize}

\textbf{Types of Diversity}

\begin{enumerate}
\def\labelenumi{\arabic{enumi}.}
\tightlist
\item
  \textbf{Time Diversity}: Transmit symbols across different time slots
  (e.g., repetition coding).\\
\item
  \textbf{Frequency Diversity}: Transmit across multiple frequencies
  (e.g., OFDM).\\
\item
  \textbf{Space Diversity}: Use multiple antennas (e.g., Alamouti code,
  MIMO).\\
\item
  \textbf{Code Diversity}: Spread symbol components across independent
  channels (e.g., rotated lattice codes).
\end{enumerate}

\textbf{Impact}

\begin{itemize}
\tightlist
\item
  \textbf{Higher diversity order} reduces the likelihood of deep fades
  and improves error performance:\\
  \(P_e \propto \rho^{-d}\)
\end{itemize}

\textbf{Examples}

\begin{enumerate}
\def\labelenumi{\arabic{enumi}.}
\tightlist
\item
  \textbf{Diversity Order 1}: SISO, \(P_e \propto \rho^{-1}\).\\
\item
  \textbf{Order 2}: Alamouti code with 2 antennas,
  \(P_e \propto \rho^{-2}\).\\
\item
  \textbf{Order 4}: Rotated 4-QAM or repetition with 4 paths,
  \(P_e \propto \rho^{-4}\).
\end{enumerate}

Higher diversity increases resilience against fading.

    \(\color{orange} \large \textbf{3)}\) (1 point). How much time diversity
will we get with the following SISO (time-diversity) channel model
\[[y_1 \; y_2 \; y_3] = [h_1u_1 \quad h_2(u_1 + u_2) \quad h_3u_2] + [w_1 \; w_2 \; w_3]\]
where the \(u_1,u_2,u_3\) are independent PAM elements. Justify your
answer.

    \begin{center}\rule{0.5\linewidth}{0.5pt}\end{center}

To determine the \textbf{time diversity} in the given channel model:

\subparagraph{Channel Model}\label{channel-model}

\([y_1 \; y_2 \; y_3] = [h_1u_1 \quad h_2(u_1 + u_2) \quad h_3u_2] + [w_1 \; w_2 \; w_3],\)
where \(u_1, u_2, u_3\) are independent PAM symbols, \(h_1, h_2, h_3\)
are the channel coefficients, and \(w_1, w_2, w_3\) are noise terms.

\subparagraph{Analysis}\label{analysis}

\begin{enumerate}
\def\labelenumi{\arabic{enumi}.}
\tightlist
\item
  \textbf{Definition of Time Diversity}:

  \begin{itemize}
  \tightlist
  \item
    Time diversity is determined by the number of independently faded
    channel coefficients (\(h_1, h_2, h_3\)) that affect the transmitted
    symbols.
  \end{itemize}
\item
  \textbf{Observation of Dependencies}:

  \begin{itemize}
  \tightlist
  \item
    \(y_1\) depends on \(h_1u_1\).
  \item
    \(y_2\) depends on \(h_2(u_1 + u_2)\).
  \item
    \(y_3\) depends on \(h_3u_2\).
  \end{itemize}
\item
  \textbf{Diversity Order}:

  \begin{itemize}
  \tightlist
  \item
    \(u_1\) is present in both \(y_1\) and \(y_2\), thus contributing to
    diversity through \(h_1\) and \(h_2\).
  \item
    \(u_2\) is present in both \(y_2\) and \(y_3\), contributing to
    diversity through \(h_2\) and \(h_3\).
  \end{itemize}
\end{enumerate}

Since \(u_1\) and \(u_2\) are affected by two \textbf{independent
channel coefficients} each, the effective \textbf{time diversity order}
is:

\(\text{Time Diversity Order} = \min(\text{number of independent fades per symbol}) = \boxed{2}.\)

\subparagraph{Justification}\label{justification}

The system achieves a time diversity order of 2 because each transmitted
symbol \(u_1\) and \(u_2\) is observed across two independently faded
channels (\(h_1, h_2\) for \(u_1\); \(h_2, h_3\) for \(u_2\)). The third
symbol \(u_3\) does not contribute additional diversity as it is only
affected by \(h_3\).

    \(\color{orange} \large \textbf{4)}\) (1 point). In a SISO case, what is
the degrees of freedom (DOF) if we have a time-diversity code (spanning
three channel uses) of the form
\(\mathcal{X} = {[u_1 + u_2 \quad u_1 + u\_3 \quad u_2 + u_3]}\) where
the \(u_1,u_2,u_3,u_4\) are independent 16-PAM elements?

    \begin{center}\rule{0.5\linewidth}{0.5pt}\end{center}

\subparagraph{\texorpdfstring{\textbf{Step 1: Code
Setup}}{Step 1: Code Setup}}\label{step-1-code-setup}

The time-diversity code is:

\(\mathcal{X} = [u_1 + u_2, \quad u_1 + u_3, \quad u_2 + u_3]\)

\begin{itemize}
\tightlist
\item
  \(u_1, u_2, u_3, u_4\) are \textbf{independent complex numbers} from
  \textbf{16-PAM}, meaning each has a \textbf{real} and
  \textbf{imaginary} part.
\item
  Since we are now counting \textbf{only the real part}, each complex
  symbol contributes \textbf{1 real degree of freedom}.
\end{itemize}

\subparagraph{\texorpdfstring{\textbf{Step 2: Apply the
Formula}}{Step 2: Apply the Formula}}\label{step-2-apply-the-formula}

The formula is:

\(\text{DOF} = \min\left( \frac{\text{\# of real symbols}}{T}, n_t \right)\)

\begin{itemize}
\tightlist
\item
  \textbf{\# of real symbols}: There are \textbf{3 complex symbols},
  each contributing \textbf{1 real part}. So, the number of real symbols
  is \textbf{3}.
\item
  \textbf{\(T\)}: Number of channel uses = 3
\item
  \textbf{\(n_t\)}: Number of transmit antennas = 1 (SISO)
\end{itemize}

Calculate:

\(\text{DOF} = \min\left( \frac{3}{3}, 1 \right) = \min(1, 1) = 1\)

\subparagraph{\texorpdfstring{\textbf{Step 3: Adjust DOF for Real Parts
Only}}{Step 3: Adjust DOF for Real Parts Only}}\label{step-3-adjust-dof-for-real-parts-only}

Since we are counting only the \textbf{real parts}, the effective real
DOF per channel use is:

\(\text{Real DOF per channel use} = \frac{1}{2} \, \text{(since each complex DOF is split between real and imaginary parts)}\)

\subparagraph{\texorpdfstring{\textbf{Final
Answer:}}{Final Answer:}}\label{final-answer}

The \textbf{real degrees of freedom (DOF)} per channel use in this
time-diversity SISO code is:

\(\boxed{\frac{1}{2} \, \text{real DOF per channel use}}\)

    \(\color{orange} \large \textbf{5)}\) (1 point). For the case of time
diversity in the SISO (quasi-static) fading channel, what is the
advantage and the disadvantage of the repetition code, compared to
uncoded transmission.

    \begin{center}\rule{0.5\linewidth}{0.5pt}\end{center}

\begin{itemize}
\tightlist
\item
  \textbf{Advantage}: Repetition code improves reliability by providing
  diversity gain, reducing the error probability in fading channels.\\
\item
  \textbf{Disadvantage}: It reduces spectral efficiency by lowering the
  transmission rate due to redundant transmissions.
\end{itemize}

    \(\color{orange} \large \textbf{6)}\) (1 point). In a SISO case, what is
the DOF and the rate (in bpcu), of the following time-diversity code
(three channel uses) that takes the form
\(\mathcal{X}= {[u_1 + u_4 \quad u_2 \quad u_1 + u_2 + u_3]}\) where the
\(u_1,u_2,u_3,u_4\) are independent 64-QAM elements?

    \begin{center}\rule{0.5\linewidth}{0.5pt}\end{center}

To analyze the \textbf{Degrees of Freedom (DOF)} and \textbf{rate} for
the given time-diversity code:

\subparagraph{Code Representation}\label{code-representation}

The transmitted codeword over three channel uses is:

\(X = \begin{bmatrix} u_1 + u_4 & u_2 & u_1 + u_2 + u_3 \end{bmatrix},\)

where \(u_1, u_2, u_3, u_4\) are independent symbols from a 64-QAM
constellation.

\subparagraph{\texorpdfstring{1. \textbf{Degrees of Freedom
(DOF):}}{1. Degrees of Freedom (DOF):}}\label{degrees-of-freedom-dof}

\begin{itemize}
\tightlist
\item
  The \textbf{DOF} corresponds to the number of \textbf{independent
  information symbols} transmitted across the given channel uses.
\item
  Here, \(u_1, u_2, u_3, u_4\) are \textbf{independent symbols}, so
  there are \textbf{4 independent symbols} transmitted over \textbf{3
  channel uses}.
\end{itemize}

\(\text{DOF} = \frac{\text{Number of Independent Symbols}}{\text{Number of Channel Uses}} = \boxed{\frac{4}{3}}.\)

\subparagraph{\texorpdfstring{2. \textbf{Rate (in
bpcu):}}{2. Rate (in bpcu):}}\label{rate-in-bpcu}

\begin{itemize}
\tightlist
\item
  Each symbol is from a 64-QAM constellation, which carries
  \(\log_2(64) = 6\) bits per symbol.
\item
  Since 4 symbols are transmitted over 3 channel uses, the rate \(R\)
  is:
\end{itemize}

\(R = \frac{\text{Total Bits Transmitted}}{\text{Number of Channel Uses}} = \frac{4}{3}  \cdot 6 = \boxed{ 8 \, \text{bpcu}}.\)

When applying:

\begin{itemize}
\tightlist
\item
  \textbf{\(n_t\)}: Number of transmit antennas = 1 (SISO)
\end{itemize}

\(\text{DOF} = \min\left( \frac{\text{Number of Independent Symbols}}{\text{Number of Channel Uses}}, n_t \right) = \min\left( \frac{4}{3}, 1 \right)\)
the answer is \(\boxed{1}\)

    \(\color{orange} \large \textbf{7)}\) (1 point). Imagine a SISO channel
model with correlated fading, where the first fading coefficient (first
transmission slot) is \(h_1 = h_1^\prime \times h_2^\prime\), and the
second fading coefficient (second transmission slot) is
\(h_2 = h_2^\prime\), where
\(h_1^\prime, h_2^\prime \sim i.i.d \; \mathbb{C}\mathcal{N}(0,1)\).
What is the maximum diversity we can achieve here?

    \subparagraph{Maximum Diversity:}\label{maximum-diversity}

\begin{itemize}
\tightlist
\item
  Fading coefficients: \(h_1 = h_1' \cdot h_2'\) and \(h_2 = h_2'\).
\item
  Independent components: \(h_1'\) and \(h_2'\)
  (\(\mathbb{C}\mathcal{N}(0,1)\), i.i.d.).
\item
  \textbf{Diversity order} = Number of independent fading coefficients =
  \(\boxed{2}\).
\end{itemize}

\subparagraph{Note:}\label{note}

\begin{itemize}
\tightlist
\item
  If \(h_2'\) is bad everything is bad
\end{itemize}

    \(\color{orange} \large \textbf{8)}\) (1 point). Describe the steps of
converting a binary vector detection problem over a time diversity
fading channel, into a scalar detection problem. Imagine that you are
sending BPSK symbols using a repetition code, and consider
\(\mathbb{C}\mathcal{N}(0,N_0)\) noise.

    \begin{center}\rule{0.5\linewidth}{0.5pt}\end{center}

\subparagraph{\texorpdfstring{\textbf{Steps to Convert to Real Scalar
Detection}}{Steps to Convert to Real Scalar Detection}}\label{steps-to-convert-to-real-scalar-detection}

\begin{enumerate}
\def\labelenumi{\arabic{enumi}.}
\item
  \textbf{Received signal model}:\\
  \(y_i = h_i x + n_i, \quad n_i \sim \mathbb{C}\mathcal{N}(0, N_0)\)
\item
  \textbf{Combine the signals} using maximum ratio combining (MRC):\\
  \(y_{\text{combined}} = \sum_{i=1}^{T} h_i^* y_i = \sum_{i=1}^{T} |h_i|^2 x + \sum_{i=1}^{T} h_i^* n_i\)
\item
  \textbf{Take the real part}:\\
  \(\tilde{y} = \text{Re}(y_{\text{combined}}) = \tilde{h} x + \tilde{n}, \quad \tilde{n} \sim \mathcal{N}(0, N_0 \tilde{h})\)
\item
  \textbf{Decision rule}:\\
  \(\boxed{\hat{x} = \begin{cases} +1, & \text{if } \tilde{y} > 0 \\ -1, & \text{if } \tilde{y} < 0 \end{cases}}\)
\end{enumerate}

This reduces the vector detection problem to \textbf{real scalar
detection}.

    \(\color{orange} \large \textbf{9)}\) (1 point). Consider a deep-space
communications scenario, where the received SNR is equal to 20dB. If you
assume low rate communications, what do you expect the probability of
error to be?

    \subparagraph{\texorpdfstring{\textbf{Step 1: Common Error Probability
Expressions}}{Step 1: Common Error Probability Expressions}}\label{step-1-common-error-probability-expressions}

In certain cases, especially for large SNR in low-rate communication
systems, error probability takes the form of:

\(P_e \approx e^{-\gamma \cdot \text{SNR}}\)

Here: - \(\gamma\) depends on the modulation scheme and coding
structure. - This approximation is typical for systems with
\textbf{diversity}, \textbf{strong coding}, or under certain
approximations (e.g., union bounds for coded error probabilities).

\subparagraph{\texorpdfstring{\textbf{Step 2: When Does
\(e^{-\text{SNR}}\)
Apply?}}{Step 2: When Does e\^{}\{-\textbackslash text\{SNR\}\} Apply?}}\label{step-2-when-does-e-textsnr-apply}

\begin{enumerate}
\def\labelenumi{\arabic{enumi}.}
\tightlist
\item
  \textbf{Coded Systems}:\\
  For strong error-correcting codes, the probability of error often
  decreases exponentially with SNR:
\end{enumerate}

\(P_e \approx e^{-\text{coding gain} \cdot \text{SNR}}\)

\begin{enumerate}
\def\labelenumi{\arabic{enumi}.}
\setcounter{enumi}{1}
\item
  \textbf{Uncoded BPSK in AWGN}:\\
  The \textbf{bit error probability} for uncoded BPSK in AWGN is:
  \(P_b = Q\left(\sqrt{2 \cdot \text{SNR}}\right)\)

  For large SNR, using
  \(Q(x) \approx \frac{1}{\sqrt{2\pi} x} e^{-\frac{x^2}{2}}\):
  \(P_b \approx \frac{1}{\sqrt{2\pi} \cdot \sqrt{2 \cdot \text{SNR}}} \, e^{-\text{SNR}}\)
\end{enumerate}

\subparagraph{\texorpdfstring{\textbf{Step 3: Deep-Space
Scenario}}{Step 3: Deep-Space Scenario}}\label{step-3-deep-space-scenario}

In deep-space communications with \textbf{low-rate transmission},
\textbf{coding} and \textbf{interleaving} are critical for reliability.
In such scenarios, error probability can behave as:
\(P_e \approx e^{-\text{SNR}}\)

This results from: 1. \textbf{Effective coding gain}, which leads to
rapid error decay. 2. \textbf{Low-rate transmissions} (few bits per
channel use), allowing strong robustness against noise.

\subparagraph{\texorpdfstring{\textbf{Step 4:
Application}}{Step 4: Application}}\label{step-4-application}

Given \textbf{SNR = 20 dB} (or \(\text{SNR} = 100\) in linear scale),
if:

\begin{itemize}
\item
  \textbf{SNR in Linear Scale}: Convert \(20 \, \text{dB}\) to linear
  scale:
  \(\boxed{\text{SNR}_{\text{linear}} = 10^{\frac{\text{SNR}_{\text{dB}}}{10}} = 10^{\frac{20}{10}} = 100}.\)
\item
  The \textbf{probability of error} is extremely small:
  \(\boxed{P_e \approx e^{-\text{SNR}} = e^{-100} \approx 3.72 \times 10^{-44}}\)
\end{itemize}

This is consistent with the extremely low error probabilities observed
in such scenarios.

\subparagraph{\texorpdfstring{\textbf{Final
Summary:}}{Final Summary:}}\label{final-summary}

\begin{itemize}
\tightlist
\item
  In deep-space communication with low-rate coding, the error
  probability often follows an \textbf{exponential decay} form:
  \(P_e \approx e^{-\text{SNR}}\)
\item
  For \textbf{SNR = 20 dB (100 linear)}, \(P_e \approx e^{-100}\),
  giving an extremely small error probability, which aligns with robust,
  low-error communications in space missions.
\item
  In deep-space communication with high SNR and low rate, errors are
  nearly negligible.
\end{itemize}

    \(\color{orange} \large \textbf{10)}\) (1 point). What is the
approximate coherence time \(T_c\) in a typical urban wireless network
if you are driving approximately 20 kilometers per hour?

    \begin{center}\rule{0.5\linewidth}{0.5pt}\end{center}

To estimate the \textbf{coherence time} \(T_c\) in a typical urban
wireless network, we use the following formula:
\(T_c \approx \frac{1}{f_d},\) where \(f_d\) is the \textbf{Doppler
spread} given by \(f_d = \frac{v}{\lambda} = \frac{v \cdot f_c}{c}.\)

\subparagraph{\texorpdfstring{\textbf{1. Given
Parameters}:}{1. Given Parameters:}}\label{given-parameters}

\begin{itemize}
\tightlist
\item
  Speed:
  \(v = 20 \, \text{km/h} = \frac{20 \times 1000}{3600} = 5.56 \, \text{m/s}\),
\item
  Carrier frequency:
  \(f_c = 2 \, \text{GHz} = 2 \times 10^9 \, \text{Hz}\) (assumed
  typical urban value),
\item
  Speed of light: \(c = 3 \times 10^8 \, \text{m/s}\).
\end{itemize}

\subparagraph{\texorpdfstring{\textbf{2. Doppler
Spread}:}{2. Doppler Spread:}}\label{doppler-spread}

\(f_d = \frac{v \cdot f_c}{c} = \frac{5.56 \cdot 2 \times 10^9}{3 \times 10^8} = 37.1 \, \text{Hz}.\)

\subparagraph{\texorpdfstring{\textbf{3. Coherence
Time}:}{3. Coherence Time:}}\label{coherence-time}

\(T_c \approx \frac{1}{f_d} = \frac{1}{37.1} \approx 0.027 \, \text{seconds} = 27 \, \text{ms}.\)

The approximate coherence time is: \(\boxed{27 \, \text{ms}}.\)

    \(\color{orange} \large \textbf{11)}\) (1 point). Consider communication
over a SISO fading channel with a delay spread of \(T_d = 3 \mu s\) and
a signal bandwidth of \(W = 1\) MHz. - Write all the received signals,
if we only send \(x[0]\) and then we stop transmitting.

    \begin{center}\rule{0.5\linewidth}{0.5pt}\end{center}

To analyze this scenario, we need to consider the \textbf{SISO fading
channel} with a \textbf{delay spread} \(T_d = 3 \, \mu s\) and a signal
bandwidth \(W = 1 \, \text{MHz}\). The delay spread indicates the
multipath environment, meaning the transmitted signal will arrive at the
receiver through multiple delayed and scaled copies.

\subparagraph{\texorpdfstring{\textbf{1. Transmitted
Signal}:}{1. Transmitted Signal:}}\label{transmitted-signal}

\begin{itemize}
\tightlist
\item
  Only \(x[0]\) is transmitted, then the transmission stops. Thus:
  \(x[n] = \begin{cases} x[0], & \text{if } n = 0, \\ 0, & \text{if } n \neq 0. \end{cases}\)
\end{itemize}

\subparagraph{\texorpdfstring{\textbf{2. Received
Signal}:}{2. Received Signal:}}\label{received-signal}

The received signal is the convolution of the transmitted signal
\(x[n]\) with the channel impulse response \(h(t)\):
\(y[n] = h[n] \ast x[n].\)

\begin{itemize}
\tightlist
\item
  The \textbf{channel impulse response} \(h(t)\) is a sum of \(L\)
  multipath components:
  \(h(t) = \sum_{l=0}^{L-1} h_l \delta(t - \tau_l),\) where:

  \begin{itemize}
  \tightlist
  \item
    \(h_l\): Fading coefficient for the \(l\)-th path
    (\(h_l \sim \mathcal{CN}(0, 1)\)),
  \item
    \(\tau_l\): Delay of the \(l\)-th path (\(0 \leq \tau_l \leq T_d\)).
  \end{itemize}
\item
  With \(T_d = 3 \, \mu s\), the maximum delay is \(3 \, \mu s\),
  corresponding to
  \(L \approx W \cdot T_d = 1 \, \text{MHz} \cdot 3 \, \mu s = \boxed{ 3 }\)
  significant paths.
\end{itemize}

\subparagraph{\texorpdfstring{\textbf{3. Writing the Received
Signals}:}{3. Writing the Received Signals:}}\label{writing-the-received-signals}

For \(x[0]\) transmitted: - The received signal \(y[n]\) consists of
\(L\) delayed copies of \(x[0]\), weighted by the fading coefficients
\(h_l\): \(y[0] = h_0 x[0],\) \(y[1] = h_1 x[0],\) \(y[2] = h_2 x[0].\)
- For \(n > 2\), no further contributions occur, as \(\tau_l \leq T_d\).

Thus:
\(y[n] = \boxed{ \begin{cases} h_0 x[0], & n = 0, \\ h_1 x[0], & n = 1, \\ h_2 x[0], & n = 2, \\ 0, & n > 2. \end{cases}}\)

\subparagraph{\texorpdfstring{\textbf{Final
Answer}:}{Final Answer:}}\label{final-answer}

The received signals are:
\(y[0] = h_0 x[0], \quad y[1] = h_1 x[0], \quad y[2] = h_2 x[0], \quad y[n] = 0 \, \text{for } n > 2.\)

    \(\color{orange} \large \textbf{12)}\) \textbf{\emph{(2 points)}}. What
is the optimal diversity order over a 2×1 MISO channel
\(h = [h_1 \; h_2], h_i \sim i.i.d \; \mathbb{C}\mathcal{N}(0,1)\)? - In
the same channel as above (again with no time diversity), consider a
space time code whose matrices take the form \[\begin{bmatrix}
x_0 & x_1 \\
x_1 & x_0
\end{bmatrix}
\]where the\(x_i\) are drawn independently from a QAM constellation.
Will this code achieve optimal diversity order? (argue why or why not) -
What is the diversity order achieved by the Alamouti code, over this
\(2 \times 1\) MISO channel? (again, you can just argue in words)

    \begin{center}\rule{0.5\linewidth}{0.5pt}\end{center}

\subparagraph{\texorpdfstring{\textbf{1. Optimal Diversity Order in a
\(2 \times 1\) MISO
Channel}}{1. Optimal Diversity Order in a 2 \textbackslash times 1 MISO Channel}}\label{optimal-diversity-order-in-a-2-times-1-miso-channel}

In a \(2 \times 1\) MISO channel, the \textbf{diversity order} is equal
to the number of independent fading paths, which corresponds to the
number of transmit antennas (\(N_t = 2\)) when there is 1 receive
antenna.\\
Thus, the \textbf{optimal diversity order} is: \(\boxed{2}.\)

\subparagraph{\texorpdfstring{\textbf{2. Diversity Order of the Given
Space-Time
Code}}{2. Diversity Order of the Given Space-Time Code}}\label{diversity-order-of-the-given-space-time-code}

The given code matrix is:
\(\mathbf{X} = \begin{bmatrix} x_0 & x_1 \\ x_1 & x_0 \end{bmatrix},\)
where \(x_0\) and \(x_1\) are independent QAM symbols.

\textbf{Key Analysis}:

\begin{itemize}
\tightlist
\item
  \textbf{Rank Criterion}: For a space-time code to achieve full
  diversity, the difference between any two distinct code matrices
  \(\mathbf{X}_1\) and \(\mathbf{X}_2\) must result in a matrix of full
  rank.
\item
  For this code:
  \(\Delta\mathbf{X} = \mathbf{X}_1 - \mathbf{X}_2 = \begin{bmatrix} x_{01} - x_{02} & x_{11} - x_{12} \\ x_{11} - x_{12} & x_{01} - x_{02} \end{bmatrix}.\)

  \begin{itemize}
  \tightlist
  \item
    The rows of \(\Delta\mathbf{X}\) are \textbf{linearly dependent}
    because the two rows are identical. This means \(\Delta\mathbf{X}\)
    is \textbf{not full rank}.
  \end{itemize}
\end{itemize}

\textbf{Conclusion}:

This code does \textbf{not achieve the optimal diversity order}, as it
does not satisfy the rank criterion for full diversity.

\subparagraph{\texorpdfstring{\textbf{3. Diversity Order of the Alamouti
Code}}{3. Diversity Order of the Alamouti Code}}\label{diversity-order-of-the-alamouti-code}

The Alamouti code for a \(2 \times 1\) MISO channel is:
\(\mathbf{X}_{\text{Alamouti}} = \begin{bmatrix} x_0 & -x_1^* \\ x_1 & x_0^* \end{bmatrix}.\)

\textbf{Key Features}:

\begin{itemize}
\tightlist
\item
  The Alamouti code satisfies the \textbf{rank criterion}, ensuring that
  \(\Delta\mathbf{X} = \mathbf{X}_1 - \mathbf{X}_2\) is always full rank
  for distinct codewords \(\mathbf{X}_1\) and \(\mathbf{X}_2\).
\item
  Each transmitted symbol experiences the full diversity of the channel,
  as it leverages both transmit antennas.
\end{itemize}

\textbf{Conclusion}:

The Alamouti code achieves the \textbf{optimal diversity order of 2}
over the \(2 \times 1\) MISO channel.

\subparagraph{\texorpdfstring{\textbf{Final
Answers}:}{Final Answers:}}\label{final-answers}

\begin{enumerate}
\def\labelenumi{\arabic{enumi}.}
\tightlist
\item
  Optimal diversity order in \(2 \times 1\) MISO: \(\boxed{2}\).
\item
  Given space-time code: \textbf{Does not achieve optimal diversity
  order} due to \(\boxed{\text{lack of full-rank}}\) property.
\item
  Alamouti code: \textbf{Achieves optimal diversity order of
  \(\boxed{2}\)}.
\end{enumerate}

    \(\color{orange}\large\textbf{13)}\) \textbf{\emph{(EXTRA CREDIT: 2
points)}}. Consider a setting where the transmit antenna array has
length of 50 cm, the received antenna array has size 20cm, the
transmission frequency is 1000 MHz, the signal bandwidth is 1 MHz, the
channel coherence time is \(T_c = 21\) ms, and the coding duration is
\(T_{coding} = 7\)ms. - How much diversity can you get, in total?

    \subparagraph{\texorpdfstring{\textbf{Explanation for selecting Space
Diversity}}{Explanation for selecting Space Diversity}}\label{explanation-for-selecting-space-diversity}

\begin{enumerate}
\def\labelenumi{\arabic{enumi}.}
\tightlist
\item
  \textbf{Only Space Diversity is usable}:

  \begin{itemize}
  \tightlist
  \item
    \textbf{Time diversity}: Not applicable since
    \(T_{coding} = 7 \, \text{ms}\) is much shorter than
    \(T_c = 21 \, \text{ms}\), so the channel does not change
    significantly.
  \item
    \textbf{Frequency diversity}: Not effective as the bandwidth (1 MHz)
    is within the coherence bandwidth.
  \end{itemize}
\item
  \textbf{Calculate Space Diversity}:

  \begin{itemize}
  \tightlist
  \item
    \textbf{Wavelength}: \(\lambda = 0.3 \, \text{m}\) (at 1000 MHz)
  \item
    \textbf{Antenna Spacing}:
    \(d = \frac{\lambda}{2} = 0.15 \, \text{m}\)
  \item
    \textbf{Transmit Array}:\\
    \(n_t = \frac{50 \, \text{cm}}{15 \, \text{cm}} \approx 4\)
  \item
    \textbf{Receive Array}:\\
    \(n_r = \frac{20 \, \text{cm}}{15 \, \text{cm}} \approx 3\)
  \end{itemize}
\item
  \textbf{Total Space Diversity}:
  \(\text{Space diversity} = n_t \times n_r = 4 \times 3 = 12\)
\end{enumerate}

\subparagraph{\texorpdfstring{\textbf{Final
Answer}:}{Final Answer:}}\label{final-answer}

\(\boxed{\text{diversity} = 12}\)


    % Add a bibliography block to the postdoc
    
    
    
\end{document}
